\documentclass[11 pt, letter]{article}

\usepackage[letterpaper,right=1in,left=1in,top=1in,bottom=1in]{geometry}
\usepackage{ae} % or {zefonts}
\usepackage[T1]{fontenc}
\usepackage[ansinew]{inputenc}
\usepackage{amsmath}
\usepackage{amssymb}
\usepackage{url}
\usepackage{setspace} %allows to change linespacing
%\usepackage{dashrule} %allows ro draw dotted lines
\usepackage{color}   % allows usage of color fonts
\definecolor{light-gray}{gray}{0.65} % defines color and its name
\definecolor{NavyBlue}{rgb}{0.,0,0.5}
\definecolor{med-gray}{gray}{0.5} % defines color and its name

% avoid clubs and widows
\clubpenalty=10000 \widowpenalty=10000
% \displaywidowpenalty=10000

\parindent=0mm  % removes indent at start of paragraphs

\setstretch{1} %linespacing
%\onehalfspacing
%\doublespacing

%%%%%%%%%%%%%%%%%%%%%%%%%%%%%%%%%%%%%%%%%%%
%%  SET  LONG  OR  SHORT  VERSION OF CV  %%
%%%%%%%%%%%%%%%%%%%%%%%%%%%%%%%%%%%%%%%%%%%
\usepackage{ifthen}
\newboolean{longcv}
\setboolean{longcv}{true}  % switch true, false, to produce long, short version of cv
\newboolean{includeDictamenes}
\setboolean{includeDictamenes}{false}
\newcommand{\condcomment}[2]{\ifthenelse{#1}{#2}{}}
%  para usarlo se pone
%  \condcomment{\boolean{longcv}}{
%   ...
%  }

%FOR SPANISH FORMATTING (HYPHENATION ETC.)
\usepackage[spanish]{babel}

% Allows to manually change space between and above items (pero jode mi environment citasmitrabajo)
%\usepackage{enumitem} %% p.ej. \begin{itemize}[topsep=-3pt, itemsep=-3pt]

% Used to cross-reference (\ref) the \label from another tex file (in this case, to import ncites from citasEMM.tex)
\usepackage{xr}
\externaldocument{citasEMM}

%% Drops space between and above items and enumerates
\let\olditemize=\itemize
\let\endolditemize=\enditemize
\renewenvironment{itemize}{%
    \vspace*{-1.5\parsep}%
    \begin{olditemize}%
      \setlength{\parskip}{0.1\parskip}%
      \setlength{\itemsep}{0.1\itemsep}%
  }%
  {%
    \end{olditemize}%
  }
%\let\oldenumerate=\enumerate
%\let\endoldenumerate=\endenumerate
%\renewenvironment{enumerate}{%
%    \vspace*{-1.5\parsep}%
%    \begin{oldenumerate}%
%      \setlength{\parskip}{0.1\parskip}%
%      \setlength{\itemsep}{0.1\itemsep}%
%  }%
%  {%
%    \end{oldenumerate}%
%  }

% MY OWN ENUMERATES WITH SPECIAL LABEL AN NO SPACING --- redundante desde que pas\'e citas a citasEMM.tex
%\newenvironment{CitasMiTrabajo}{
%    \begin{tiny}
%    \begin{color}{med-gray}
%    \begin{enumerate}
%        \renewcommand\theenumi{\tiny{\emph{cita \arabic{enumi}}}}
%        \setlength{\itemsep}{.1\itemsep}
%        \setlength{\parskip}{.1\parskip}
%    }{\end{enumerate}\end{color}\end{tiny}}

\usepackage[colorlinks=true,urlcolor=NavyBlue]{hyperref}

\usepackage{Tabbing} % allows accented letters (e.g. \'a) in tabbing environment without corrupting. \=, \+, and \> become \TAB=, \TAB+, and \TAB=. See http://cs.brown.edu/about/system/software/latex/doc/Tabbing.pdf

\usepackage{mathptmx}           % set font type to Times
\usepackage[scaled=.90]{helvet} % set font type to Times (Helvetica for some special characters)
\usepackage{courier}            % set font type to Times (Courier for other special characters)

\begin{document}

\selectlanguage{spanish}
%Makes itemized lists look as in English despite language=Spanish
\renewcommand{\labelitemi}{\textbullet}
\renewcommand{\labelitemii}{\normalfont\bfseries\textendash}
\renewcommand{\labelitemiii}{$\star$}
\renewcommand{\labelitemiv}{\textperiodcentered}

\textbf{\LARGE{Eric Magar}} \\
\emph{\Large{curriculum vitae}}   \\ [-2ex]
\makebox[\textwidth][r]{\hrulefill
\textcolor{light-gray}{\tiny{\texttt{~puesto al d\'ia el \today, actua-}}}} \\ [-1.5ex]
\makebox[\textwidth][r]{
\textcolor{light-gray}{\tiny{\texttt{lizaciones en \href{https://emagar.github.io/cv/}{http://emagar.github.io/cv/}}}}}

\medskip

\begin{Tabbing}

\textbf{Puesto actual:} \TAB= Profesor, Departamento de Ciencia Pol\'itica, ITAM, M\'exico. \\
\\
\textbf{Contacto:} \TAB= \TAB+ ITAM-Ciencia Pol\'itica, R\'io Hondo 1, Col.\ Tizap\'an San \'Angel, 01000 M\'exico DF. \\ Tel. +52(55)5628-4079, Fax +52(55)5490-4672. \href{mailto:emagar@itam.mx}{\small{\nolinkurl{emagar@itam.mx}}}. \TAB- \\
\\
\textbf{Datos personales:} \TAB= \TAB+ Ciudadano mexicano. Nacido el 12 de enero de 1970 en M\'exico DF. \\ Casado, padre de Aurelia y Le\'on Mart\'in. \TAB- \\
\\
\textbf{Perfil en Google Scholar:} \TAB= \TAB+ \href{http://scholar.google.com.mx/citations?user=Rs7pVXQAAAAJ}{\small{\nolinkurl{http://scholar.google.com.mx/citations?user=Rs7pVXQAAAAJ}}} \TAB- \\
\\
\textbf{ORCID:} \TAB= \TAB+ \href{https://orcid.org/0000-0002-6766-1677}{0000-0002-6766-1677} \TAB- \\
\\
\textbf{Web of Science ResearcherID:} \TAB= \TAB+ S-6603-2019 \TAB- \\
\\
\textbf{P\'agina web:} \TAB= \TAB+ \href{http://ericmagar.com}{\small{\nolinkurl{http://ericmagar.com}}} \TAB- \\
\\
\textbf{Citas documentadas:} \TAB= \TAB+ \href{https://github.com/emagar/cv/blob/master/citasEMM.pdf}{obtenga el documento} \TAB- \\

\end{Tabbing}

% \textbf{Puesto actual:} Profesor, Departamento de Ciencia Pol\'itica, ITAM, M\'exico. \\
% \\
% \textbf{Contacto:} ITAM-Ciencia Pol\'itica, R\'io Hondo 1, Col.\ Tizap\'an San \'Angel, 01000 M\'exico DF. \\ Tel. +52(55)5628-4079, Fax +52(55)5490-4672. \href{mailto:emagar@itam.mx}{\small{\nolinkurl{emagar@itam.mx}}}. \\
% \\
% \textbf{Datos personales:} Ciudadano mexicano. Nacido el 12 de enero de 1970 en M\'exico DF. \\ Casado, una hija. \\
% \\
% \textbf{Perfil en Google Scholar:} \href{http://scholar.google.com.mx/citations?user=Rs7pVXQAAAAJ}{\small{\nolinkurl{http://scholar.google.com.mx/citations?user=Rs7pVXQAAAAJ}}} \\

\bigskip

\section{T\'itulos}

\begin{itemize}

\item Licenciado en ciencia pol\'itica (1994, Instituto Tecnol\'ogico Aut\'onomo de M\'exico).

\item Doctor en ciencia pol\'itica (2001, Universidad de California, San Diego). \\ \condcomment{\boolean{longcv}}{

\end{itemize}



\section{Educaci\'on}

\begin{itemize}

\item 1995--2001, Doctorado en Ciencia Pol\'itica,
    Departamento de Ciencia Pol\'itica,
    Universidad de California, San Diego (UCSD).

    \begin{itemize}

    \item} \textbf{Tesis doctoral:} ``\href{http://sites.google.com/site/emagar/MagarDiss.pdf}
    {Bully Pulpits: Posturing, Bargaining, and Polarization in the Legislative Process
    of the Americas}'' co-asesorada por Gary W.\ Cox y Paul W.\ Drake. \condcomment{\boolean{longcv}}{\textbf{\color{red}\ref{ncites:magar.2001} citas documentadas}.}
        %(La \'ultima p\'agina de este CV contiene una sinopsis de mi tesis.)

\condcomment{\boolean{longcv}}{

    \end{itemize}

\item 1989--1994, Licenciatura en Ciencia Pol\'itica,
    Departamento Acad\'emico de Ciencias Sociales,
    Instituto Tecnol\'ogico Aut\'onomo de M\'exico (ITAM).}

\end{itemize}


\section{\'Areas de investigaci\'on}

\begin{itemize}

\item Instituciones democr\'aticas comparadas; proceso legislativo; elecciones y sistemas electorales; teor\'ia formal; estad\'istica avanzada.

\end{itemize}



\condcomment{\boolean{longcv}}{

\section{Otros estudios}

\begin{itemize}

\item Verano 1992, Cursos de an\'alisis de regresi\'on linear y teor\'ia de juegos,
    Inter-university Consortium for Political and Social Research (ICPSR),
    Universidad de Michigan, Ann Arbor.

\end{itemize}
}




\section{Empleos}

\begin{itemize}

\item 2014--15, Investigador visitante, \href{https://polisci.wustl.edu/}{Department of Political Science}, Washington University in St.\ Louis.

\item 2000--a la fecha, Profesor de Tiempo Completo, \href{http://politica.itam.mx/}{Departamento de Ciencia Pol\'itica}, ITAM.

\item 2007, Profesor Visitante, \href{http://www.ibei.org}{Institut Barcelona d'Estudis Internacionals}, Barcelona, Espa\~na.

\condcomment{\boolean{longcv}}{\item 2007, Profesor Visitante,
Departamento de Ciencias Pol\'iticas y Sociales, Universidat Pompeu
Fabra, Barcelona, Espa\~na.}

\item 2004--2007, Jefe del \href{http://politica.itam.mx/}{Departamento de Ciencia Pol\'itica}, ITAM.

\item 2000--2003, Director de la licenciatura en Ciencia Pol\'itica, ITAM.

\condcomment{\boolean{longcv}}{\item 1998--2000, Asistente de
investigaci\'on, de docencia, y editorial para diversos profesores,
\href{http://polisci.ucsd.edu/}{Departamento de Ciencia Pol\'itica},
UCSD.}

\condcomment{\boolean{longcv}}{\item 1995--1998, Asistente de
investigaci\'on, \href{http://usmex.ucsd.edu/}{Center for
US-Mexican Studies}, UCSD.}

\item 1994, Asesor del Director Ejecutivo de Prerrogativas y Partidos Pol\'iticos, \href{http://www.ife.org.mx}{Instituto Federal Electoral}, M\'exico DF.

\condcomment{\boolean{longcv}}{\item 1993, Asistente de
investigaci\'on para diversos profesores, Depto.\ de Ciencias
Sociales, ITAM.}

\condcomment{\boolean{longcv}}{\item 1992, Miembro de la Facultad Menor de Ciencias Sociales, ITAM.}

\end{itemize}





\section{Premios y reconocimientos}

\begin{itemize}
\item 2014, Premio al M\'erito Profesional en el Sector Acad\'emico, ex-ITAM. 

\item 2014, SSRN Top Ten Paper Award (``Effects of automated redistricting'').

\item 2011, SSRN Top Ten Paper Award (``Factions with clout'').

\item 2010, SSRN Top Ten Paper Award (``Partisanship among the Experts'').

\item 2010, SSRN Top Ten Paper Award (``Legalist vs.\ Interpretativist'').

\item 2009, SSRN Top Ten Paper Award (``Factions with clout'').

\item 2008, SSRN Top Ten Paper Award (``Gubernatorial coattails'').

\item 1 ene.\ 2017--31 dic.\ 2020, Investigador Nivel II del SNI\condcomment{\boolean{longcv}}{ (expediente 25965, dictamen del 9 de sep., 2016)}.

\item 1 ene.\ 2013--31 dic.\ 2016, Investigador Nivel I del \href{http://www.conacyt.gob.mx/SNI/_layouts/xlviewer.aspx?id=/SNI/Documents/VIGENTES_SNI2014.xlsx}{SNI}\condcomment{\boolean{longcv}}{ (expediente 25965, dictamen del 2 de sep., 2012)}.

\item 1 ene.\ 2009--31 dic.\ 2012, Investigador Nivel I del SNI\condcomment{\boolean{longcv}}{ (expediente 25965, dictamen del 1 de sep., 2008)}.

\item 1 ene.\ 2006--31 dic.\ 2008, Investigador Nivel I del SNI\condcomment{\boolean{longcv}}{ (dictamen del 15 de ago.\ 2005)}.

\item 1 jul.\ 2002--30 jun.\ 2005, Candidato a Investigador del SNI\condcomment{\boolean{longcv}}{ (dictamen del 1 de jul.\ 2002)}.

\item 2001, The \href{http://polisci.ucsd.edu/about/recognition.html}{Peggy Quon Prize} for distinguished research contribution to political science,
    Depto.\ de Ciencia Pol\'itica, UCSD.

\item 1994, Premio Los Mejores Estudiantes de M\'exico,
    \condcomment{\boolean{longcv}}{Ateneo Nacional de Ciencia y Tecnolog\'ia,} \emph{El Diario de M\'exico}.

\item 1994, Premio a la Excelencia Acad\'emica Miguel Palacios Macedo,
    ITAM.

\item 1994, \href{http://politica.itam.mx/exalumnos/documentos/ex-alumnos_cp.pdf}{Menci\'on honor\'ifica} en el examen profesional,
    ITAM.

%\condcomment{\boolean{longcv}}{\item 1988,   Mention Assez-Bien por las calificaciones obtenidas en los
%    ex\'amenes del Baccalaur\'eat,
%    \href{http://www.lfm.edu.mx/}{Liceo Franco Mexicano}, M\'exico D.F.}

\end{itemize}




\section{Becas recientes}

\begin{itemize}

\item 2014--15, Beca Fulbright-Garc\'ia Robles para Estancias de Investigaci\'on en Estados Unidos.
    \condcomment{\boolean{longcv}}{US\$24,500.}

\item 2008--a la fecha, Beca anual de investigaci\'on de la Divisi\'on de Econom\'ia, Derecho y Ciencias Sociales, ITAM.
    \condcomment{\boolean{longcv}}{US\$5,000 anuales.}

\item 2005, Beca para investigar la representaci\'on de mujeres en la C\'amara de Diputados de M\'exico entre 1994 y 2003
    \condcomment{\boolean{longcv}}{(proyecto ``�Qu\'e hace falta para que haya
    m\'as mujeres en el gobierno mexicano?  Determinantes del acceso de mujeres a las
    candidaturas y los puestos de elecci\'on en la C\'amara de Diputados Federal, 1994--2003'')},
    \href{http://www.inmujeres.gob.mx/}{INMUJERES} y \href{http://www.conacyt.mx/}{CONACYT}.
    \condcomment{\boolean{longcv}}{Mx\$232,000.}

\item 2004--2006, Beca anual de investigaci\'on de la Divisi\'on de Econom\'ia, Derecho y Ciencias Sociales, ITAM.
    \condcomment{\boolean{longcv}}{US\$5,000 anuales.}

\condcomment{\boolean{longcv}}{\item 1999, Dean's Social Science
    Travel Research Fund Grant, \href{http://weber.ucsd.edu/}{Division of Social Sciences}, UCSD.
    US\$1,500.

\item 1999, \href{http://cilas.ucsd.edu/}{Center for Iberian and Latin American Studies} Field Research Grant, UCSD.
    US\$1,500.

\item 1998, \href{http://icenter.ucsd.edu/pao/friends.htm}{Friends of the International Center scholarship}, UCSD.
    US\$1,000.

\item 1995--1998, Beca-cr\'edito CONACYT para estudios de posgrado
    (media beca de manutenci\'on sin beca para la colegiatura),
    \href{http://www.conacyt.mx/}{Consejo Nacional de Ciencia y Tecnolog\'ia}, M\'exico DF.
    US\$21,390.

\item 1995--1997, Ford-McArthur scholarship for graduate studies
    (beca parcial de manutenci\'on),
    \href{http://www.iie.org/}{Institute of International Education}, New York.
    US\$12,400.}

\end{itemize}

\condcomment{\boolean{longcv}}{\medskip \makebox[10mm]{} \emph{Total
$\approx$ US\$91,000}.}
% El 24 ago 2010 eran 86310 con tipo de cambio 12.5




\section{Publicaciones}

\subsection{En revistas arbitradas}

    \begin{itemize}

    \item \href{https://ssrn.com/abstract=3514866}{``Presidents on the Fast Track: Fighting Floor Amendments with Restrictive Rules''} con Valeria Palanza y Gisela Sin, \emph{Journal of Politics} (aceptado en junio 2019).% \textbf{\color{red}\ref{ncites:magar-palanza-sin2020jop} citas documentadas}}.

    \item \href{http://cienciassociales.edu.uy/institutodecienciapolitica/rucp-vol-26-no-1/}{``La partici\'on de un cartel: votaciones nominales y guerra faccional en la Asamblea Legislativa del Distrito Federal''}, \emph{Revista Uruguaya de Ciencia Pol\'itica}, vol.~26, n\'um.~1 (2017) pp.~35--58\condcomment{\boolean{longcv}}{ (ISSN:0797-9789)}.% \textbf{\color{red}\ref{ncites:magar2017.rucp} citas documentadas}}.
      
    \item \href{https://github.com/emagar/cv/blob/master/papers/magar.etalComponentsPartisanBiasMultiparty2017pg.pdf}{``Components of partisan bias originating from single-member districts in multi-party systems: The case of Mexico''} with Alejandro Trelles, Micah Altman, and Michael P.\ McDonald, \emph{Political Geography} vol.\ 57, March 2017, pp.~1--12\condcomment{\boolean{longcv}}{ (ISSN:0962-6298, \url{http://dx.doi.org/10.1016/j.polgeo.2016.11.015}). \textbf{\color{red}\ref{ncites:magar.etal.2017.polgeo} cita documentada}}.
% DOI: http://dx.doi.org/10.1016/j.polgeo.2016.11.015
% ISSN:0962-6298

    \item \href{http://www.politicaygobierno.cide.edu/index.php/pyg/article/view/825/606}{``Datos abiertos, transparencia y redistritaci\'on en M\'exico''} con Alejandro Trelles, Micah Altman y Michael P.\ McDonald, \href{http://www.politicaygobierno.cide.edu/}{\emph{Pol\'itica y Gobierno}} vol.\ 23, n\'um.\ 2, $2^{do}$ semestre 2016, pp.\ 331--64\condcomment{\boolean{longcv}}{ (ISSN:1665-2037)}. Tambi\'en apareci\'o traducido al ingl\'es \href{http://www.politicaygobierno.cide.edu/index.php/pyg/article/view/822/614}{aqu\'i}. 

    \item \href{http://www.politicaygobierno.cide.edu/index.php/pyg/article/view/18}{``Consideraciones metodol\'ogicas para estudiantes de pol\'itica legislativa mexicana: sesgo por selecci\'on en votaciones nominales''} con Francisco Cant\'u y Scott Desposato, \emph{Pol\'itica y Gobierno} vol.\ 21, n\'um.\ 1, $1^{er}$ semestre 2014, pp.\ 25--53\condcomment{\boolean{longcv}}{ (ISSN:1665-2037). \textbf{\color{red}\ref{ncites:cantuDesMagar2014} citas documentadas}}. 

    \item \href{https://github.com/emagar/cv/blob/master/papers/magarGuberCoattails2012jop.pdf} {``Gubernatorial Coattails in Mexican Congressional Elections''} \href{http://www.journalofpolitics.org/}{\emph{The Journal of Politics}} vol.\ 74, n\'um.\ 2, abr.\ 2012, pp.\ 383--99\condcomment{\boolean{longcv}}{ (ISSN:0022-3816)}. \condcomment{\boolean{longcv}}{\textbf{\color{red}\ref{ncites:magar.gubCoatMx.2012} citas documentadas.}}%\\ [-25pt]
% DOI: http://dx.doi.org/10.1017/S0022381611001629

    \item \href{https://github.com/emagar/cv/blob/master/papers/magarMoraes-Factions2012pp.pdf} {``Factions with clout: presidential cabinet coalition and policy in the Uruguayan Parliament''} con Juan Andr\'es Moraes, \href{http://ppq.sagepub.com/cgi/content/abstract/18/3/427}{\emph{Party Politics}} vol.\ 18, n\'um.\ 3 (mayo 2012) pp.\ 427--51\condcomment{\boolean{longcv}}{ (ISSN:1354-0688)}. \condcomment{\boolean{longcv}}{\textbf{\color{red}\ref{ncites:magar.moraes.2012pp} citas documentadas.}} % http://papers.ssrn.com/abstract=1328668
% DOI: http://dx.doi.org/10.1177/1354068810377460

    \item \href{http://www.fcs.edu.uy/icp/downloads/revista/RUCP17/RUCP-17-02_Magar&Moraes.pdf} {``Coalici\'on y resultados: aprobaci\'on y duraci\'on del tr\'amite parlamentario en Uruguay (1985--2000)''} con Juan Andr\'es Moraes, \href{http://www.fcs.edu.uy/icp/revista.htm}{\emph{Revista Uruguaya de Ciencia Pol\'itica}} vol.\ 17, n\'um.\ 1 (ene.-dic.\ 2008), pp.\ 39--70\condcomment{\boolean{longcv}}{ (ISSN:0797-9789). \textbf{\color{red}\ref{ncites:magar.moraes.2008.rucp} citas documentadas}}.
    %http://www.fcs.edu.uy/icp/downloads/revista/RUCP16/RevistaICP17-02.pdf

    \item \href{https://github.com/emagar/cv/blob/master/papers/estevezMagarRosasIfeElecStud2008.pdf} {``Partisanship in Non-Partisan Electoral Agencies and Democratic Compliance: Evidence from Mexico's Federal Electoral Institute''} con Federico Est\'evez y Guillermo Rosas, \href{http://dx.doi.org/10.1016/j.electstud.2007.11.013}{\emph{Electoral Studies}} vol.\ 27, n\'um.\ 2 (junio 2008), pp.~257--71\condcomment{\boolean{longcv}}{ (ISSN:0261-3794). \textbf{\color{red}\ref{ncites:estevez.magar.rosas.2008} citas documentadas}}.

    \item \href{http://www.scielo.cl/scielo.php?script=sci_arttext&pid=S0718-090X2008000100013&lng=es&nrm=iso}{``M\'exico: reformas pese a un gobierno dividido''} con Vidal Romero, \href{http://www.scielo.cl/scielo.php?pid=0718-090X&script=sci_serial}{\emph{Revista de Ciencia Pol\'itica}} vol.\ 28, n\'um.\ 1 (junio 2008) pp.~265--85\condcomment{\boolean{longcv}}{ (ISSN:0716-1417). \textbf{\color{red}\ref{ncites:magar.romero.2008} citas documentadas}}.

    \item \href{https://github.com/emagar/cv/blob/master/papers/magar-romero-Accidentada-consolidacion2007rcp.pdf}{``M\'exico en 2006: la accidentada consolidaci\'on democr\'atica''} con Vidal Romero, \href{http://www.scielo.cl/scielo.php?pid=0718-090X&script=sci_serial}{\emph{Revista de Ciencia Pol\'itica}} vol.\ 27, n\'um.\ especial (julio 2007), pp.~184--204\condcomment{\boolean{longcv}}{ (ISSN:0716-1417). \textbf{\color{red}\ref{ncites:magar.romero.rcp.2007} citas documentadas}}.

    \item \href{https://github.com/emagar/cv/blob/master/papers/cox-magar1999apsr.pdf}{``How Much is Majority Status in the U.S. Congress Worth?''} con Gary W.\ Cox, \href{http://www.jstor.org/pss/2585397}{\emph{American Political Science Review}} vol.\ 93, n\'um.\ 2 (junio 1999), pp.~299--309\condcomment{\boolean{longcv}}{ (ISSN:0003-0554, \url{https://doi.org/10.2307/2585397}). \textbf{\color{red}\ref{ncites:cox.magar.1999} citas documentadas}}.

    \item \href{https://github.com/emagar/cv/blob/master/papers/magar-rosenblum-samuels1998.pdf}{``On the Absence of Centripetal Incentives in Double-Member Districts: The Case of Chile''} con Marc R.\ Rosenblum y David Samuels, \href{http://cps.sagepub.com/cgi/content/abstract/31/6/714}{\emph{Comparative Political Studies}} vol.\ 31, n\'um.\ 6 (diciembre 1998), DOI \href{https://doi.org/10.1177/0010414098031006002}, pp.~714--39\condcomment{\boolean{longcv}}{ (ISSN:0010-4140). \textbf{\color{red}\ref{ncites:magar.etal.1998} citas documentadas}}.

    \end{itemize}



\subsection{En revistas sin arbitraje}

    \begin{itemize}

    \item \href{http://allman.rhon.itam.mx/~emagar/cv/pdfs/magarGCP2009.pdf}{``El inmovilismo democr\'atico: un modelo de relaciones ejecutivo-legislativo en reg\'imenes con poderes separados''}, \href{http://gacetadecienciapolitica.itam.mx/}{\emph{La Gaceta de Ciencia Pol\'itica}} a\~no 6, n\'um.\ 1 oto\~no/invierno 2009, pp.~11--25\condcomment{\boolean{longcv}}{ (ISSN: 2007-6398)}.

    \item \href{http://sites.google.com/site/emagar/MagarRomeroImpasseFAeE2007.pdf}{``El impasse mexicano desde la perspectiva latinoamericana''} con Vidal Romero, \href{http://fal.itam.mx/FAE/}{\emph{Foreign Affairs en espa\~nol}} 
    vol.\ 7, n\'um.\ 1 ene.--mar.\ 2007,
    pp.~117--31\condcomment{\boolean{longcv}}{ (ISSN:1665-1707). \textbf{\color{red}\ref{ncites:magar.romero.fae.2007} citas documentadas}}.

    \end{itemize}

\subsection{Cap\'itulos en vol\'umenes colectivos}

    \begin{itemize}

    \item \href{http://allman.rhon.itam.mx/~emagar/cv/pdfs/mxEl2012}{``The electoral institutions: party subsidies, campaign decency, and entry barriers''} in \emph{Mexico's Evolving Democracy: A Comparative Study of the 2012 Elections}, coord.\ por Jorge I. Dom\'inguez, Kenneth G. Greene, Chappell Lawson y Alejandro Moreno (Washington DC: Johns Hopkins University Press, 2015, pp.\ 63--85, ISBN 1421415542).  \textbf{\color{red}\ref{ncites:magar.2007ref.2015} citas documentadas}}.

    \item \href{http://allman.rhon.itam.mx/~emagar/cv/pdfs/ePod03.pdf}{``Los contados cambios al equilibrio de poderes''} en \emph{Reformar sin mayor\'ias: La din\'amica del cambio constitucional en M\'exico, 1997--2012}, coord.\ por Mar\'ia Amparo Casar e Ignacio Marv\'an Laborde (M\'exico D.F.: Taurus, 2014, pp.~259--94, ISBN 9786071129581).

    \item \href{http://allman.rhon.itam.mx/~emagar/cv/pdfs/sanchez+magaloni+magarChapter2011.pdf}{``Legalist vs.\ Interpretativist: The Supreme Court and Democratic Transition in Mexico''} con Arianna S\'anchez y Beatriz Magaloni en \emph{Courts in Latin America}, coordinado por Gretchen Helmke y Julio R\'ios Figueroa (Nueva York: Cambridge University Press,
    2011\condcomment{\boolean{longcv}}{ ISBN-13:978-110-700-109-1}). Apareci\'o traducido al \href{http://allman.rhon.itam.mx/~emagar/cv/pdfs/sanchez+magaloni+magarCapitulo2011espanol.pdf}{castellano} en
    \emph{Tribunales constitucionales de Am\'erica Latina}, coordinado por Gretchen Helmke y Julio R\'ios Figueroa
    (M\'exico DF: Suprema Corte de Justicia de la Naci\'on, Coordinaci\'on de Compilaci\'on y Sistematizaci\'on de
    Tesis, 2011\condcomment{\boolean{longcv}}{ ISBN:978-607-468-255-7}). Reimpreso en \emph{Veinte a\~nos no es nada: La Suprema Corte y la justicia constitucional antes y despu\'es de la reforma judicial de 1994}, compilado por Camilo Emiliano Saavedra Herrera (Ciudad de M�xico: Centro de Estudios Constitucionales SCJN, 2018, pp.\ 469--516, ISBN 978-607-552-054-4).
    \condcomment{\boolean{longcv}}{\textbf{\color{red}\ref{ncites:sanchez.magaloni.magar.2011} citas documentadas}.}

    \item \href{https://github.com/emagar/cv/blob/master/papers/magar-middlebrook.pdf}{``National Election Results for Argentina, Brazil, Chile, Colombia, El Salvador, Peru, and Venezuela during the 1980s and 1990s''} con Kevin J.\ Middlebrook, ap\'endice estad\'istico en
    \emph{Conservative Parties, the Right, and Democracy in Latin America} coordinado por
    Kevin J.\ Middlebrook (Baltimore: Johns Hopkins University Press, 2000, ISBN 0801863864).

    \item
      \href{https://github.com/emagar/cv/blob/master/papers/MagarMolinar1995.pdf}{``Medios
        de comunicaci\'on y democracia''} con Juan Molinar Horcasitas, en
    \emph{Elecciones, di\'alogo y reforma: M\'exico 1994}, vol.\ 2, coordinado por Jorge Alcocer
    y Jorge Carpizo (M\'exico: Centro de Estudios para un Proyecto
    Nacional Alternativo, 1996, pp.\ 125--42, ISBN 9686981071).
    \condcomment{\boolean{longcv}}{\textbf{\color{red}\ref{ncites:magar.molinar.1995} citas documentadas}.}

    \end{itemize}

\subsection{Libro}

    \begin{itemize}

    \item \href{http://sites.google.com/site/emagar/mujereslegisladoras.pdf}{\emph{Mujeres legisladoras en M\'exico: avances, obst\'aculos, consecuencias y propuestas}},
    con Magdalena Huerta Garc\'ia (M\'exico DF: INMUJERES-CONACYT-Friedrich Ebert Stiftung-ITAM,
    2006, ISBN 970-95099-0-X, 575 pp.) Hasta enero 2012, la versi\'on electr\'onica del libro hab\'ia sido descardada un total de 264 veces del sitio internet de \href{http://www.unifemweb.org.mx/index.php?option=com_remository&Itemid=2&func=fileinfo&id=111}{ONU Mujeres}. \condcomment{\boolean{longcv}}{\textbf{\color{red}\ref{ncites:magar.huerta.2006} citas documentadas}.}

    \end{itemize}

\subsection{En revista de divulgaci\'on}

    \begin{itemize}

    \item \href{http://www.nexos.com.mx/?P=leerarticulo&Article=291}{``IFE: La casa de la partidocracia''} con Federico Est\'evez y
    Guillermo Rosas, \emph{Nexos} 376 (abril 2009), pp.\ 124--7\condcomment{\boolean{longcv}}{ (ISSN:0185-1535)}.

    \end{itemize}


\subsection{En revistas electr\'onicas}

    \begin{itemize}

    \item \href{http://www.panoramas.pitt.edu/art-and-culture/considerations-mexican-legislative-politics-selection-bias-roll-call-votes}{``Considerations for Mexican Legislative Politics: Selection Bias in Roll-Call Votes''} con Francisco Cant\'u y Scott Desposato, \emph{Panoramas}, University of Pittsburgh (enero 2016).

    \item \href{http://ssrn.com/abstract=2486885}{``The Effects of Automated Redistricting and Partisan Strategic Interaction on Representation: The Case of Mexico''} con Micah Altman, Michael P.\ McDonald y Alejandro Trelles (agosto 2014).

    \item \href{http://ssrn.com/abstract=2103866}{``The Veto as Electoral Stunt: EITM and a Test with Comparative Data''} (abril 2013).

    \item \href{http://ssrn.com/abstract=1683498}{``Partisanship among the experts: the dynamic party
    watchdog model of IFE, 1996--2010''} con Federico Est\'evez y Guillermo Rosas (septiembre 2010).
    \condcomment{\boolean{longcv}}{\textbf{\color{red}\ref{ncites:magar.estevez.rosas.2010} citas documentadas}.}

    \item \href{http://ssrn.com/abstract=1642431}{``No self-control: Decentralized Agenda Power and the Dimensional Structure of the Mexican Supreme Court''} con Beatriz Magaloni y Arianna S\'anchez (agosto 2010).

    \item \href{http://ssrn.com/abstract=1991066}{``The constructive veto and parliamentary discipline''} (abril 2010).

    \item \href{http://ssrn.com/abstract=1963863}{``The political economy of fiscal reforms in Latin America: Mexico''} con Vidal Romero y Jeffrey Timmons, comisionado por el Departamento de Investigaci\'on del Banco Interamericano de Desarrollo (marzo 2009).
        % tambi\'en disponible en http://allman.rhon.itam.mx/~emagar/mywork/IADB-Fiscal_reform_Mexico_Chapterv30.pdf

    \item \href{http://ssrn.com/abstract=1991076}{``Judges' law: ideology and coalitions in Mexico's Election Tribunal 1996--2006''} con Federico Est\'evez (noviembre 2008).

    \item \href{http://ssrn.com/abstract=1491112}{``The Incidence of Executive Vetoes in Comparative Perspective: The Case of American State Governments, 1979--1999''} (febrero 2007).

    \item \href{http://ssrn.com/abstract=1486804}{``A Model of Executive Vetoes as Electoral Stunts with Testable Hypotheses''} (febrero 2007).

    \item \href{http://ssrn.com/abstract=1991099}{``The Paradox of the Veto in Mexico (1917--1997)''}
    con Jeffrey A.\ Weldon (septiembre 2001).
    \condcomment{\boolean{longcv}}{\textbf{\color{red}\ref{ncites:magar.weldon.2001} citas documentadas.}}

    \item \href{http://ssrn.com/abstract=1400409}{``Making Sound out of Fury: The Posturing Use of Vetoes in Chile and in Mexico''} (agosto 2001).

    \item \href{http://ssrn.com/abstract=1400408}{``The Elusive Authority of Argentina's Congress:
    Decrees, Statutes, and Veto Incidence, 1983--1994''} (agosto 2001).
    \condcomment{\boolean{longcv}}{\textbf{\color{red}\ref{ncites:magarArgDecrees2001} citas documentadas}.}

    \item \href{http://ssrn.com/abstract=1991086}{``Veto Bargaining and Coalition Formation: A Theory of Presidential Policymaking with Application to Venezuela''}
    con Octavio Amorim Neto (marzo 2000).
    \condcomment{\boolean{longcv}}{\textbf{\color{red}\ref{ncites:amorim.magar.2000} citas documentadas}.}

    \item \href{http://ssrn.com/abstract=1400402}{``The Value of Majority Status in the US House''} con Gary W.\ Cox (mayo 1999).

    \item ``Elecciones municipales en el norte de M\'exico, 1970--1993: Bases de apoyo partidistas y
    alineaciones electorales'' tesis de licenciatura (julio 1994).
    \condcomment{\boolean{longcv}}{\textbf{\color{red}\ref{ncites:magar.1994} citas documentadas}.}

    \end{itemize}

\subsection{Rese\~nas bibliogr\'aficas}

    \begin{itemize}

    \item ``\href{http://allman.rhon.itam.mx/~emagar/cv/pdfs/statonJudPowStratCommlaps2011PUBLISHEDVERSOIN.pdf}{Judicial Power and Strategic Communication in Mexico. By Jeffrey K. Staton},''
    \href{http://www.blackwellpublishing.com/journal.asp?ref=1531-426x}{\emph{Latin American Politics and Society}} vol.\ 53, n\'um.\ 3 (oto\~no 2011), pp.\ 185--8\condcomment{\boolean{longcv}}{ (DOI:\href{https://doi.org/10.1017/S1531426X00009766}, ISSN:1531-426X, online ISSN:1548-2456)}.
    %ISSN:1531-426X (online ISSN:1548-2456)

    \item ``\href{http://www.politicaygobierno.cide.edu/Vol_XVII_N2_2010/06_PyG-Resenas_380-408.pdf}{Informal Coalitions and Policymaking in Latin America: Ecuador in Comparative Perspective. Por Andr\'es Mej\'ia Acosta},'' \href{http://www.politicaygobierno.cide.edu/}{\emph{Pol\'itica y Gobierno}} vol.\ 17, n\'um.\ 2 (segundo semestre 2010), pp.\ 384--6\condcomment{\boolean{longcv}}{ (ISSN:1665-2037)}.

    \item \href{https://muse.jhu.edu/article/193835/pdf}{``Ambition, Federalism, and Legislative Politics in Brazil. By David Samuels,''} \href{https://muse.jhu.edu/article/193835/pdf}{\emph{The Americas}} vol.\ 62, n\'um.\ 3 (enero 2006), pp.\ 498--500\condcomment{\boolean{longcv}}{ (DOI:\href{https://doi.org/10.1353/tam.2006.0031}, ISSN:0003-1615)}.

    \item \href{http://redalyc.uaemex.mx/src/inicio/ArtPdfRed.jsp?iCve=11500311}{``Nested Games: Rational Choice in Comparative Politics de George
    Tsebelis,''} \emph{Perfiles Latinoamericanos de FLACSO} vol.\ 2, n\'um.\ 3 (diciembre 1993), pp.~194--8\condcomment{\boolean{longcv}}{ (ISSN:0188-7653)}.

    \end{itemize}

\subsection{En series de documentos de trabajo}

    \begin{itemize}

    \item \href{http://papers.ssrn.com/abstract=1159621}{``Of Coalition and Speed: Passage and Duration of Statutes in Uruguay's Parliament,
    1985--2000''}
    con Juan Andr\'es Moraes, IBEI working papers no.\ 2008/15 ISSN:1886-2802 (junio
    2008).

    \end{itemize}


%% \subsection{Trabajo en proceso de dictamen}

%%     \begin{itemize}

%%     \item ``Presidents on the Fast Track: Fighting Floor Amendments with Restrictive Rules'' con Valeria Palanza y Gisela Sin \emph{Journal of Politics} (enviado junio 2018, revisado y reenviado para segundo dictamen marzo 2019). 

%%     \end{itemize}
    
\subsection{Trabajo in\'edito y art\'iculos de conferencias}

    \begin{itemize}

    \item \href{http://ssrn.com/abstract=2486885}{``The Effects of Automated Redistricting and Partisan Strategic Interaction on Representation: The Case of Mexico''} con Micah Altman, Michael P.\ McDonald y Alejandro Trelles (agosto 2014).

    \item \href{http://ssrn.com/abstract=2103866}{``The Veto as Electoral Stunt: EITM and a Test with Comparative Data''} (junio 2014).

    \item \href{http://ssrn.com/abstract=1683498}{``Partisanship among the experts: the dynamic party watchdog model of IFE, 1996--2011''} con Guillermo Rosas y Federico Est\'evez (abril 2013). 
    \condcomment{\boolean{longcv}}{\textbf{\color{red}\ref{ncites:magar.estevez.rosas.2010} citas documentadas}.}

    %\item \href{http://allman.rhon.itam.mx/~emagar/cv/pdfs/ePod03.pdf}{``Los contados cambios al equilibrio de poderes''} (octubre 2012).

    \item ``Who calls the tune? The search for constituent effects in the Mexican Congress'' con Federico Est\'evez (abril 2012).

    \item ``The Circuitous Path to Democracy: Legislative Control of the Bureaucracy in Presidencial Systems, the Case of Mexico'' con Alejandra R\'ios C\'azares (octubre 2004).

    \item ``El efecto de arrastre de gobernadores en elecciones de diputados federales de M\'exico, 1979--2003'' (octubre 2004).

    \item \emph{Bully pulpits: Veto politics in the legislative process of the Americas}, manuscrito de libro (mayo 2003).

    %\item ``The Pulse of Uruguayan Politics: Factions, Elections, and Veto Incidence, 1985--2002'' con Juan Andr\'es Moraes (abril 2003).

    \condcomment{\boolean{longcv}}{\item ``Decrees and Statutes in Argentina, 1983--1994: Polarization and Usurpation of Authority or Bargaining and Concessions?'' (junio 2000).}

    \condcomment{\boolean{longcv}}{\item ``The Interplay of Separation of Power and the Legislative Process'' (mayo 2000).}

    \condcomment{\boolean{longcv}}{\item ``Patterns of Executive-Legislative Conflict in Latin America and the US'' (mayo 1999).}

    \condcomment{\boolean{longcv}}{\item ``The Deadlock of Democracy Revisited: A model of Executive-Legislative Relations in Separation-of-Power Regimes'' (septiembre 1998).}

    \end{itemize}

\subsection{\href{http://sites.google.com/site/emagar/op-eds}{Entradas de blog y editoriales period\'isticos}}

    \begin{itemize}

    \item \href{https://emagar.github.io/residuales-2018-english/}{``The measurement of electoral history 1994--2018''} \emph{El blog de CiPol}, jueves 6 de noviembre 2019. 
      
    \item \href{https://emagar.github.io/residuales-2018/}{``La medici�n de la historia electoral 1994--2018''} \emph{El blog de CiPol}, martes 22 de octubre 2019. 
      
    \item \href{http://redaccion.nexos.com.mx/?p=9684}{``Saldos de la reelecci\'on municipal''} \emph{Nexos en l\'inea}, viernes 26 de octubre 2018.

    \item \href{https://emagar.github.io/alcaldes-reelectos-2/}{``Quieren abortar el experimento reeleccionista''} \emph{El blog de CiPol}, mi�rcoles 17 de octubre 2018. 
      
    \item \href{https://emagar.github.io/carta-a-si-mismo/}{``Anecdotario reeleccionista''} \emph{El blog de CiPol}, domingo 16 de septiembre 2018. 
      
    \item \href{https://emagar.github.io/alcaldes-reelectos/}{``Incumbency advantage a la mexicana''} \emph{El blog de CiPol}, s�bado 15 de septiembre 2018. 
      
    \item \href{https://emagar.github.io/los-morenos-del-pri/}{``La defenestraci�n''} \emph{El blog de CiPol}, viernes 3 de agosto 2018. 

    \item \href{https://emagar.github.io/nyt-plot/}{``AMLO jal� el mapa hacia la izquierda''} \emph{El blog de CiPol}, viernes 6 de julio 2018. 

    \item \href{https://emagar.github.io/colapso-sis-partidos/}{``El colapso del sistema de partidos mexicano''} \emph{El blog de CiPol}, lunes 2 de julio 2018. 

    \item \href{https://emagar.github.io/candidatos-pes/}{``Los candidatos de AMLO''} \emph{El blog de CiPol}, jueves 21 de junio 2018. 

    \item \href{https://emagar.github.io/los-candidatos/}{``Los candidatos al Congreso mexicano''} \emph{El blog de CiPol}, jueves 10 de mayo 2018. 
      
    \item \href{https://emagar.github.io/magar-novela-politica/}{``Doce libros pol�ticos memorables''} \emph{El blog de CiPol}, jueves 26 de abril 2018. 
      
    \item \href{https://emagar.github.io/magar-ambiciosos-2018/}{``Alcaldes que ambicionan reelegirse en 2018''} \emph{El blog de CiPol}, s�bado 31 de marzo 2018. 
      
    \item \href{https://emagar.github.io/map-distritos/}{``Un vistazo a la redistritaci�n reciente''} \emph{El blog de CiPol}, mi�rcoles 20 de diciembre 2017. 

    \item \href{https://emagar.github.io/las-tesis/}{``Tesis que investigan la reelecci�n en M�xico''} \emph{El blog de CiPol}, martes 29 de agosto 2017. 

    \item \href{https://emagar.github.io/resenha-coahuila/}{``Primeras campa�as para reelegir diputados''} \emph{El blog de CiPol}, jueves 22 de junio 2017. 

    \item \href{https://emagar.github.io/magar-instituciones/}{``Calendarios e instituciones de reelecci�n consecutiva en M�xico''} \emph{El blog de CiPol}, s�bado 20 de mayo 2017. 

    \item \href{http://redaccion.nexos.com.mx/?p=652}{``Quid pro quo''} \emph{Nexos en l\'inea}, lunes 25 de enero 2010.

    \item \href{http://redaccion.nexos.com.mx/?p=322}{``Para que bailen al son de sus representados''} \emph{Nexos en l\'inea}, mi\'ercoles 2 de diciembre 2009.

    \item \href{http://laloncheria.com/2009/11/02/impuestos-presupuesto-partidos/}{``La ecuaci\'on que no cuadra: impuestos vs.\ gasto de
    partidos''} LaLoncheria.com, lunes 2 de noviembre 2009.

    \item ``Fallos esperanzadores'' \emph{El Centro}, jueves 4 de septiembre 2008, p.~14.

    \item ``Crimen y castigo'' \emph{El Centro}, jueves 21 de agosto 2008, p.~17.

    \item ``El sufragio sin reelecci\'on'' \emph{El Centro}, jueves 7 de agosto 2008, p.~15.

    \item ``El presidente mulato'' \emph{El Centro}, jueves 24 de julio 2008, p.~12.

    \item ``Forever young'' \emph{El Centro}, jueves 10 de julio 2008, p.~15.

    \item ``La maldici\'on del oro negro'' \emph{El Centro}, jueves 26 de junio 2008, p.~17.

    \item ``Marabunta'' \emph{El Centro}, jueves 12 de junio 2008, p.~7.

    \item ``Los piratas del Cofipe'' \emph{El Centro}, jueves 29 de mayo 2008, p.~13.

    \item ``Juego en dos canchas'' \emph{El Centro}, jueves 15 de mayo 2008, p.~10.

    \item ``La equidad de g\'enero: pendiente en el Congreso'' con Magdalena Huerta Garc\'ia, \emph{La Jornada} (segmento Masiosare, n\'um.\ 420), domingo 8 de enero, 2006, p.~5.

    \item ``Un superveto para el Ejecutivo'' con Jeffrey A.\ Weldon, \emph{Reforma}, martes 17 de abril 2001, p.~6-Negocios.

    \item ``Los vetos no son excepcionales'' con Jeffrey A.\ Weldon, \emph{Reforma}, s\'abado 17 de marzo 2001, p.~\textsc{5a}.

    \end{itemize}


\section{Bases de datos p\'ublicas}

\begin{itemize}

   \item  Repositorio ``Recent Mexican electoral geography'', \url{https://github.com/emagar/mxDistritos}, GitHub (�ltima actualizaci�n 26 de nov.\ 2019). 
  
   \item  Repositorio ``Recent Mexican Election Vote Returns'', \url{https://github.com/emagar/elecRetrns}, GitHub (�ltima actualizaci�n 20 de nov.\ 2019). 
  
   \item  Repositorio ``Roll call votes data for the Mexican Chamber of Deputies'', \url{https://github.com/emagar/dipMex}, GitHub (�ltima actualizaci�n 11 de sep.\ 2018). 
  
   \item  ``Consecutive reelection institutions and electoral calendars since 1994 in Mexico'', \url{http://dx.doi.org/10.7910/DVN/X2IDWS}, Harvard Dataverse (Abr.\ 2017). 
  
    \item ``Banco de informaci\'on para la investigaci\'on aplicada en ciencias sociales (BIIACS)'' con Alejandra R\'ios C\'azares y C\'eline Gonz\'alez Schont LINK? (Abr.\ 2016). 

%    \item ``Mexican Chamber of Deputies Roll Call Votes 2006--2012'' con Francisco Cant\'u y Scott Desposato, \url{http://ericmagar.com/data/rollcall/dipFed/} and \url{https://github.com/emagar/dipMexRepo} (Feb.\ 2016).

\end{itemize}
    


\section{Presentaciones en conferencias profesionales (desde 2013)}

\begin{itemize}

\item  17 may.\ 2019, ``The dark side of electoral reform: the geography of public good distribution'', IV Taller La ciencia pol\'itica desde M\'exico, Casa de la Marquesa, ITAM.

\item 7 abr.\ 2019, \href{https://github.com/emagar/cv/blob/master/talks/lucardi-magar-presentation-mpsa2019.pdf}{``The dark side of electoral reform''} (con Adri\'an Lucardi) en la conferencia anual de la Midwest Political Science Association, Chicago, IL. 
  
\item 4 abr.\ 2019, comentarista de la mesa ``JSS SESSION: Don't Stop 'Til You Get Enough: Resources and Redistribution,'' conferencia anual de la Midwest Political Science Association, Chicago, IL. 
  
\item 7 dic.\ 2018, \href{https://github.com/emagar/cv/blob/master/talks/cideReelec2018.pdf}{``\emph{Incumbency advantage} en elecciones municipales''}, en el seminario preparatorio de un n\'umero especial de la elecci\'on de 2018 para \emph{Pol\'itica y Gobierno}, CIDE, Cd. de M\'exico.
  
\item 5 oct.\ 2018, \href{https://github.com/emagar/cv/blob/master/talks/tepjf2018.pdf}{``Del gobierno abierto a la participaci\'on efectiva''}, en el Observatorio de participaci\'on ciudadana y cultura de la transparencia y la legalidad: E-lecciones en tiempos de internet, organizado por el TEPJF y la Universidad Aut\'onoma de Coahuila, Saltillo.
  
\item 10 ago.\ 2018, \href{https://github.com/emagar/cv/blob/master/talks/gelReelec2018.pdf}{``The removal of single-term limits, redistricting, and name recognition: The case of Coahuila's state races''}, IV encuentro del Grupo de Estudios Legislativos, Cd. de M\'exico.
  
\item 10 ago.\ 2018, \href{https://github.com/emagar/cv/blob/master/talks/gelUrge2018.pdf}{``Presidents on the Fast Track: Fighting Floor Amendments with Restrictive Rules''}, IV encuentro del Grupo de Estudios Legislativos, Cd. de M\'exico.

\item 25 jun.\ 2018, \href{https://github.com/emagar/cv/blob/master/talks/magar2018wwc.pdf}{``Mexico's 2018 Congressional elections''}, Mexico Institute, Woodrow Wilson Center, Washington DC.

\item 18 may.\ 2018, \href{https://github.com/emagar/cv/blob/master/talks/taller2018.pdf}{``\emph{Name recognition} en Coahuila''}, III Taller La ciencia pol\'itica desde M\'exico, Casa de la Marquesa, ITAM.

\item 28 feb.\ 2018, ``Experiencias acad\'emicas'', mesa redonda en el 25 aniversario del departamento de ciencia pol\'itica, ITAM.
  
\item 9 nov.\ 2017, \href{https://github.com/emagar/cv/blob/master/talks/uamSelenium2017.pdf}{``Un m\'etodo para obtener texto del internet y analizarlo''}, 1er Simposio TLH en las ciencias sociales, UAM-Cuajimalpa.

\item 8 nov.\ 2017, ``Lo que nos espera en 2018'', mesa redonda en la entrega del Premio Alonso Lujambio, ITAM. 

\item 2 nov.\ 2017, \href{https://github.com/emagar/cv/blob/master/talks/usMxReelec2017.pdf}{``Another nail in the coffin of Mexican exceptionalism: The removal of (most) single-term limits''}, en la Mexico 2018 Election Conference, Center for U.S.--Mexican Studies, UCSD.

\item 14 sep.\ 2017, \href{https://github.com/emagar/cv/blob/master/talks/itam2017urge.pdf}{``Restrictive rules in the Chilean C\'amara: Fighting floor amendments with urgency authority''}, en el Seminario de Investigaci\'on Pol\'itica, ITAM. 
  
\item 2 sep.\ 2017, comentarista de la mesa ``Legislative Activity and Output,'' conferencia anual de la American Political Science Association, San Francisco, CA. 
  
\item 1 sep.\ 2017, \href{https://github.com/emagar/cv/blob/master/talks/apsa2017urge.pdf}{``Restrictive rules in the Chilean C\'amara: Fighting floor amendments with urgency authority,''} en la convenci\'on anual de la American Political Science Association, San Francisco, CA. 
  
\item 24 ago.\ 2017, ``Tensiones esperables entre la reelecci\'on consecutiva y las normas electorales,'' Seminario Permanente de An\'alisis de Justicia Electoral en M\'exico, ITAM-CIDE-IIJ.

\item 26 may.\ 2017, \href{https://github.com/emagar/cv/blob/master/talks/urgeTaller2017.pdf}{``Restrictive rules in the Chilean Congress''}, II Taller La ciencia pol\'itica desde M\'exico, Casa de la Marquesa, ITAM.

\item 11 may.\ 2017, ``Mesa de encuestas,'' mesa redonda acerca de los sondeos electorales, ITAM. 

\item 5 abr.\ 2017, ``Elecciones en Francia: contexto pol\'itico e implicaciones macroecon\'omicas,'' mesa de discusi\'on, ITAM. 

\item 9 mar.\ 2017, ``La reelecci\'on consecutiva de legisladores y alcaldes,'' mesa redonda sobre el centenario de la constituci\'on, ITAM. 

\item 12 dic.\ 2016, \href{https://github.com/emagar/cv/blob/master/talks/tepjf2016.pdf}{``Redistritaci\'on automatizada en M\'exico: retos y oportunidades''}, en el Seminario Redistritaci\'on electoral, reelecci\'on legislativa y el rol de las cortes: perspectiva comparada desde M\'exico y los Estados Unidos de Am\'erica, TEPJF.

\item 24 nov.\ 2016, \href{https://github.com/emagar/cv/blob/master/talks/urgeICP2016.pdf}{``Presidential obstruction of the agenda in Chile's Congress''}, en el Seminario ICP--UC, Santiago, Chile.

\item 18 nov.\ 2016, \href{https://github.com/emagar/cv/blob/master/talks/2016uconcGel.pdf}{``The crowded plenary: Urgency, logrolls, and the conclusive procedure in Brazil's C\^amara''}, en el III Encuentro GEL--ALACIP, Santiago, Chile.

\item 21 sep.\ 2016, \href{https://github.com/emagar/cv/blob/master/talks/2016uconc.uam.pdf}{``Saturaci\'on del pleno: urgencias del ejecutivo y el poder conclusivo en la C\^amara brasile\~na''}, en el Seminario Racionalidad, evoluci\'on y aprendizaje, claves del cambio institucional en M\'exico UAM-Xochimilco y AMEP, M\'exico DF.

\item 28 may.\ 2016, \href{https://github.com/emagar/cv/blob/master/talks/2016urgeLasa.pdf}{``Presidential obstruction of the agenda in Chile's Congress''}, en la convenci\'on sesquianual de la Latin American Studies Association, Nueva York, NY. 

\item 13 may.\ 2016, ``Automated redistricting and partisan strategic interaction in Mexico'', Taller La ciencia pol\'itica desde M\'exico, Casa de la Marquesa, ITAM.

\item 29 oct.\ 2015, \href{https://github.com/emagar/cv/blob/master/talks/2015strategy.colmex.pdf}{``Transparency, automated redistricting, and partisan startegic interaction in Mexico''}, en El Colegio de M\'exico. 

\item 29 oct.\ 2015, \href{https://github.com/emagar/cv/blob/master/talks/2015pubMap.ogpSummit.pdf}{``From open data to participation in redistricting''}, en el Open Government Partnership Summit, M\'exico DF. 

\item 25 sep.\ 2015, \href{https://github.com/emagar/cv/blob/master/talks/2015partisanBias.itam.pdf}{``Measuring malapportionment, gerrymander, and turnout effects in multi-party systems''}, en el Seminario de Investigaci\'on Pol\'itica, ITAM. 

\item 4 sep.\ 2015, \href{https://github.com/emagar/cv/blob/master/talks/2015urgenciaChile.apsa.pdf}{``Presidential obstruction of the agenda in Chile's Congress''}, en la convenci\'on anual de la American Political Science Association, San Francisco, CA. 

\item 2 sep.\ 2015, \href{https://github.com/emagar/cv/blob/master/talks/2015automatedRedistricting.eip.pdf}{``Transparency, automated redistricting, and partisan startegic interaction in Mexico''}, en el Electoral Integrity Project workshop, San Francisco, CA. 

\item 28 jul.\ 2015, \href{https://github.com/emagar/cv/blob/master/talks/2015partisanBias.casaMateOaxaca.pdf}{``Measuring malapportionment, gerrymander, and turnout effects in multi-party systems''}, en el encuentro Political Economy of Social Choices, Casa Matem\'atica Oaxaca. 

\item 18 abr.\ 2015, \href{https://github.com/emagar/cv/blob/master/talks/2015bias.mpsa.pdf}{``The effects of malapportionment, turnout, and gerrymandering in Mexico's mixed-member system''}, en la convenci\'on anual de la Midwest Political Science Association, Chicago, IL, EE.UU.

\item 13 mar.\ 2015, \href{https://github.com/emagar/cv/blob/master/talks/2015bias.uf.pdf}{``Malapportionment, party bias, and responsiveness in Mexico's mixed-member system''}, en el Seminar in Politics, University of Florida, Gainesville. 

\item 9 feb.\ 2015, \href{https://github.com/emagar/cv/blob/master/talks/2015urge.urbanaFeb.pdf}{``Congress skips turn, again: agenda obstruction in the Chilean Congress''}, en el Comparative Politics Workshop, University of Illinois at Urbana-Campaign. 

\item 14 nov.\ 2014, \href{https://github.com/emagar/cv/blob/master/talks/uhBias2014.pdf}{``Malapportionment and representation: party bias and responsiveness in Mexico''}, en la conferencia Analizing Latin American Politics, University of Houston. 

\item 3 oct.\ 2014, ``Who calls the tune? The search for constituent effects in the Mexican Congress'', Rice University, Houston, EE.UU.

\item 31 Aug.\ 2014, ``The Effects of Automated Redistricting and Partisan Strategic Interaction on Representation: The Case of Mexico'', en la convenci\'on anual de la American Political Science Association, Washington, DC, EE.UU.

\item 13 jun.\ 2014, ``The veto as electoral stunt: EITM and test with subnational comparative data'', Executive Politics Conference, Washington University in St.\ Louis, EE.UU.

\item 16 may.\ 2014, ``Partidismo entre expertos'', conferencia en l\'inea en el evento Reconexi\'on, ITAM, M\'exico DF.

\item 3 abr.\ 2014, Comentarista en el p\'anel 6--5 ``Candidate Selection and Party Entry'' de la convenci\'on anual de la Midwest Political Science Association, Chicago, IL.

\item 4 abr.\ 2014, Comentarista en el p\'anel 26--5 ``When Candidates Win Elections'' de la convenci\'on anual de la Midwest Political Science Association, Chicago, IL.

\item 20 nov.\ 2013, ``El PAN ante las reformas pendientes,'' Comit\'e Ejecutivo Nacional del Partido Acci\'on Nacional, M\'exico DF.

\item 8 nov.\ 2013, ``Partidismo entre expertos,'' TEDxITAM, M\'exico DF.

\item 29 ago.\ 2013, ``Judges' Law: Ideology and coalitions in
  Mexico's Electoral Tribunal, 1996--2006'' en la conferencia anual de
  la American Political Science Association, Chicago, IL.

\item 25 abr.\ 2013, Presentaci\'on del libro \emph{T\'acticas parlamentarias hispanomexicanas} de Alonso Lujambio y Rafael Estrada Michel, ITAM, M\'exico DF.

\item 17 abr.\ 2013, ``Dibuja M\'exico: quienquiera podr\'a redistritar con esta herramienta'' ponencia presentada en el IV congreso de la Asociaci\'on Mexicana de Estudios Parlamentarios, Universidad Iberoamericana, M\'exico DF.

\item 13 abr.\ 2013, Comentarista en el p\'anel 6--22 ``Ruling politics'' de la convenci\'on anual de la Midwest Political Science Association, Chicago, IL.

\item 12 abr.\ 2013, ``The Veto as Electoral Stunt: EITM and a Test with Comparative Data'' en la convenci\'on anual de la Midwest Political Science Association, Chicago, IL.

\item 11 abr.\ 2013, ``Partisanship among the experts: The dynamic party watchdog model of IFE, 1996--2011'' ponencia presentada en la convenci\'on anual de la Midwest Political Science Association, Chicago, IL.

%% \item 30 nov.\ 2012, ``Gubernatorial coattails in Mexican Congressional elections,'' Banco de M\'exico.

%% \item 19 oct.\ 2012, ``Methodological Considerations for Students of Mexican Legislative Politics: Selection Bias in Roll Call Publications'' en la Conferencia de Legislaturas de M\'exico y las Am\'ericas de la Asociaci\'on Mexicana de Estudios Parlamentarios, Center for U.S.--Mexican Studies, UCSD.

%% \item 17 oct.\ 2012, ``Why and when is the election referee trustworthy? IFE�s dynamic party watchdog model, 1996�2012'' en el Festschrift en honor a la carrera de Wayne Cornelius, Center for U.S.--Mexican Studies, UCSD.

%% \item 30 ago.\ 2012, ``The Veto as Electoral Stunt: EITM and a Test with Comparative Data'' ponencia presentada en la convenci\'on anual de la American Political Science Association, New Orleans, LA.

%% \item 9 may.\ 2012, Comentarista de ``Misfits, groundbreakers, or plain old politics'' de Ver\'onica Hoyo presentado en el Seminario de Pol\'itica y Gobierno, CIDE, M\'exico DF.

%% \item 27 abr.\ 2012, ``Who calls the tune? The search for constituent effects in the Mexican Congress'' ponencia presentada en el Workshop on Evidence-Based Approaches to Latin American Constitutionalism, University of Texas, Austin.

%% \item 15 abr.\ 2012, Comentarista en el p\'anel 6--19 ``Comparative Legislative Politics and Executive Power'' de la convenci\'on anual de la Midwest Political Science Association, Chicago, IL.

%% \item 13 abr.\ 2012, ``Are missing votes a problem for research on the Mexican Congress? No'' (con Francisco Cant\'u y Scott Desposato) p\'oster presentado en la convenci\'on anual de la Midwest Political Science Association, Chicago, IL.

%% \item 12 abr.\ 2012, ``Who calls the tune? Gubernatorial and constituent effects in roll-call voting in the Mexican Congress'' ponencia presentada en la convenci\'on anual de la Midwest Political Science Association, Chicago, IL.


%% \item 30 mar.\ 2012 \href{http://allman.rhon.itam.mx/~ebarrios/marzo2012}{``Preferencias en regulaci\'on electoral: partidismo en el IFE''} en el Encuentro de usuarios de R, ITAM, M\'exico DF.

%% \item 14 mar.\ 2012, ``Equilibrio de poderes (o y, sin embargo, no se mueve)'' en la conferencia pluralismo y reformas constitucionales en M\'exico: 1997--2012, CIDE--PNUD, M\'exico DF.

% \item 2 jun.\ 2011, ``Partidismo entre los expertos: din\'amica de votaci\'on en el Consejo General del IFE, 1996--2010'' en el seminario de investigaci\'on del Tribunal Electoral del Poder Judicial de la Federaci\'on, M\'exico DF.

% \item 2 abr.\ 2011, ``When cartels split: roll call votes and majority factional warfare in the Mexico City Assembly'' ponencia presentada en la convenci\'on anual de la Midwest Political Science
%     Association, Chicago.

% \item 1 abr.\ 2011, Comentarista en el panel ``6--5 Institutional Incentives and Policy Provision'' de la convenci\'on anual de la Midwest Political Science
%     Association, Chicago.

% \item 12 oct.\ 2010, Moderador de la mesa ``La revoluci\'on pol\'itica'' en el Congreso Nacional La Revoluci\'on Mexicana: nuevas visiones e interpretaciones, ITAM, M\'exico DF.

% \item 2 oct.\ 2010, Comentarista de la mesa ``Los motivos del votante'' en el cuarto Seminario Encuestas y Elecciones del Instituto Federal Electoral, Cocoyoc, Morelos.

% \item 23 sep.\ 2010, ``Partisanship among the experts: the dynamic party watchdog model of IFE, 1996--2010'' en el seminario de investigaci\'on Electoral Administration in Mexico organizado por el Center for U.S.--Mexican Studies, UC San Diego.

% \item 3 sep.\ 2010, ``No self-control: Decentralized Agenda Power and the Dimensional Structure of the Mexican Supreme Court'' en la convenci\'on anual de la American Political Science Association, Washington DC.

% \item 18 ago.\ 2010, Comentarista de la ponencia ``A long history of the political manipulation of the supreme courts in Central America and the Caribbean, 1900--2009'' de Andrea Castagnola en el Centro de Estudios y Programas Interamericanos, ITAM, M\'exico DF.

% \item 26 may.\ 2010, ``�Una caja de Pandora? Activismo y deriva ideol\'ogica en la Suprema Corte de M\'exico'' en el seminario de Pol\'itica y Gobierno del CIDE, M\'exico DF.

% \item 22 abr.\ 2010, ``The constructive veto and parliamentary discipline''
%     ponencia presentada en la convenci\'on anual de la Midwest Political Science
%     Association, Chicago.

% \item 15 abr.\ 2010, Comentarista de la mesa ``Las propuestas para nuevo equilibrio de poderes'' con ponencias de Mar\'ia Amparo Casar y Julio R\'ios Figueroa en el seminario Las agendas de reforma pol\'itica 2010 en perspectiva comparada, CIDE, M\'exico DF.

% \item 24 mar.\ 2010, Comentarista de la ponencia ``El peso de los partidos: comportamiento legislativo en
%     los estados mexicanos'' de Ra\'ul C.\ Gonz\'alez presentada en el Seminario Pol\'itica y Gobierno del
%     CIDE, M\'exico DF.

% \item 12 mar.\ 2010, ``Reflexiones sobre tres propuestas de Reforma Pol\'itica'' en la Reuni\'on de Rectores de Universidades del Centro de M\'exico de la Federaci\'on de Instituciones Mexicanas Particulares de Educaci\'on Superior (FIMPES), ITAM, M\'exico DF.

% \item 19 ene.\ 2010, Participante en la mesa redonda ``Agenda Ciudadana y Gobernabilidad: La Reforma Pol\'itica. Di\'alogo Ciudadano'' organizado por la Secretar\'ia de Gobernaci\'on, Centro Cultural Bella \'Epoca, M\'exico DF.

% \item 3 dic.\ 2009, ``La Suprema Corte en la transici\'on democr\'atica de M\'exico''
%     en el Seminario organizado por la Representaci\'on de Estudiantes de Ciencia Pol\'itica, ITAM, M\'exico DF.

% \item 5 sep.\ 2009, ``Voting and sincerity. Ideal point drift and strategy in a regulatory
%     board'' en la convenci\'on anual de la American Political Science Association, Toronto, Canad\'a.

% \item 7 mar.\ 2009, ``Activists vs.\ Legalists: The Mexican Supreme Court and its
%     Ideological Battles'' en la conferencia sobre pol\'itica judicial en Am\'erica Latina, CIDE, M\'exico DF.

% \item 3 dic.\ 2008, ``�Partidismo entre expertos? Un an\'alisis de las votaciones en el
%     Consejo General del IFE'' en el seminario de pol\'itica y gobierno del
%     CIDE, M\'exico DF.

% \item 6 nov.\ 2008, ``Judges' law: ideology and coalitions in Mexico's Election Tribunal 1996--2006''
%     ponencia en el seminario Bayesian methods in the social sciences, CIDE, M\'exico DF.

% %Esto debiera de ir en la secci\'on `cursos' o `docencia'...
% \item 24--5 oct.\ 2008, ense\~n\'e un curso corto de teor\'ia pol\'itica positiva en la
%     maestr\'ia en gobierno, Instituto de Ciencia Pol\'itica, Universidad de la Rep\'ublica, Uruguay.

% \item 20 oct.\ 2008, ``Factions with clout: presidential cabinet coalition and policy
%     in the Uruguayan Parliament'' ponencia en el Segundo Congreso
%     de la Asociaci\'on Uruguaya de Ciencia Pol\'itica, Intendencia Municipal de Montevideo,
%     Uruguay\condcomment{\boolean{longcv}}{ (mesa D3 Instituciones Pol\'iticas en Uruguay)}.

% \item 1 oct.\ 2008, Comentarista de la ponencia ``Sources of competitiveness
%     in authoritarian elections'' de Andreas Schedler
%     presentada en el seminario de pol\'itica y gobierno del
%     CIDE, M\'exico DF.

% \item 26 sep.\ 2008, ``Factions with clout: presidential cabinet coalition and policy
%     in the Uruguayan Parliament'' ponencia en el II Congreso
%     Internacional de la Asociaci\'on Mexicana de Estudios Parlamentarios, Puebla,
%     M\'exico\condcomment{\boolean{longcv}}{ (mesa 13 Relaciones entre poderes III)}.

% \item 6 ago.\ 2008, ``Raising the odds of legislation: cabinet and bill-sponsor coalition in
%     Uruguay's Parliament, 1985--2005'' ponencia en el IV Congreso
%     de la Asociaci\'on Latinoamericana de Ciencia Pol\'itica, San Jos\'e, Costa Rica\condcomment{\boolean{longcv}}{ (\'area tem\'atica 1, mesa 9 ``Instituciones pol\'iticas e inestabilidad en el Cono Sur'')}.

% \item 18 abr.\ 2008, ``The constructive veto and floor cooperation under open rule: factional co-sponsoring in
%     Uruguay'' ponencia en el Seminario de Investigaci\'on en Ciencia Pol\'itica, ITAM, M\'exico DF.

%\item 26 Oct.\ 2007, Comentarista de la ponencia ``The legal toolbox
%    for the different areas of Regional Integration'' de Ram\'on Torrent
%    (Universidad Aut\'onoma de Barcelona) presentada en la Fifth Annual
%    Conference of the Euro-Latin Study Network on Integration and Trade
%    del Banco Interamericano de Desarrollo, Barcelona,
%    Espa\~na.
%
%\item 6 Sep.\ 2007, ``Electoral stunts in parliamentary government with exogenous timetable: Evidence
%    from Spain's Autonom\'ias'', ponencia presentada en la $4^{a}$ Conferencia General del
%    European Consortium of Political Research, Pisa, Italia\condcomment{\boolean{longcv}}{ (secci\'on 18, panel 307)}.
%
%\item 3 May.\ 2007, ``\'Exito legislativo del presidente y duraci\'on del tr\'amite parlamentario: el caso de Uruguay,
%    1985--2000,'' ponencia presentada en el Instituto Interuniversitario de Iberoam\'erica, Universidad de Salamanca, Espa\~na.
%
%\item 22 Feb.\ 2007, ``Partisanship in Non-Partisan Electoral Agencies and Democratic Compliance:
%    Evidence from Mexico's Federal Electoral Institute,'' ponencia presentada en el IBEI, Barcelona, Espa\~na.
%
%\item 19 Feb.\ 2007, ``Veto Politics in the Americas: A Study of the Legislative Process,'' ponencia presentada en el Political Economy Workshop del Departamento de Gobierno, Universidad de
%    Essex, Reino Unido.
%
%\condcomment{\boolean{longcv}}{\item 8 Nov.\ 2006,  Presentaci\'on del
%libro \emph{Mujeres legisladoras en M\'exico: avances, obst\'aculos,
%consecuencias y
%    propuestas} ante la Comisi\'on de Equidad y G\'enero de la C\'amara de Diputados del
%    Congreso de la Uni\'on, M\'exico DF.}
%
%\condcomment{\boolean{longcv}}{\item 24 Oct.\ 2006, Conferencia
%``Principales Sistemas de Gobierno en Europa y Am\'erica'' ante el
%    personal de la $26^{a}$ Antig\"uedad de la Maestr\'ia en Administraci\'on Militar para la
%    Seguridad y la Defensa Nacionales, Colegio de Defensa Nacional, M\'exico DF.}
%
%\item 18 Oct.\ 2006,  Mesa redonda ``Perspectivas Ejecutivo-Legislativo'' en el Primer Congreso de la
%    Asociaci\'on Mexicana de Estudios Parlamentarios, UNAM, M\'exico DF.
%
%\condcomment{\boolean{longcv}}{\item 17 Oct.\ 2006,  Taller de
%Presentaci\'on de Resultados sobre Participaci\'on Pol\'itica y Toma de
%    Decisiones de las Mujeres en M\'exico de los proyectos de investigaci\'on
%    correspondientes a la Convocatoria 2004 del Fondo Sectorial de Investigaci\'on y
%    Desarrollo INMUJERES-CONACYT (con Magdalena Huerta Garc\'ia).}
%
%\item 12 Sep.\ 2006, ``El veto y las relaciones presidente-congreso 2006--2009,'' ponencia presentada en la
%    conferencia Coaliciones multipartidistas: condici\'on para un nuevo gobierno
%    organizada por la Fundaci\'on Rafael Preciado Hern\'andez, Club France, M\'exico DF.
%
%\item 7 Jun.\ 2006,   Mesa redonda ``Equidad y medios en el proceso electoral 2006'' del Seminario
%    Internacional Instituciones y Procesos de la Democracia Mexicana, Facultad de
%    Ciencias Pol\'iticas y Sociales, UNAM, M\'exico DF.
%
%\item 22 Abr.\ 2006,  Comentarista en el p\'anel 3--11 ``The formation of national party systems'', en la
%    convenci\'on anual de la Midwest Political Science Association, Chicago.
%
%\item 21 Abr.\ 2006, ``Can parties reduce agency slack through the judges?''
%    ponencia presentada en la convenci\'on anual de la Midwest Political Science
%    Association, Chicago.
%
%\item 15 Nov.\ 2005, ``Relaciones presidente--congreso: panorama 2006--2009,'' ponencia presentada en el
%    Departamento de Estudios Sociopol\'iticos de Banamex, M\'exico DF.
%
%\item 2 Sep.\ 2005, ``Are Non-Partisan Technocrats the Best Party Watchdogs Money Can Buy?  An
%    Examination of Mexico's Instituto Federal Electoral'' p\'oster presentado en la convenci\'on anual de la American
%    Political Science Association, Washington DC.
%
%\item 9 Abr.\ 2005, ``Party Sponsorship and Voting Behavior in Small Committees: Mexico's Instituto
%    Federal Electoral'' ponencia presentada en la convenci\'on anual de la Midwest Political Science Association, Chicago.
%
%\item 4--5 Mar.\ 2005, ``Gubernatorial Coattails and Mexican Congressional Elections since 1979,''
%    ponencia presentada en la conferencia What Kind of Democracy Has Mexico? The
%    Evolution of Presidentialism and Federalism en el Center for US-Mexican Studies,
%    UC San Diego.
%
%\item 7 Oct.\ 2004, ``The Circuitous Path to Democracy: Legislative Control of the Bureaucracy in
%    Presidencial Systems, the Case of Mexico'' en la conferencia Democratic Accountability and Rule of Law in Mexico, Universidad de Stanford.
%
%\item 1 Oct.\ 2004, ``El efecto de arrastre de gobernadores en elecciones de diputados federales de
%    M\'exico, 1979--2003'' Seminario de Investigaci\'on en Ciencia Pol\'itica de los Viernes,
%    ITAM, M\'exico DF.
%
%\item 30 Sep.\ 2004, ``El efecto de arrastre de gobernadores en elecciones de diputados federales de
%    M\'exico, 1979--2003'' ponencia presentada en el II Congreso de la Asociaci\'on
%    Latinoamericana de Ciencia Pol\'itica, M\'exico DF.
%
%\condcomment{\boolean{longcv}}{\item 15 Jul.\ 2004,  Moderador en la mesa ``Institutions in Presidential Democracy: Comparative
%    Perspective from the Americas'' en la conferencia Democratic Institutions in Latin
%    America: Implications for Mexico's Evolving Democracy, Center for US-Mexican
%    Studies, UC San Diego.}
%
%\item 19 Abr.\ 2004,  Comentarista en la mesa 3--17 ``The new Mexican Political Economy'' de la
%    convenci\'on anual de la Midwest Political Science Association, Chicago.
%
%\item 6--9 May.\ 2003, Presentaci\'on del borrador de mi libro  \emph{Bully Pulpits} en el seminario ``Political
%    Institutions and Public Choice'', Dept. of Political Science, Michigan State University.
%
%\item 7 Abr.\ 2003, ``The Pulse of Uruguayan Politics: Factions, Elections, and Veto Incidence, 1985--2002,'' ponencia presentada en la reuni\'on anual de la Midwest Political Science
%    Association, Chicago.
%
%\condcomment{\boolean{longcv}}{\item 7 Abr.\ 2003, ``Theoretical and Empirical Models of Post-Election Coalition Bargaining in
%    presidential democracies: Ecuador and Uruguay in comparative perspective,''
%    p\'oster presentado con Andr\'es Mej\'ia Acosta en la convenci\'on anual de la Midwest
%    Political Science Association, Chicago.}
%
%\item 12 Feb.\ 2003,  Participante en el seminario ``The Mexican Left: Resurgence or Default Option,''
%    Center for US-Mexican Studies, UC San Diego.
%
%\item 6 Sep.\ 2002,  Comentarista de la ponencia ``Majoritarian Electoral Systems and Consumer Power:
%    Price-Level Evidence from the OECD Countries'' de Mark Andreas Kayser y
%    Ronald Rogowski, hecha por el primero en el Seminario de Investigaci\'on en Ciencia
%    Pol\'itica, ITAM, M\'exico DF.
%
%\item Jun.--Jul.\ 2002, Participante en el Empirical Implications of Theoretical Models Summer Institute
%    en el Center for Basic Research in the Social Sciences, Universidad de Harvard.
%
%\item 12 Abr.\ 2002,   Comentarista de la ponencia ``Presidential Vetoes in the Early Republic'' de Nolan
%    McCarty en el Seminario de Investigaci\'on en Ciencia Pol\'itica, ITAM, M\'exico DF.
%
%\item 30 Nov.\ 2001,   Comentarista en la presentaci\'on del libro \emph{Lawmaking and Bureaucratic Discretion in
%    Modern Democracies}, de John D. Huber y Charles D. Shipan, hecha por el primero en
%    el Seminario de Investigaci\'on Pol\'itica, CIDE, M\'exico DF.
%
%\item 6 Sep.\ 2001,  ``The Paradox of the Veto in Mexico (1917--1997)'' ponencia presentada en la convenci\'on anual de la Latin American Studies Association, Washington DC.
%
%\item 31 Ago.\ 2001,  ``Making Sound out of Fury: The Posturing Use of Vetoes in Chile and in
%    Mexico,'' ponencia presentada en la convenci\'on anual de la American Political
%    Science Association, San Francisco.
%
%\item 30 Ago.\ 2001,  ``The Elusive Authority of Argentina's Congress: Decrees, Statutes, and Veto
%    Incidence, 1983--1994,'' ponencia presentada en la convenci\'on anual de la American
%    Political Science Association, San Francisco.
%
%\item 20 Abr.\ 2001, ``The Incidence of Executive Vetoes in Comparative Perspective: Position-Taking
%    and Uncertainty in US State Governments, 1983--1993,'' ponencia presentada en
%    la convenci\'on anual de la Midwest Political Science Association, Chicago.
%
%\item 16 Mar.\ 2001, ``El position-taking y la incertidumbre como fuentes de vetos del ejecutivo en los
%    gobiernos estatales de los EEUU'', ponencia presentada en el Seminario de
%    Investigaci\'on en Ciencia Pol\'itica, ITAM, M\'exico DF.
%
%\item 13 Oct.\ 2000,  Comentarista de la ponencia ``Democratization and the Economy in Mexico:
%    Equilibrium (PRI) Hegemony and its Demise,'' escrita por Alberto D\'iaz
%    Cayeros, Beatriz Magaloni y Barry R. Weingast, y presentada por el primero en el
%    Seminario de Investigaci\'on en Ciencia Pol\'itica, ITAM, M\'exico DF.
%
%\item 17 Mar.\ 2000, ``Veto Bargaining and Coalition Formation: A Theory of Presidential Policymaking
%    with Application to Venezuela,'' ponencia presentada en la convenci\'on anual de la
%    Latin American Studies Association, Miami.
%
%\item 15 Oct.\ 1999, ``La incidencia de vetos en perspectiva comparativa: el caso de los gobiernos
%    estatales de los EEUU,'' ponencia presentada en el Centro de Estudios para
%    el Desarrollo Institucional, Universidad de San Andr\'es, Argentina.
%
%\item 23 Ago.\ 1999, ``Veto Incidence in Comparative Perspective: The Case of US State
%    Governments,'' ponencia presententada en el Departamento de Ciencia Pol\'itica,
%    UC San Diego.
%
%\item 7--14 Jul.\ 1999,    Discusi\'on del manuscrito final de \emph{Political Economics}, de Torsten Persson
%    y Guido Tabellini, en el seminario ``Encounters with Authors'' del Center for
%    Basic Research in Social Science, Universidad de Harvard.
%
%\item 9 May.\ 1999,
%    \href{http://sites.google.com/site/emagar/MagarUCLA-2.PDF}{``Patterns of
%    Executive-Legislative Conflict in Latin America and the
%    US''}
%    ponencia presentada en el First International Graduate Student Retreat for
%    Comparative Research, organizado por la Society for Comparative Research y el
%    Center for Comparative Social Analysis, UC Los Angeles.
%
%\item 4 Sep.\ 1998, ``The Deadlock of Democracy Revisited: A model of Executive-Legislative
%    Relations in Separation-of-Power Regimes,'' ponencia presentada en la reuni\'on
%    anual de la American Political Science Association, Boston.

\end{itemize}




\section{Dictaminador}

\subsection{Art\'iculos para revistas}

\begin{itemize}

\item \emph{American Journal of Political Science} (2000, 2004, 2004, 2006)

\item \emph{American Political Science Review} (2003, 2004, 2004, 2007, 11/2018)

\item \emph{Applied Geography} (11/2017)
  
\item \emph{Comparative Political Studies} (1998, 2009, 3/2018)

\item \emph{Econom\'ia Mexicana} Nueva \'Epoca (2008, 2011)

\item \emph{EconoQuantum} (2012, 2012)

\item \emph{Foro Internacional} (2012)

\item \emph{Journal of Legislative Studies} (2010)

\item \emph{Journal of Politics} (2011, 2015, 2016, 1/2017, 6/2019)

\item \emph{Journal of Politics in Latin America} (2014)

\item \emph{Latin American Politics and Society} (2015)

\item \emph{Legislative Studies Quarterly} (2013, 2014)

\item \emph{Revista Sociol\'ogica} de la UAM (2003)

\item \emph{Party Politics} (2008)

\item \emph{Perfiles Latinoamericanos} de FLACSO--M\'exico (2009)

\item \emph{Pol\'itica y Gobierno} (2001, 2002, 2010, 2011, 2012, 2013, 2015, 5/2018)

\item \emph{Political Research Quarterly} (2013)

\item \emph{Revista Mexicana de Sociolog\'ia} (2001, 2005)

\item \emph{Social Choice and Welfare} (2012)

\item \emph{World Politics} (2004)

\end{itemize}

\subsection{Libros}

\begin{itemize}

\item El Colegio de M\'exico (6/2019)

\item Facultad Latinoamericana de Ciencias Sociales--Sede M\'exico (2003, 2004)

\item Fondo de Cultura Econ\'omica (2013)

\item Oxford University Press (2008)

\end{itemize}

\subsection{Cap\'itulos de libro}

\begin{itemize}

\item Asociaci\'on Latinoamericana de Ciencia Pol\'itica (\href{http://ericmagar.com/cv/moreEvid/dictamenALACIP2012n1.pdf}{2012}, 2012)

\end{itemize}

\subsection{Proyectos de investigaci\'on}

\begin{itemize}

\item CONICYT (Chile) (\condcomment{\boolean{longcv}}{proyecto 1200317 en }9/2019.

% evaluaci�n del proyecto 1200317 ASSESSING THE ROLE OF COMMITTEES IN THE CHILEAN CONGRESS, 1932-1973 AND 1990-2018, fondecyt conicyt
  
\item Consejo Nacional de Ciencia y Tecnolog\'ia (\condcomment{\boolean{longcv}}{proyectos 101632 y 102870 en }2009)

% 1 evaluacion del  proyecto "000000000102870" con titulo "CALIDAD DEMOCR\'ATICA Y `NUEVA GOBERNANZA �: CUESTIONES METODOL\'OGICAS DEL PRINCIPAL-AGENTE EN LAS DEMOCRACIAS LATINOAMERICANAS CONTEMPOR\'ANEAS" del proponente "avaro sosa dante adalberto" que fue  presentado en respuesta a la Convocatoria de "CIENCIA BASICA 2008" del Fondo Fondo SEP - CONACYT
%
% 2 evaluacion del  proyecto "000000000101632" con titulo "EL FRAMING DE LOS LIDERES POLITICOS Y LOS MOVIMIENTOS SOCIALES" del proponente "Aquiles Chihu Amparan" que fue  presentado en respuesta a la Convocatoria de "CIENCIA BASICA 2008" del Fondo Fondo SEP - CONACYT

\end{itemize}




\section{Comit\'es acad\'emicos}

\begin{itemize}

\item 2020, \href{https://www.mpsanet.org/Professional-Development/Awards-Call-for-Nominations/Award-Recipients-2020}{Evan Ringquist Award} for best paper in the topic of political institutions selection committee, Midwest Political Science Association.

\item 2016, Jewell/Loewenberg award selection committee, Legislative Studies Section, American Political Science Association.

\item 2014, Comit\'e de selecci\'on de becarios Fulbright-Garc\'ia Robles, (COMEXUS, M\'exico DF).

\item 2014--18, \href{http://ericmagar.com/cv/moreEvid/EricMagarComiteEditorialPyG.2014.pdf}{miembro} del Comit\'e Editorial de la revista \emph{Pol\'itica y Gobierno} (CIDE, M\'exico DF).

\item 2014, \href{http://mpsanet.org/Awards/2014AwardRecipients/tabid/873/Default.aspx}{Kellogg/Notre Dame Award selection committee}, Midwest Political Science Association.

\item 2012--14,  \href{http://ericmagar.com/cv/moreEvid/comiteFlacso2012.pdf}{miembro} del Comit\'e de Coordinaci\'on Acad\'emica de la Maestr\'ia en Gobierno y Asuntos P\'ublicos, FLACSO M\'exico.

\item 2010--12, miembro del Comit\'e Editorial de la FLACSO M\'exico.

\item 2009--16, Research Associate, Center for U.S.-Mexican Studies (UC, San Diego).

\item 2006--,   miembro del comit\'e editorial, \emph{La Gaceta de Ciencia Pol\'itica} (ITAM, M\'exico DF).

\item 2001--2004,  miembro del Comit\'e de Coordinaci\'on Acad\'emica de la Maestr\'ia en Gobierno y Asuntos P\'ublicos, FLACSO M\'exico.

\end{itemize}



\condcomment{\boolean{longcv}}{

\section{Otros comit\'es y seminarios}

\begin{itemize}

\item 13 mar.\ 2019, Mesa Sectorial PND 2019--2024 Dessarrollo Democr\'atico, Secretar\'ia de Gobernaci\'on.
  
\item 2012,   miembro del Comit\'e T\'ecnico de Especialistas de la Comisi\'on Temporal de Debates, Instituto Federal Electoral.

\item 2006,   miembro del Consejo de Investigaci\'on, diario \emph{Reforma}.

\item 2005--2006,  miembro del Consejo Directivo de Tu Rock es Votar, AC.

\end{itemize}
}



\condcomment{\boolean{longcv}}{

\section{Asociaciones profesionales}

\begin{itemize}

\item American Political Science Association (desde 1998)

\item Asociaci\'on Latinoamericana de Ciencia Pol\'itica (desde 2002)

\item Asociaci\'on Mexicana de Estudios Parlamentarios (desde 2011)

\item Latin American Studies Association (desde 2015)

\item Midwest Political Science Association (desde 2001 por lo menos)

\item Public Choice Society (desde 2006)

\end{itemize}
}



%\condcomment{\boolean{longcv}}{

\section{Eventos acad\'emicos que he organizado}

\begin{itemize}

\item 17 may.\ 2019, ``IV Taller La Ciencia Pol\'itica desde M\'exico'' \condcomment{\boolean{longcv}}{(duraci\'on: 10 horas)}, Ciudad de M{\'e}xico, ITAM. \condcomment{\boolean{longcv}}{Veinte investigadores establecidos en la ciudad de M\'exico y uno en el extranjero presentaron trabajos recientes. \href{http://ericmagar.com/taller/}{https://emagar.github.io/programa-taller/}.}

\item 9--10 ago.\ 2018, ``IV encuentro del Grupo de Estudios Legislativos de la Asociaci\'on Latino Americana de Ciencia Pol\'itica'' \condcomment{\boolean{longcv}}{(duraci\'on: 22 horas)}, Ciudad de M{\'e}xico. \condcomment{\boolean{longcv}}{Treinta y ocho especialistas de pol\'itica legislativa de nueve pa\'ises presentaron investigaciones recientes. Organizado junto con Adri\'an Lucardi y Juan Pablo Micozzi con financiamiento conjunto del ITAM, Washington University in St.\ Louis y el \emph{Legislative Studies Quarterly}.}
  
\item 18 may.\ 2018, ``III Taller La Ciencia Pol\'itica desde M\'exico'' \condcomment{\boolean{longcv}}{(duraci\'on: 10 horas)}, Ciudad de M{\'e}xico, ITAM. \condcomment{\boolean{longcv}}{Diecis\'eis investigadores establecidos en la ciudad de M\'exico y uno en el extranjero presentaron trabajos recientes. \href{http://ericmagar.com/taller/}{https://emagar.github.io/programa-taller/}.}

\item 26 may.\ 2017, ``Taller La Ciencia Pol\'itica desde M\'exico'' \condcomment{\boolean{longcv}}{(duraci\'on: 10 horas)}, Ciudad de M{\'e}xico, ITAM. \condcomment{\boolean{longcv}}{Catorce investigadores establecidos en la ciudad de M\'exico presentaron trabajos recientes. \href{http://ericmagar.com/taller/}{http://ericmagar.com/taller/}.}

\item 13 may.\ 2016, ``Taller La Ciencia Pol\'itica desde M\'exico'' \condcomment{\boolean{longcv}}{(duraci\'on: 10 horas)}, M{\'e}xico DF, ITAM. \condcomment{\boolean{longcv}}{Diecis\'eis investigadores establecidos en la ciudad de M\'exico presentaron trabajos recientes. \href{http://ericmagar.com/taller/}{http://ericmagar.com/taller/2016/}.}

\item 13--17 jun.\ 2011, ``Taller de estimaci\'on de variables latentes'' \condcomment{\boolean{longcv}}{(duraci\'on: 15 horas)} impartido junto con el Profesor Guillermo Rosas del Departamento de Ciencia Pol\'itica de la Washington University en St.\ Louis, M{\'e}xico DF, ITAM. \condcomment{\boolean{longcv}}{Esto ha dado lugar a un grupo de investigaci\'on en m\'etodos de investigaci\'on avanzados del que, adem\'as, forman parte Rosario Aguilar Pariente (CIDE) y Luis Estrada Straffon (iniciativa privada), dedicado a repetir este taller todos los veranos con una mayor oferta de cursos.}

\item 6--7 nov.\ 2008, Seminario ``Bayesian methods in the social sciences'' co-patrocinado por
    el Departamento de Ciencia Pol{\'i}tica de la Washington University in St.\ Louis, la Divisi\'on de
    Estudios Pol{\'i}ticos del CIDE y el Departamento de Ciencia
    Pol\'itica del ITAM, M{\'e}xico DF CIDE e ITAM.

\item 8 jun.\ 2006, Seminario ``2006: las encuestas en la mira'' evento co-patrocinado por WAPOR-
    M{\'e}xico, el Center for US-Mexican Studies de UCSD y el Departamento de Ciencia
    Pol\'itica del ITAM, Casa de California, Chimalistac DF.

\item Oct.\ 2004, Organizador junto con la Fundaci\'on Friedrich Naumann y el Instituto Federal
    Electoral de la conferencia magistral del Dr. Hans-J\"urgen Beerfeltz sobre el sistema
    electoral alem\'an, ITAM.

\item 2001--2003, Coordinador del Seminario de Investigaci\'on Pol\'itica de los Viernes,
    Depto.\ de Ciencia Pol\'itica, ITAM.

\end{itemize}
%}

\condcomment{\boolean{longcv}}{
\section{Clases y cursos impartidos (desde 2014)}

\begin{itemize}

 \item Oto\~no 2019, Pol\'itica Comparada II, lic.\ en ciencia pol\'itica ITAM (48 horas).
 \item Oto\~no 2019, Elecci\'on P\'ublica III, lic.\ en ciencia pol\'itica ITAM (48 horas).
 \item Primavera 2019, Pol\'itica Comparada II, lic.\ en ciencia pol\'itica ITAM (48 horas).
 \item Primavera 2019, Elecci\'on P\'ublica III, lic.\ en ciencia pol\'itica ITAM (48 horas).
 \item Oto\~no 2018, Pol\'itica Comparada III, lic.\ en ciencia pol\'itica ITAM (48 horas).
 \item Oto\~no 2018, Elecci\'on P\'ublica III, lic.\ en ciencia pol\'itica ITAM (48 horas).
 \item Primavera 2018, Pol\'itica Comparada II, lic.\ en ciencia pol\'itica ITAM (48 horas).
 \item Primavera 2018, Elecci\'on P\'ublica III, lic.\ en ciencia pol\'itica ITAM (48 horas).
 \item Oto\~no 2017, Seminario de Investigaci\'on Pol\'itica D, lic.\ en ciencia pol\'itica ITAM (48 horas).
 \item Oto\~no 2017, Elecci\'on P\'ublica III, lic.\ en ciencia pol\'itica ITAM (48 horas).
 \item 17 jun.\ 2017, An\'alisis de datos georreferenciados, Curso de periodismo de datos ITAM-DataCivica (3 horas).
 \item Primavera 2017, Pol\'itica Comparada II, lic.\ en ciencia pol\'itica ITAM (48 horas).
 \item Primavera 2017, Elecci\'on P\'ublica III, lic.\ en ciencia pol\'itica ITAM (48 horas).
 \item Oto\~no 2016, Pol\'itica Comparada II, lic.\ en ciencia pol\'itica ITAM (48 horas).
 \item Oto\~no 2016, Elecci\'on P\'ublica III, lic.\ en ciencia pol\'itica ITAM (48 horas).
 \item Primavera 2016, Seminario de Investigaci\'on Pol\'itica D, lic.\ en ciencia pol\'itica ITAM (48 horas).
 \item Primavera 2016, Elecci\'on P\'ublica III, lic.\ en ciencia pol\'itica ITAM (48 horas).
 \item Oto\~no 2015, Pol\'itica Comparada II, lic.\ en ciencia pol\'itica ITAM (48 horas).
 \item Oto\~no 2015, Elecci\'on P\'ublica III, lic.\ en ciencia pol\'itica ITAM (48 horas).
 \item 17 jun.\ 2015, La separaci\'on de poderes en Am\'erica Latina y Estados Unidos, Judicial Training Program for Argentine Judges, 
       Washington University in St.\ Louis (con Guillermo Rosas (2 horas).  
 \item Primavera 2014, Pol\'itica Comparada II, lic.\ en ciencia pol\'itica ITAM (48 horas).
 \item primavera 2014, Elecci\'on P\'ublica III, lic.\ en ciencia pol\'itica ITAM (48 horas).
 %% \item Oto\~no 2013, Pol\'itica Comparada II, lic.\ en ciencia pol\'itica ITAM (48 horas).
 %% \item Oto\~no 2013, Seminario de Investigaci\'on Pol\'itica: pol\'itica y elecciones legislativas, lic.\ en ciencia pol\'itica ITAM (48 horas).
 %% \item 20 sep.\ 2013, La relaci\'on Ejecutivo--Legislativo, dipl.\ en planeaci\'on y operaci\'on legislativa ITAM (5 horas).
 %% \item Primavera 2013, Seminario de Investigaci\'on Pol\'itica: relaci\'on Ejecutivo--Legislativo, lic.\ en ciencia pol\'itica ITAM (48 horas).
 %% \item Primavera 2013, Elecci\'on P\'ublica III, lic.\ en ciencia pol\'itica ITAM (48 horas).
 %% \item Oto\~no 2012, Pol\'itica Comparada II, lic.\ en ciencia pol\'itica ITAM (48 horas).
 %% \item Oto\~no 2012, Elecci\'on P\'ublica III, lic.\ en ciencia pol\'itica ITAM (48 horas).
 %% \item 8 jun.\ 2012, An\'alisis del poder legislativo, dipl.\ en an\'alisis pol\'itico ITAM (4 horas).
 %% \item Primavera 2012, Pol\'itica Comparada II, lic.\ en ciencia pol\'itica ITAM (48 horas).
 %% \item Primavera 2012, Elecci\'on P\'ublica III, lic.\ en ciencia pol\'itica ITAM (48 horas).
 % \item 11 nov.\ 2011, La nueva correlaci\'on de fuerzas entre los poderes del estado mexicano, dipl.\ en planeaci\'on y operaci\'on legislativa ITAM (5 horas).
 % \item 26 y 28 sept.\ 2011, Investigaci\'on aplicada al poder judicial, dipl.\ en m\'etodos de investigaci\'on para ciencias sociales, Suprema Corte de Justicia de la Naci\'on (6 horas).
 % \item Oto\~no 2011, Pol\'itica Comparada II, lic.\ en ciencia pol\'itica ITAM (48 horas).
 % \item Oto\~no 2011, Elecci\'on P\'ublica III, lic.\ en ciencia pol\'itica ITAM (48 horas).
 % \item 13 al 17 de junio 2011, Taller de estimaci\'on de variables latentes ITAM-Washington University St.\ Louis (15 horas).
 % \item 11--12 feb.\ 2011, An\'alisis del poder legislativo y del poder judicial, dipl.\ en an\'alisis pol\'itico ITAM (9 horas).
 % \item Primavera 2011, Seminario de Investigaci\'on Pol\'itica: elecciones legislativas, lic.\ en ciencia pol\'itica ITAM (48 horas).
 % \item Primavera 2011, Elecci\'on P\'ublica III, lic.\ en ciencia pol\'itica ITAM (48 horas).
 % \item 3 dic.\ 2010, La nueva correlaci\'on de fuerzas entre los poderes del estado mexicano, dipl.\ en planeaci\'on y operaci\'on legislativa ITAM (5 horas).
 % \item Oto\~no 2010, Pol\'itica Comparada II, lic.\ en ciencia pol\'itica ITAM (48 horas).
 % \item Oto\~no 2010, Elecci\'on P\'ublica III, lic.\ en ciencia pol\'itica ITAM (48 horas).
 % \item Primavera 2010, Pol\'itica Comparada II, lic.\ en ciencia pol\'itica ITAM (48 horas).
 % \item Primavera 2010, Elecci\'on P\'ublica III, lic.\ en ciencia pol\'itica ITAM (48 horas).
 % \item 22--23 ene.\ 2010, La nueva correlaci\'on de fuerzas entre los poderes del estado mexicano, dipl.\ en planeaci\'on y operaci\'on legislativa ITAM (9 horas).
 % \item Oto\~no 2009, Seminario de Investigaci\'on Pol\'itica: estudios legislativos, lic.\ en ciencia pol\'itica ITAM (48 horas).
 % \item Oto\~no 2009, Elecci\'on P\'ublica III, lic.\ en ciencia pol\'itica ITAM (48 horas).
 % \item Primavera 2009, Pol\'itica Comparada II, lic.\ en ciencia pol\'itica ITAM (48 horas).
 % \item Primavera 2009, Elecci\'on P\'ublica III, lic.\ en ciencia pol\'itica ITAM (48 horas).
 % \item 23--24 oct.\ 2008, Teor\'ia pol\'itica positiva y estudios legislativos en Latinoam\'erica, mtr\'ia.\ y doctorado en ciencia pol\'itica del Instituto de Ciencia Pol\'itica, Universidad de la Rep\'ublica, Uruguay (6 horas).
 % \item Oto\~no 2008, Pol\'itica Comparada II, lic.\ en ciencia pol\'itica ITAM (48 horas).
 % \item Oto\~no 2008, Elecci\'on P\'ublica III, lic.\ en ciencia pol\'itica ITAM (48 horas).
 % \item Primavera 2008, Pol\'itica Comparada II�, lic.\ en ciencia pol\'itica ITAM (48 horas).
 % \item Primavera 2008, Elecci\'on P\'ublica III, lic.\ en ciencia pol\'itica ITAM (48 horas).
 % \item Oto\~no 2007, Estudios de Am\'erica Latina, M\'aster en Relaciones Internacionales IBEI, Barcelona (16 horas).
 % \item Oto\~no 2006, Pol\'itica Comparada II, lic.\ en ciencia pol\'itica ITAM (48 horas).
 % \item Oto\~no 2006, Elecci\'on P\'ublica III, lic.\ en ciencia pol\'itica ITAM (48 horas).
 % \item Primavera 2006, Introducci\'on al An\'alisis Pol\'itico, mtr\'ia.\ en pol\'iticas p\'ublicas ITAM (33 horas).
 % \item Primavera 2006, Elecci\'on P\'ublica III, lic.\ en ciencia pol\'itica ITAM (48 horas).
 % \item Oto\~no 2005, Pol\'itica Comparada II, lic.\ en ciencia pol\'itica ITAM (48 horas).
 % \item Primavera 2005, Elecci\'on P\'ublica III, lic.\ en ciencia pol\'itica ITAM (48 horas).
 % \item Oto\~no 2004, Pol\'itica Comparada II, lic.\ en ciencia pol\'itica ITAM (48 horas).
 % \item Primavera 2004, seminario de Investigaci\'on Pol\'itica : relaciones ejecutivo-legislativo, lic.\ en ciencia pol\'itica ITAM (48 horas).
 % \item Primavera 2004, Elecci\'on P\'ublica III, lic.\ en ciencia pol\'itica ITAM (48 horas).
 % \item Oto\~no 2003, An\'alisis pol\'itico institucional, mtr\'ia.\ en pol\'iticas p\'ublicas ITAM (33 horas).
 % \item Oto\~no 2003, Pol\'itica Comparada II, lic.\ en ciencia pol\'itica ITAM (48 horas).
 % \item Primavera 2003, Pol\'itica Comparada II, lic.\ en ciencia pol\'itica ITAM (48 horas).
 % \item Primavera 2003, Elecci\'on P\'ublica III, lic.\ en ciencia pol\'itica ITAM (48 horas).
 % \item Oto\~no 2002, An\'alisis pol\'itico institucional, mtr\'ia.\ en pol\'iticas p\'ublicas ITAM (33 horas).
 % \item Oto\~no 2002, Pol\'itica Comparada II, lic.\ en ciencia pol\'itica ITAM (48 horas).
 % \item Primavera 2002, Seminario de Investigaci\'on Pol\'itica: historia pol\'itica de M\'exico, lic.\ en ciencia pol\'itica ITAM (48 horas).
 % \item Oto\~no 2001, An\'alisis pol\'itico institucional, mtr\'ia.\ en pol\'iticas p\'ublicas ITAM (33 horas).
 % \item Oto\~no 2001, Pol\'itica Comparada III: Am\'erica Latina y EEUU, lic.\ en ciencia pol\'itica ITAM (48 horas).
 % \item Primavera 2001, Seminario de Investigaci\'on Pol\'itica II: relaciones ejecutivo-legislativo, lic.\ en ciencia pol\'itica ITAM (48 horas).
 % \item Oto\~no 2000, An\'alisis pol\'itico institucional, mtr\'ia.\ en pol\'iticas p\'ublicas ITAM (33 horas).
 % \item Oto\~no 2000, Pol\'itica Comparada III: Am\'erica Latina y EEUU, lic.\ en ciencia pol\'itica ITAM (48 horas).

\end{itemize}

}



\section{Asesor\'ia de tesis}

\subsection{Nivel Doctorado}

    \begin{itemize}

    \item Dic.\ 2009, Mario Alejandro Torrico Ter{\'a}n,
    ``Factores explicativos y dimensiones de la estabilidad pol{\'i}tica: un estudio mundial,''
    Doctorado en Ciencias Sociales, FLACSO--M{\'e}xico.

        \begin{list}{--}{\topsep=-10pt \parsep=0pt \parskip=0pt} %% itemize sin espacios de m\'as para premios

        \item Obtuvo Calificaci{\'o}n Excelente con Recomendaci{\'o}n para Publicaci{\'o}n en el examen
        doctoral.

        \end{list}

    \end{itemize}


\subsection{Nivel Maestr\'ia}

    \begin{itemize}

    \item Sep.\ 2007, Tom{\'a}s Pinheiro Fiori, ``The Political Economy of Hydrocarbons in Latin America,''
    M{\'a}ster en Relaciones Internacionales, IBEI, Barcelona.

    \item Sep.\ 2007, Silvia Mar{\'i}a G{\'a}ndara Berger, ``Aproximaci{\'o}n a la situaci{\'o}n actual de las relaciones civiles-militares
    en Guatemala desde una perspectiva democr{\'a}tica,'' M{\'a}ster en Relaciones Internacionales, IBEI, Barcelona.

    \item Nov.\ 2004, Octael Nieto V{\'a}zquez, ``La estructura, el gasto y el desempe\~no de las organizaciones
    culturales federales en M\'exico, 1990--2003,'' Maestr\'ia en Sociolog\'ia Pol\'itica,
    Instituto Mora.

    \item Jul.\ 2004, Ren\'e Ram\'irez Gallegos, ``Pseudo-salida, silencio y �deslealtad?: entre la inacci\'on
    colectiva, la desigualdad de bienestar y la pobreza de capacidades,'' Maestr\'ia en
    Gobierno y Asuntos P\'ublicos, FLACSO-M\'exico.

        \begin{list}{--}{\topsep=-10pt \parsep=0pt \parskip=0pt} %% itemize sin espacios de m\'as para premios

        \item Obtuvo calificaci\'on Excelente en la defensa oral.

        \end{list}

    \end{itemize}

\subsection{Nivel licenciatura}

\begin{itemize}

    \item Ago.\ 2019, Lorea Urruch\'ua Mercader ``El efecto de las circunscripciones provinciales en la proporcinalidad del caso espa�ol'' (caso, ITAM).

    \item Jun.\ 2019, Daniel Saavedra Llad\'o ``M\'aquinas partidistas vs m\'aquinas cartogr\'aficas: algunas consecuencias pol\'iticas de la redistritaci\'on automatizada en los estados de M\'exico'' ITAM.

        \begin{list}{--}{\topsep=-10pt \parsep=0pt \parskip=0pt} %% itemize sin espacios de m\'as para premios

        \item Obtuvo una Menci\'on Especial en el examen profesional.

        \end{list}

    \item Jun.\ 2019, Patricia Cruz Mar\'in ``El cumplimiento de sentencias de la Corte Interamericana de Derechos Humanos'' ITAM.

        \begin{list}{--}{\topsep=-10pt \parsep=0pt \parskip=0pt} %% itemize sin espacios de m\'as para premios

        \item Obtuvo una Menci\'on Honor\'ifica en el examen profesional.

        \end{list}

    \item Mar.\ 2019, Julia Madrazo Clavijo ``Diferencias etnoling\"u\'isticas y fronteras electorales: el caso de la consulta ind\'igena en Chiapas'' ITAM.

        \begin{list}{--}{\topsep=-10pt \parsep=0pt \parskip=0pt} %% itemize sin espacios de m\'as para premios

        \item Obtuvo una Menci\'on Especial en el examen profesional.

        \end{list}

    \item Dic.\ 2016, Luis Daniel Cubr\'ia Trujillo ``Determinantes y efectos del calendario electoral en M\'exico: un estudio subnacional'' ITAM.

    \item Dic.\ 2014, Eduardo Alejandro Sierra Albarr\'an ``La insolvencia fiscal del municipio mexicano: ra\'ices institucionales de su endeudamiento'' ITAM.

    \item Ago.\ 2014, Adriana Mael S\'anchez L\'opez ``La conformaci\'on del bloque obregonista en la XXVII Legislatura y su relaci\'on con el gobierno de Carranza'' ITAM.

        \begin{list}{--}{\topsep=-10pt \parsep=0pt \parskip=0pt} %% itemize sin espacios de m\'as para premios

        \item Obtuvo una Menci\'on Especial en el examen profesional.

        \end{list}

    \item Abr.\ 2014, Mariana Meza Hern\'andez ``Tipo de r\'egimen y la ratificaci\'on de tratados internacionales de derechos humanos'' ITAM.

        \begin{list}{--}{\topsep=-10pt \parsep=0pt \parskip=0pt} %% itemize sin espacios de m\'as para premios

        \item Obtuvo una Menci\'on Especial en el examen profesional.

        \end{list}

    \item Feb.\ 2014, Luis Fernando Godoy Rueda ``Reelecci\'on en la C\'amara de Diputados, 1917--1933: federalismo y ambici\'on pol\'itica'' ITAM.

        \begin{list}{--}{\topsep=-10pt \parsep=0pt \parskip=0pt} %% itemize sin espacios de m\'as para premios

        \item Obtuvo una Menci\'on Especial en el examen profesional.

        \end{list}

    %\item Dic.\ 2013, Beatriz Guerrero Auna ``Caso: Formaci\'on de gobiernos en democracias parlamentarias: el caso del Knesset 19 en Israel'' ITAM.

    \item Sep.\ 2013, Nayeli Carolina Hern\'andez Medell\'in ``Caso: Negociaciones del Primer Ministro: la formaci\'on de una nueva coalici\'on de gobierno'' ITAM.

    \item Nov.\ 2011, Beatriz Ruiz Manning ``Caso: Esquemas de representatividad y su impacto en el escenario pol\'itico que antecedi\'o a la guerra civil de los Estados Unidos'' ITAM.

    \item Nov.\ 2011, Paloma Alejandra Franco L\'opez ``Caso: La formaci\'on del nuevo gobierno suizo: un modelo para el abandono de arreglos no institucionales'' ITAM.

    \item Oct.\ 2011, Blanca Mart\'inez ``Caso: Organos electorales locales, de la neutralidad a la autonom\'ia partidista: el caso del Distrito Federal'' ITAM.

    \item Jun.\ 2011, Luis Everdy Mej\'ia L\'opez ``Procesos de intermediaci\'on pol\'itica en la elecci\'on presidencial mexicana del 2006'' ITAM.

        \begin{list}{--}{\topsep=-10pt \parsep=0pt \parskip=0pt} %% itemize sin espacios de m\'as para premios

        \item Obtuvo una Menci\'on Especial en el examen profesional.

        \end{list}

    \item May.\ 2011, David Alejandro Orozco Olvera ``Caso: Votaciones divididas en el primer Consejo del Instituto Electoral Del Distrito Federal (IEDF) 1999--2006'' ITAM.

    \item Mar.\ 2011, Fernando Pe\'on Molina ``Caso: Discrecionalidad en la Asignaci\'on de las Participaciones Municipales de Yucat\'an, 1995--2002'' ITAM.

    \item Nov.\ 2010, Williams Oswaldo Ochoa Gallegos ``Caso: La reconfiguraci\'on de la din\'amica pol\'itica en las regiones: nuevos actores en el sistema pol\'iticos mexicano'' ITAM.

    \item Oct.\ 2010, Jaime Arredondo S\'anchez Lira ``El subsidio municipal para la seguridad p\'ublica: an\'alisis de la f\'ormula de elegibilidad'' ITAM.

        \begin{list}{--}{\topsep=-10pt \parsep=0pt \parskip=0pt} %% itemize sin espacios de m\'as para premios

        \item Obtuvo una Menci\'on Especial en el examen profesional.

        \end{list}

    \item Sep.\ 2010, Gabriel Reyes Galv\'an ``Caso: Delincuencia, una estimaci\'on desde el crecimiento econ\'omico'' ITAM.

    \item Ago.\ 2010, C\'esar Eduardo Montiel Olea ``Repensando los poderes presidenciales: un estudio del veto total y parcial en M\'exico, 1997--2009'' ITAM.

        \begin{list}{--}{\topsep=-10pt \parsep=0pt \parskip=0pt} %% itemize sin espacios de m\'as para premios

        \item Obtuvo una Menci\'on Especial en el examen profesional.

        \end{list}

    \item Jun.\ 2010, Xunaaxi Barcel\'o Monroy ``Caso: El IFE y la Transici\'on Democr\'atica en M\'exico'' ITAM.

    \item Oct.\ 2009, Roberto Ponce L{\'o}pez ``La geograf{\'i}a del voto en M\'exico: organizaci\'on partidista en las secciones electorales, 1997--2003'' ITAM.

        \begin{list}{--}{\topsep=-10pt \parsep=0pt \parskip=0pt} %% itemize sin espacios de m\'as para premios

        \item Obtuvo una Menci\'on Especial en el examen profesional.

        \end{list}

    \item Mar.\ 2009, Patricia Isabel Zapata Morales ``Las cuotas de g\'enero en los congresos locales
    mexicanos: una aproximaci\'on emp\'irica a un debate normativo'' ITAM.

        \begin{list}{--}{\topsep=-10pt \parsep=0pt \parskip=0pt} %% itemize sin espacios de m\'as para premios

        \item Obtuvo una Menci\'on Especial en el examen profesional.

        \end{list}

    \item Ene.\ 2009, Luis Esteban Islas Bacilio ``El Sindicato Nacional de Trabajadores
    de la Educaci\'on (SNTE) y la calidad de la educaci\'on en M\'exico'' ITAM.

    \item Nov.\ 2008, Gabriel Alfredo Peto Velasco ``Participaci\'on electoral, concurrencia y movilizaci\'on en
    las elecciones municipales en M\'exico, 1991--2006'' ITAM.

    \item Jun.\ 2008, Ana Quesada Reynoso ``Noche de gatos pardos: autoridad, se\~nales y
    aprobaci\'on en el ITAM'' ITAM.

    \item Dic.\ 2007, Luis Alejandro Trelles Yarza y Diego Eduardo Mart\'inez Cant\'u
    ``Fronteras electorales: aportaciones del modelo de redistritaci\'on mexicano al estado de
    California'' ITAM.

        \begin{list}{--}{\topsep=-10pt \parsep=0pt \parskip=0pt} %% itemize sin espacios de m\'as para premios

        \item Obtuvo una Menci\'on Especial en el examen profesional;

        \item $1^{er}$ lugar del Premio de Investigaci\'on Ex-ITAM
        2007.

        \end{list}

    \item Jun.\ 2006, Magdalena Guadalupe Huerta Garc\'ia ``�Qu\'e hace falta para que haya m\'as mujeres
    en la C\'amara de Diputados? Determinantes del acceso de mujeres a candidaturas y a
    puestos de elecci\'on en la C\'amara de Diputados, 1994--2006'' ITAM.

        \begin{list}{--}{\topsep=-10pt \parsep=0pt \parskip=0pt} %% itemize sin espacios de m\'as para premios

        \item Obtuvo una Menci\'on Especial en el examen profesional.

        \end{list}

    \item Jun.\ 2006, Hilda Gabriela Enrigue Gonz\'alez ``La Suprema Corte y la desigualdad en el acceso a
    la justicia: �a qui\'en sirve el amparo en materia fiscal?'' ITAM.

        \begin{list}{--}{\topsep=-10pt \parsep=0pt \parskip=0pt} %% itemize sin espacios de m\'as para premios

        \item Obtuvo una Menci\'on Especial en el examen profesional;

        \item $2^{o}$ lugar del Premio Banamex de Econom\'ia 2006;

        \item $1^{er}$ lugar del Premio de Investigaci\'on Ex-ITAM 2006;

        \item Menci\'on Honor\'ifica del Premio Nacional Tlaca\'elel de Consultor\'ia
        Econ\'omica 2008.

        \end{list}

    \item Jun.\ 2006, Yunuel Patricia Cruz Guerrero ``Del simbolismo a la equidad, el impacto del g\'enero
    en los partidos pol\'iticos y la ALDF: representaci\'on en comisiones y prioridades
    legislativas'' ITAM.

    \item Jun.\ 2006, Leticia Lowenberg Cruz ``El sesgo de g\'enero en la nominaci\'on de candidatos y la
    elecci\'on de diputados locales para la Asamblea Legislativa del Distrito Federal,
    1994--2003'' ITAM.

    \item Oct.\ 2004, Gildardo Enrique Zafra Amescua ``Efectos de arrastre en elecciones concurrentes en
    M\'exico'' ITAM.

    \item Sep.\ 2004, Manlio Alberto Guti\'errez V\'azquez ``La coordinaci\'on de las acciones legislativas del
    poder ejecutivo en M\'exico: una propuesta de reorganizaci\'on'' ITAM.

    \item May.\ 2003, Jimena Otero Zorrilla ``�\emph{Gerrymandering} en M\'exico? La geograf\'ia pol\'itica federal, 1994--1997'' ITAM.

        \begin{list}{--}{\topsep=-10pt \parsep=0pt \parskip=0pt} %% itemize sin espacios de m\'as para premios


        \item Obtuvo una Menci\'on Especial en el examen profesional;

        \item Menci\'on Especial del Premio de Investigaci\'on Ex-ITAM
        2003.

        \end{list}

    \item May.\ 2003, Cristina Rivas Ochoa y Diana Evangelina Toledo Figueroa ``Comisiones: la brecha
    institucional'' ITAM.

    \end{itemize}

%\condcomment{\boolean{longcv}}{
% \subsection{Nivel Doctorado, tesis en curso}
%
%     \begin{itemize}
%
%     \end{itemize}
%
%
% \subsection{Nivel Maestr\'ia, tesis en curso}
%
%     \begin{itemize}
%
%     \end{itemize}
%
%
%\subsection{Nivel Licenciatura, tesis en curso}
%
%    \begin{itemize}
%
%    \item Laura Enr\'iquez ``Ideolog\'ia y consistencia de diputados y sus votantes'' ITAM.
%
%    \item Fernando Pe\'on ``Participaciones municipales en Yucat\'an 1995--2006'' ITAM.
%
%    \item Laura X ``Partidos en comisiones'' ITAM.
%
%    \end{itemize}
%
%}

\condcomment{\boolean{longcv}}{
\section{Sinodal en ex\'amenes de grado}

\subsection{Nivel doctorado}

    \begin{itemize}

    \item 27 feb.\ 2014, Fanny Sleman Vald\'es ``El poder presupuestal de los gobernadores en 
    M\'exico (2000--2012): la influencia de las reglas presupuestales y del pluralismo legislativo'' 
    Doctorado en Ciencias Sociales, FLACSO--M\'exico.

    \item 15 dic.\ 2009, Esteban Colla de Robertis ``Contributions to the
    Economic Analysis of Monetary Committees'' Doctorado en Econom\'ia, ITAM.

    \item 15 dic.\ 2009, Mario Alejandro Torrico Ter\'an,
    ``Factores explicativos y dimensiones de la estabilidad pol\'itica: un estudio mundial,''
    Doctorado en Ciencias Sociales, FLACSO--M\'exico.

    \end{itemize}

\subsection{Nivel maestr\'ia}

    \begin{itemize}

    \item 25 nov.\ 2011, Julia V\'azquez Guti\'errez ``Trabajo infantil y su efecto sobre la educaci\'on: el caso de M\'exico 2010'' Maestr\'ia en Econom\'ia, ITAM.

    \item 8 sep.\ 2008, Jos\'e Jaime Sainz Santamar\'ia 
          ``Variables pol\'iticas y geogr\'aficas en la asignaci\'on de comisiones legislativas en la C\'amara de Diputados: el caso de la Legislatura LIX (2003--06)'' 
          Maestr\'ia en Pol\'iticas P\'ublicas, ITAM.

   \item 9 nov.\ 2004, Octael Nieto V\'azquez, ``La estructura, el gasto y el
   desempe\~no de las organizaciones culturales federales en M\'exico, 1990--2003,''
   Maestr\'ia en Sociolog\'ia Pol\'itica, Instituto Mora.

   \item 23 ago.\ 2002, Victor Manuel Esp\'indola D\'iaz y Vicente Mendoza T\'ellez
   Gir\'on, ``La no reelecci\'on legislativa y municipal en M\'exico: una
   estructura de incentivos perversa'', Maestr\'ia en Pol\'iticas P\'ublicas,
   ITAM.

    \end{itemize}

\subsection{Nivel licenciatura (desde 2012; si no hay indicaci\'on es en ciencia pol\'itica--ITAM)}

    \begin{itemize}

    \item 12 feb.\ 2016, Reynaldo Lecona Esteban ``La reforma electoral de Nayarit de 2008: Representaci\'on municipal y su influencia en la distribuci\'on del gasto''.

    \item 26 ago.\ 2014, Adriana Mael S\'anchez L\'opez ``La conformaci\'on del bloque obregonista en la XXVII Legislatura y su relaci\'on con el gobierno de Carranza''.

    \item 23 jul.\ 2014, Mercedes Recke Velasco ``Perfiles de las personas probables a donar y a hacer voluntariado en M\'exico''.

    \item 20 jun.\ 2014, Adri\'an Luzanilla Hern\'andez ``�El fin del presidencialismo mexicano? An\'alisis de las iniciativas de reforma pol\'itica en el Congreso Mexicano (2010- 2013)''.

    \item 20 jun.\ 2014, Jorge Adalberto C\'a\~nez Fern\'andez ``Subiendo al votante mediano a la bicicleta: pol\'itica y movilidad urbana sustentable en el Distrito Federal''.

    \item 23 may.\ 2014, Yanixel Palacios Gal\'an ``Los tintes semi-presidenciales en el r\'egimen mexicano''.

    \item 28 mar.\ 2014, Jessica Zarkin Notni ``La revocaci\'on de mandato y su efecto como mecanismo de disuasi\'on''.

    \item 1 abr.\ 2014, Mariana Meza Hern\'andez ``Tipo de r\'egimen y la ratificaci\'on de tratados internacionales de derechos humanos''.

    \item 5 feb.\ 2014, Luis Fernando Godoy Rueda ``Reelecci\'on en la C\'amara de Diputados, 1917-1933: federalismo y ambici\'on pol\'itica''.

    \item 16 dic.\ 2013, Juan Jaffet Mill\'an M\'arquez ``Los tiempos electorales en el Estado de M\'exico''.

    \item 16 dic.\ 2013, Beatriz Guerrero Auna ``Formaci\'on de gobiernos en democracias parlamentarias: el caso del Knesset 19 en Israel''.

    \item 25 oct.\ 2013, Juan Francisco Valerio M\'endez ``Quien se mueve si sale en la foto: el restablecimiento de la reelecci\'on consecutiva en M\'exico''.

    \item 25 oct.\ 2013, Germ\'an Raziel M\'endez Dom\'inguez ``La amenaza cre\'ible de una militancia costosa, refrendo PAN 2012''.

    \item 20 sep.\ 2013, Laura Guadalupe Vargas Vargas ``Reelecci\'on inmediata: condici\'on necesaria pero no suficiente''.

    \item 20 sep.\ 2013, Mauricio Cano Fern\'andez ``De la oposici\'on al gobierno y de vuelta a la oposici\'on: la evoluci\'on del Partido Acci\'on Nacional y sus repercusiones en la afiliaci\'on de militantes''.

    \item 20 sep.\ 2013, Nayeli Carolina Hern\'anez Medell\'in ``Negociaciones del Primer Ministro: la formaci\'on de una nueva coalici\'on de gobierno''.

    \item 14 ago.\ 2013, Carolina Iris May\'en Huerta ``Evaluaci\'on al dise\~no e implementaci\'on del Prosoft''.

    \item 13 jun.\ 2013, Paola Gonz\'alez Rubio Novoa ``El \'exito de los partidos en la oposici\'on que se debilitan al llegar al poder: una aproximaci\'on cualitativa al caso del Partido Acci\'on Nacional''.

    \item 13 jun.\ 2013, Arturo Cuevas Bautista ``Liderazgo, dogm\'aticos y pragm\'aticos''.

    \item 12 jun.\ 2013, Naxhieli Mariel Sandoval Guerrero ``Los saldos de la derrota: en PAN �Un partido sin militantes?''.

    \item 12 jun.\ 2013, Ana Carina L\'opez Zerme\~no ``Estrategias distributivas para marcos institucionales diversos''.

    \item 12 jun.\ 2013, Sof\'ia Viridiana Arias Ba\~nuelos ``PAN: Veh�culo de office-seekers a partir de 2000''.

    \item 29 may.\ 2013, Alan Adame Pinacho ``El PAN en evoluci\'on: complejidades del desarrollo organizacional del partido y su v\'inculo con la atracci\'on de activistas''.

    \item 29 may.\ 2013, Daniela Casar Gar\'ia ``La centralizaci\'on del poder al interior del Partido Acci\'on Nacional y sus efectos en la militancia''.

    \item 29 may.\ 2013, Laura Lizette Enr\'iquez Rodr\'iguez ``Militancia fantasma: an\'alisis del crecimiento de la militancia del Partido Acci\'on Nacional''.

    \item 29 may.\ 2013, Ana Karen Basurto Scott ``Ganar gobierno sin perder el partido: la tensi\'on organizacional del PAN ante la democratizaci\'on''.

    \item 28 may.\ 2013, Eduardo Heredia Ayala ``La restauraci\'on de las carreras legislativas en M\'exico: esbozando una reforma reeleccionista''.

    \item 28 may.\ 2013, Selene Garc\'ia Gonz\'alez ``El papel de los backbenchers en el gobierno de coalici\'on del Reino Unido''.

    \item 30 abr.\ 2013, Francisco Javier Varela Sandoval ``Elecciones en Israel: �continuidad o cambio en la coalici\'on gobernante?''.

    \item 25 abr.\ 2013, David Aldama Navarrete ``Una radiograf\'ia del debate voto leal vs. voto vol\'atil en la literatura sobre pol\'itica distributiva''.

    \item 14 dic.\ 2012, Francisco Zavala Garc\'ia ``Modelo de decisi\'on de los partidos pol\'iticos para la formaci\'on de coaliciones''.

    \item 14 dic.\ 2012, Guillermo Eduardo Figueroa Calder\'on ``La IV Legislatura de la Asamblea Legislativa del Distrito Federal: una legislatura transformadora con mayor\'ia partidista''.

    \item 14 dic.\ 2012, Alejandra Lecona Medina ``Prospectiva de la reelecci\'on legislativa en M\'exico: un an\'alisis comparado''.

    \item 10 dic.\ 2012, Pablo Z\'arate Ramos ``Endeudamiento de las entidades federativas de M\'exico y sus determinantes pol\'iticos: an\'alisis emp\'irico del periodo 1993-2009''.

    \item 9 nov.\ 2012, Roberto Carlos Estrada Mart\'inez ``Hacia un modelo de democracia participativa en M\'exico''.

    \item 26 oct.\ 2012, Ricardo Alexys Valencia Lara ``Partidos, facciones y gasto p\'ublico: el caso de Guanajuato''.

    \item 5 jun.\ 2012, Daniel Rub\'in Reyes ``An\'alisis pol\'itico, social y econ\'omico desde la perspectiva de las instituciones de EE.UU. que detonaron la secesi\'on de los estados sure\~nos''.

    \item 3 may.\ 2012, Alehira Orozco Reyes ``El modelo de Laver y Shepsle en el caso de Suiza''.

    \item 3 may.\ 2012, Sim\'on Cohen Cats ``La causa de la Guerra de Secesi\'on: m\'as all\'a de la esclavitud''.

    \item 26 abr.\ 2012, Jorge Ignacio Echevarr\'ia Gonz\'alez ``Reelecci\'on ilimitada de legisladores federales: efectos en el sistema pol\'itico mexicano'' .

   \item 19 abr.\ 2012, Paola Margarita Navarrete Hern\'andez ``An\'alisis del conflicto seccional en la d\'ecada de 1850: Estados Unidos''.

    \item 6 mar.\ 2012, Sandra P\'erez Toxqui S\'anchez ``Estados Unidos y la semilla de la secesi\'on: intereses esclavizados por la libertad constitucional''.

    \item 6 mar.\ 2012, Elvira Garc\'ia Aguayo ``Notas sobre democracia en Estados Unidos de Am\'erica hacia la mitad del siglo XIX''.

    \item 6 feb.\ 2012, Luis Rodolfo Oropeza Ch\'avez ``Impacto de la alternancia pol\'itica en el desempe\~no del gobierno, a diez a\~nos del cambio de partido a nivel federal''.

    \item 6 feb.\ 2012, Eduardo Fragoso Salom\'on ``El perfil del elector priista 1989--2006''.

    % \item 15 dic.\ 2011, Beatriz Ruiz Manning ``Esquemas de representatividad y su impacto en el escenario pol\'itico que antecedi\'o a la Guerra Civil de los Estados Unidos''.

    % \item 15 dic.\ 2011, Paloma Alejandra Franco L\'opez ``La formaci\'on del nuevo gobierno suizo: un modelo para el abandono de arreglos no institucionales''.

    % \item 7 dic.\ 2011, Iracema Infante Barbosa ``Rumbo a la construcci\'on de la democracia en Irak''.

    % \item 2 dic.\ 2011, Isabel Gil Evaert ``Justicia y memoria: Resistencia ciudadana y cultural en Argentina''.

    % \item 1 dic.\ 2011, Welmar Eduardo Rosado Buenfil ``Modelo unidimensional de alianzas electorales: impactos electorales de las diferentes estrategias de revelaci\'on de informaci\'on y de la consolidaci\'on de una alianza electoral'' Lic.\ en Econom\'ia, ITAM.

    % \item 13 oct.\ 2011, Blanca Gloria Mart\'inez Navarro ``Organos electorales locales, de la neutralidad a la autonom\'ia partidista: el caso del Distrito Federal''.

    % \item 23 ago.\ 2011, Edgar Samuel Moreno G\'omez  ``El reclutamiento de los diputados en M\'exico: experiencia pol\'itica y carreras parlamentarias entre 1997 y 2009''.

    % \item 16 ago.\ 2011, Ignacio Rafael Camacho Mier y Ter\'an ``El subsidio a la vivienda en M\'exico: eficiencia econ\'omica y viabilidad pol\'itica''.

    % \item 11 ago.\ 2011, Sergio Jes\'us Ascencio Bonfil ``Cambio institucional end\'ogeno: las reformas a los sistemas electorales de los congresos locales en M\'exico, 1979--2009''.

    % \item 24 jun.\ 2011, Virgilio Alberto Mu\~noz Alberich ``El elector bajacaliforniano, entre la participaci\'on y el abstencionismo''.

    % \item 23 jun.\ 2011, Luis Everdy Mej\'ia L\'opez ``Procesos de intermediaci\'on pol\'itica en la elecci\'on presidencial mexicana del 2006''.

    % \item 22 jun.\ 2011, Brenda Samaniego de la Parra ``Programa de Desarrollo Humano Oportunidades: an\'alisis de duraci\'on y determinantes de salida bajo un contexto pol\'itico-electoral''.

    % \item 2 may.\ 2011, David Alejandro Orozco Olvera ``Caso: Votaciones divididas en el primer Consejo del Instituto Electoral Del Distrito Federal (IEDF) 1999--2006''.

    % \item 23 mar.\ 2011, Fernando Pe\'on Molina ``Caso: Discrecionalidad en la Asignaci\'on de las Participaciones Municipales de Yucat\'an, 1995--2002'' Lic.\ en Ciencias Sociales, ITAM.

    % \item 14 dic.\ 2010, Juan Cristobal Gil Ram\'irez ``Caso: El Sistema Electoral en las C\'amaras y Ayuntamientos de M\'exico''.

    % \item 23 nov.\ 2010, Williams Oswaldo Ochoa Gallegos ``Caso: La reconfiguraci\'on de la din\'amica pol\'itica en las regiones: nuevos actores en el sistema pol\'iticos mexicano''.

    % \item 21 oct.\ 2010, Eynar de los Cobos Carmona ``Caso: La representaci\'on proporcional y la evoluci\'on democr\'atica del Estado de M\'exico''.

    % \item 7 oct.\ 2010, Jaime Arredondo S\'anchez Lira ``El subsidio municipal para la seguridad p\'ublica: an\'alisis de la f\'ormula de elegibilidad''.

    % \item 28 sep.\ 2010, Gabriel Reyes Galv\'an ``Caso: Delincuencia, una estimaci\'on desde el crecimiento econ\'omico''.

    % \item 19 ago.\ 2010, C\'esar Eduardo Montiel Olea ``Repensando los poderes presidenciales: un estudio del veto total y parcial en M\'exico, 1997--2009''.

    % \item 29 jul.\ 2010, Jos\'e Antonio Tejeda Le\'on ``Caso: �La afiliaci\'on de asegurados al Seguro Popular sigue un patr\'on clientelista?''.

    % \item 29 jul.\ 2010, Ana Mar\'ia L\'eon Miravalles ``Caso: Democracia y participaci\'on ciudadana''.

    % \item 29 jul.\ 2010, Elda Mar\'ia Vecchi Martini ``Caso: El c\'ancer de mama en M\'exico''.

    % \item 18 jun.\ 2010, Xunaaxi Barcel\'o Monroy ``Caso: El IFE y la transici\'on democr\'atica en M\'exico'' Lic.\ en Ciencias Sociales, ITAM.

    % \item 9 jun.\ 2010, Ismael Estrada Mercado ``Caso: Coaliciones y disciplina parlamentaria en la legislatura del Estado de M\'exico''.

    % \item 19 abr.\ 2010, Aileen Fern\'andez Villase\~nor ``Caso: El veto del ejecutivo al presupuesto de egresos de la federaci\'on''.

    % \item 15 oct.\ 2009, Roberto Ponce L\'opez ``La geograf\'ia del voto en M\'exico:
    % organizaci\'on partidista en las secciones electorales, 1997--2003''.

    % \item 28 ago.\ 2009, Ang\'elica Pulido ``La representaci\'on sustantiva de
    % las mujeres en la C\'amara de Diputados'' Lic.\ en Ciencia Pol\'itica,
    % ITAM.

    % \item 14 ago.\ 2009, Gustavo Adolfo Robles Peiro ``Modelos espaciales de votaci\'on
    % legislativa: una aplicaci\'on para la C\'amara de Diputados de la 60 legislatura del
    % Congreso de la Uni\'on 2006--2009'' Lic.\ en Econom\'ia, ITAM.

    % \item 9 jun.\ 2009, Dolores Bernal Fontanals ``El gobierno dividido en M\'exico:
    % coaliciones ganadoras en la C\'amara de Diputados'' Lic.\ en Ciencia Pol\'itica,
    % ITAM.

    % \item 6 mar.\ 2009, Patricia Isabel Zapata Morales ``Las cuotas de g\'enero en los congresos locales
    % mexicanos: una aproximaci\'on emp\'irica a un debate normativo'' Lic.\ en Ciencia Pol\'itica,
    % ITAM.

    % \item 12 feb.\ 2009, Luis Esteban Islas Bacilio ``El Sindicato Nacional de Trabajadores
    % de la Educaci\'on (SNTE) y la calidad de la educaci\'on en M\'exico'' Lic.\ en Ciencia Pol\'itica,
    % ITAM.

    % \item 17 dic.\ 2008, Lorenzo G\'omez-Morin Escalante ``El voto cat\'olico en M\'exico
    % a catorce a\~nos de las reformas constitucionales en materia religiosa,
    % 1992--2006''.

    % \item 1 dic.\ 2008, Gabriel Alfredo Peto Velasco ``Participaci\'on electoral, concurrencia y movilizaci\'on en
    % las elecciones municipales en M\'exico, 1991--2006''.

    % \item 12 jun.\ 2008, Ana Quesada Reynoso ``Noche de gatos pardos: autoridad, se\~nales y
    % aprobaci\'on en el ITAM''.

    % \item 20 may.\ 2008, Omar Ernesto Alejandre Galaz ``Autonom\'ia e imparcialidad en el IFE: unanimidad
    % y divisi\'on en el Consejo General, 1996--2007''.

%    \item 27 oct.\ 2006, Jos\'e Manuel Gonz\'alez Echevarr\'ia ``La disciplina de mercado en los
%    dep\'ositos a plazo en M\'exico: un enfoque institucional'' Lic.\
%    en Ciencia Pol\'itica, ITAM.
%
%    \item 24 oct.\ 2006, Eunice Ma. Elena S\'anchez Garc\'ia ``Ciclos partidistas
%    y electorales en los municipios de M\'exico: 1998--2002'' Lic.\
%    en Ciencia Pol\'itica, ITAM.
%
%    \item 24 oct.\ 2006, Karina \'Alvarez Torres ``Ciclos partidistas y electorales
%    en los municipios de M\'exico: 1998--2002'' Lic.\ en Ciencia
%    Pol\'itica, ITAM.
%
%    \item 29 jun.\ 2006, Sergio Holgu\'in Rom\'an ``La descapitalizaci\'on del
%    capitalismo: el sistema financiero como barrera de entrada a
%    mercados competitivos''.
%
%    \item 26 jun.\ 2006, Magdalena Guadalupe Huerta Garc\'ia ``�Qu\'e hace falta
%    para que haya m\'as mujeres en la C\'amara de Diputados? Determinantes del
%    acceso de mujeres a candidaturas y a puestos de elecci\'on en la C\'amara
%    de Diputados, 1994-2006''.
%
%    \item 5 jun.\ 2006, Leticia Lowenberg Cruz ``El sesgo de g\'enero en la
%    nominaci\'on de candidatos y en la elecci\'on de Diputados Locales para
%    la Asamblea Legislativa del Distrito Federal, 1994--2003''
%   .
%
%    \item 5 jun.\ 2006, Yunuel Patricia Cruz Guerrero ``Del simbolismo a la
%    equidad, el impacto del g\'enero en los partidos pol\'iticos y la ALDF:
%    representaci\'on en Comisiones y prioridades legislativas''
%   .
%
%    \item 28 abr.\ 2006, Adriana Berrueto Canales ``Las comisiones estatales
%    de derechos humanos en M\'exico: una l\'ogica de delegaci\'on''
%   .
%
%    \item 17 mar.\ 2006, Maira Gonz\'alez Baudouin ``Evoluci\'on ideol\'ogica en el
%    Chile postautoritario: influencia de \'elites y cambio institucional''
%   .
%
%    \item 13 ene.\ 2006, Francisco Antonio Cant\'u Morales ``El impacto electoral
%    de Progresa-Oportunidades'' Lic.\ en Ciencia Pol\'itica,
%    ITAM.
%
%    \item 21 oct.\ 2005, Fernando Rodr\'iguez Doval ``Ortodox\'ia con pragmatismo:
%    activismo partidario y actitudes estrat\'egicas en el PAN''
%   .
%
%    \item 13 jun.\ 2005, Jaime Mart\'inez Bowness ``La agenda de gobierno:
%    actores, teor\'ias y estrategias en el surgimiento y evoluci\'on de las
%    pol\'iticas p\'ublicas'' Lic.\ en Ciencia Pol\'itica,
%    ITAM.
%
%    \item 16 dic.\ 2004, Melisa del Refugio Z\'u\~niga Garc\'ia ``La Transici\'on en
%    Tabasco: de la hegemon\'ia al bipartidismo, 1982--2000'' Lic.\
%    en Ciencia Pol\'itica, ITAM.
%
%    \item 12 nov.\ 2004, Manuel L\'opez Gonz\'alez ``La disputa por las bases: el
%    paulatino ascenso electoral del PRD en Michoac\'an, 1980--2001''
%   .
%
%    \item 5 oct.\ 2004, Gildardo Enrique Zafra Amezcua ``Efectos de arrastre
%    en elecciones concurrentes en M\'exico'' Lic.\ en Ciencia
%    Pol\'itica, ITAM.
%
%    \item 27 sep.\ 2004, Manlio Alberto Guti\'errez V\'azquez ``La coordinaci\'on de
%    las acciones legislativas del poder ejecutivo en M\'exico: una
%    propuesta de reorganizaci\'on'' Lic.\ en Ciencia Pol\'itica,
%    ITAM.
%
%    \item 10 sep.\ 2004, Tom\'as Gerardo Granados Salinas, ``Investigaci\'on de mercados
%    culturales. Estudio del p\'ublico que asiste a la Feria Internacional
%    de Libro de Guadalajara'', Lic.\ en Matem\'aticas
%    Aplicadas, ITAM.
%
%    \item 30 jun.\ 2004, Andr\'es Aguayo Mazzucato ``El voto dividido en la Ciudad
%    de M\'exico: elecciones ejecutivas del a\~no 2000'' Lic.\ en
%    Ciencia Pol\'itica, ITAM.
%
%    \item 7 jun.\ 2004, Julio Ernesto Herrera Segura ``Flujos de telefon\'ia de
%    larga distancia internacional, comercio exterior y ciclo econ\'omico:
%    determinantes pol\'iticos y econ\'omicos de las telecomunicaciones en
%    M\'exico y la OCDE''.
%
%    \item 24 may.\ 2004, Marcela Vel\'azquez Bolio ``Desigualdad y crecimiento
%    econ\'omico: nuevos v\'inculos institucionales para la teor\'ia de A.\
%    Alesina y D.\ Rodrik'' Lic.\ en Ciencia Pol\'itica,
%    ITAM.
%
%    \item 24 may.\ 2004, Laura Cecilia Saavedra Granja ``Desigualdad y
%    crecimiento econ\'omico: nuevos v\'inculos institucionales para la
%    teor\'ia de A.\ Alesina y D.\ Rodrik'' Lic.\ en Ciencia
%    Pol\'itica, ITAM.
%
%    \item 20 oct.\ 2003, Iv\'an Felipe Gonz\'alez Montoya ``Determinantes
%    de la satisfacci\'on y confianza con la democracia en M\'exico,
%    1999--2002''.
%
%    \item 9 sep.\ 2003, Arianna Magdalena S\'anchez Galindo ``Los jueces ante la
%    pol\'itica: la democracia y el ejercicio de control constitucional''
%   .
%
%    \item 15 jul.\ 2003, Javier Ram\'irez Guti\'errez ``Facultades
%    meta-constitucionales del ejecutivo estatal en Tamaulipas: un
%    estudio comparado de siete legislaturas'' Lic.\ en Ciencia
%    Pol\'itica, ITAM.
%
%    \item 2 jul.\ 2003, Gabriela Quijano Cuellar ``Las comisiones del Senado 1917-2000.  El impacto del cambio pol\'itico''.
%
%    \item 16 may.\ 2003, Ver\'onica Hoyo Ulloa ``La extrema derecha francesa: identidad, inmigraci\'on y voto, 1972--2003'' Lic.\ en Relaciones Internacionales, ITAM.
%
%    \item 15 nov.\ 2002, Alonso Ru\'iz Belmont ``\'Elites pol\'iticas en el M\'exico posrevolucionario''.
%
%    \item 14 jun.\ 2002, Erika Contreras Licea ``Contigo: estrategia de desarrollo social, 2001-2006. Un an\'alisis de g\'enero, pobreza y microcr\'editos''.
%
%    \item 30 oct.\ 2001, Fernanda Alva Ru\'iz Caba\~nas ``La cl\'ausula democr\'atica en el Acuerdo de Asociaci\'on Econ\'omica, Concertaci\'on Pol\'itica y Cooperaci\'on entre la Uni\'on Europea y M\'exico: soberan\'ia y comercio''.
%
%    \item 24 ago.\ 2001, Martha Patricia G\'omez Straffon ``La compra y coacci\'on del voto en las elecciones federales del a\~no 2000: resultados de una encuesta de salida''.
%
%    \item 7 jun.\ 2001, Mar\'ia del Rosario Aguilar Pariente ``Factores que determinan la participaci\'on de mujeres en asambleas: un estudio comparado''.
%
%    \item 14 sep.\ 2000, Mar\'ia Paola de San Nicol\'as Martorelli Hern\'andez ``El
%    lado azul de la c\'amara. El activismo legislativo del Partido Acci\'on
%    Nacional, 1946--2000'' Lic.\ en Ciencia Pol\'itica,
%    ITAM.
%
%    \item 6 sep.\ 2000, Pilar Campos Bola\~no ``Tres ensayos sobre la econom\'ia pol\'itica de las finanzas populares''.

    \end{itemize}

}


\condcomment{\boolean{includeDictamenes}}{
\section{Tesis dictaminadas en el ITAM (desde noviembre 2008)}

    \begin{itemize}

    \item 15 feb.\ 2010, Aileen Fern\'andez Villase\~nor ``El veto del ejecutivo al presupuesto de egresos de la federaci\'on''.

    \item 16 oct.\ 2009, Manuel Victoria G.\ ``El n\'umero de partidos y el
    gasto excluyente en los estados de M\'exico''.

    \item 20 ago.\ 2009, Ang\'elica Pulido ``La representaci\'on sustantiva de
    las mujeres en la C\'amara de Diputados''.

    \item 28 may.\ 2009, Dolores Bernal Fontanals ``El gobierno dividido en
    M\'exico: coaliciones ganadoras en la C\'amara de Diputados''.

    \item 15 mar.\ 2009, Ariel Berdichevski Cimet ``Durabilidad de las
    coaliciones gubernamentales: Israel, un estudio de caso'' (segunda revisi\'on).

    \item 11 mar.\ 2009, Ismael Estrada Mercado ``Coaliciones y disciplina
    parlamentaria en la legislatura del Estado de M\'exico''.

    \item 5 dic.\ 2008, Jorge Esquinca ``Reforma a los sistemas de
    pensiones en Am\'erica Latina''.

    \item 19 nov.\ 2008, Lorenzo G\'omez Mor\'in ``El voto cat\'olico en M\'exico a catorce a\~nos de las reformas constitucionales en materia religiosa, 1992--2006''.

    \end{itemize}

}


\section{Consultor\'ia}
\begin{itemize}

\item 2008--09, paper de fondo ``The Political Economy of Fiscal Reforms in Latin America: The Case of Mexico'' con Vidal Romero y Jeffrey F.\ Timmons, para el estudio intitulado \emph{The Political Economy of Fiscal Reforms in Latin America} coordinado por Mark Hallerberg.

\item 2006, paper de fondo ``El cambio en los patrones de gobierno de los estados de M\'exico: �Qu\'e indican sobre las direcciones futuras de la pol\'itica nacional?'' con Vidal Romero para el An\'alisis Institucional y de Gobernabilidad intitulado \emph{Gobernabilidad democr\'atica en M\'exico: m\'as all\'a de la captura del estado y la polarizaci\'on social}, coordinado por Philip Keefer (M\'exico DF: Banco Mundial-INAP-CIDE, 2007).

\end{itemize}

\section{Comentario en radio y TV}
\begin{itemize}

\item 20 may.\ 2010, Reforma pol\'itica: Iniciativas de ley, Programa ``Voces de la Democracia'' del Instituto Federal Electoral, Canal del Congreso, M\'exico DF.

\item 19 may.\ 2010, Reforma pol\'itica: Iniciativas de ley, Programa ``Voces de la Democracia'' del Instituto Federal Electoral, Radio UNAM, M\'exico DF (\textsc{fm96.1mh}z y \textsc{am860kh}z).

\item 18 nov.\ 2008, Mesa redonda sobre el presupuesto de egresos de la federaci\'on, Programa ``Punto y coma'' Canal 40, M\'exico DF.

\item 25 ago.\ 1995, ``Madrazo contra Zedillo'' Radio UNAM, M\'exico DF (\textsc{fm96.1mh}z y \textsc{am860kh}z).

\item 14 jul.\ 1995, ``Impuesto a la gasolina'' Radio UNAM, M\'exico DF (\textsc{fm96.1mh}z y \textsc{am860kh}z).

\item 7 jul.\ 1995, ``Masacre en Aguas Blancas, Gro.'' Radio UNAM, M\'exico DF (\textsc{fm96.1mh}z y \textsc{am860kh}z).

\item 30 jun.\ 1995, ``Relevan a Esteban Moctezuma'' Radio UNAM, M\'exico DF (\textsc{fm96.1mh}z y \textsc{am860kh}z).

\end{itemize}

\condcomment{\boolean{longcv}}{

\section{Lenguas}

Franc\'es e ingl\'es. }


\section{Referencias}
\begin{itemize}

\item \textsc{Gary Cox},
Professor of Political Science, Stanford \\
+1(650)723-4278, \url{gwcox@stanford.edu}.

\item \textsc{Paul Drake},
Professor of Political Science, UCSD \\
+1(858)534-6868, \url{pdrake@weber.ucsd.edu}.

\item \textsc{Mathew McCubbins},
Professor of Political Science and Law, Duke University \\
+1(919)660-4324, \url{mathew.mccubbins@duke.edu}.

\item \textsc{Wayne Cornelius},
CCIS Director Emeritus and Theodore Gildred Distinguished Professor Emeritus of Political Science, UCSD \\
+1(885)822-4447, \url{wcorneli@weber.ucsd.edu}.

\item \textsc{Federico Est\'evez},
Profesor de Ciencia Pol\'itica, ITAM \\
+52(55)5628-4000x3702, \url{festevez@itam.mx}.

% \item \textsc{Beatriz Magaloni},
% Professor of Political Science, Stanford \\
% +1(650)724-5949, \url{magaloni@stanford.edu}.

\item \textsc{Elisabeth Gerber},
Jack L.\ Walker, Jr.\ Professor of Public Policy, Gerald R.\ Ford School of Public Policy, Univ.\ of Michigan \\
+1(734)647-4004, \url{egerber@umich.edu}.

\end{itemize}


\end{document}


\newpage

\begin{center}
\textbf{\large{Sinopsis de mi tesis doctoral}} \\
``Bully pulpits: Posturing, bargaining, and polarization in the legislative process of the Americas''

por

Eric Magar \\
Doctor en Ciencia Pol\'itica \\
Universidad de California, San Diego, 2001 \\
Asesorada por los profesores Gary W. Cox y Paul W. Drake
\end{center}

Este es un estudio del proceso legislativo bajo la forma
presidencialista de separaci\'on del poder (SdeP).  Tres temas
convergen en mi trabajo.  Pregunto primero por qu\'e ocurre el
fen\'omeno del veto en sistemas de SdeP.  En Estados Unidos la
respuesta convencional es que los vetos son maniobras t\'acticas
propias de la normalidad democr\'atica y no son heraldos de una
inminente ruptura democr\'atica ni evidencia de par\'alisis
gubernamental; m\'as bien, son parte de la negociaci\'on cotidiana de la
pol\'itica p\'ublica.

\medskip En segundo lugar cuestiono si Estados Unidos es una excepci\'on entre los sistemas presidencialistas.  �Es esta la l\'ogica apropiada para entender los casos latinoamericanos de SdeP?  �Es apropiada para gobiernos a nivel sub-nacional?  En la medida en que los vetos son t\'acticas de negociaci\'on su uso deber\'ia variar de acuerdo con el contexto institucional.

\medskip En tercer lugar, inquiero sobre distintas explicaciones de la incidencia del veto.  �Son los vetos producto de informaci\'on incompleta entre quienes negocian la pol\'itica p\'ublica y, por tanto, errores de pol\'iticos ``miopes''?  �Son, acaso, estratagemas de negociaci\'on destinados a construir una reputaci\'on de rudeza a la luz de informaci\'on asim\'etrica?  �O bien es mejor entenderlos como argucias publicitarias, maniobras dirigidas al p\'ublico en b\'usqueda de su apoyo electoral?  La informaci\'on incompleta y el posicionamiento son factores que compiten en el plano te\'orico por explicar la incidencia del veto.

\medskip En mi tesis trabajo dos vetas en la literatura que comparten el tema (la SdeP) pero no al auditorio (anglo- y latino-americanistas), brindando as\'i un puente sobre esta brecha.  A lo largo de la investigaci\'on empleo una extensi\'on del modelo formal de negociaci\'on ejecutivo-legislativo (el modelo de agenda-setter) incorpor\'andole una motivaci\'on de posicionamiento que provoca vetos en distintas modalidades.  Pongo a prueba este modelo revisionista utilizando los gobiernos estatales de Estados Unidos. Posteriormente traslado mi investigaci\'on a Latinoam\'erica con un estudio de la incidencia del veto en Argentina y con tres estudios de caso de negociaci\'on entre poderes en Argentina, Chile y M\'exico.

\medskip El estudio de reg{\'i}menes presidencialistas en la \'ultima d\'ecada ha cambiado el \'enfasis de la disfuncionalidad de la par\'alisis (p. ej., Linz, Sunquist) a un amplio rango de maniobras t\'acticas que ofrece la SdeP a los pol\'iticos (p. ej., Kernell, Cameron).  Recurro a la literatura reciente para analizar una modalidad de relaciones ejecutivo-legislativo mucho m\'as rica, variada e interesante.



\condcomment{\boolean{longcv}}{
\bigskip
\textbf{\large{}}
\begin{itemize}
\item
\end{itemize}
}
