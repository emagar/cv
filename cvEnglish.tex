\documentclass[11 pt, letter]{article}

\usepackage[letterpaper,right=1in,left=1in,top=1in,bottom=1in]{geometry}
\usepackage{ae} % or {zefonts}
\usepackage[T1]{fontenc}
\usepackage[ansinew]{inputenc}
\usepackage{amsmath}
\usepackage{amssymb}
\usepackage{url}
\usepackage{setspace} %allows to change linespacing
%\usepackage{dashrule} %allows ro draw dotted lines
\usepackage{color}   % allows usage of color fonts
\definecolor{light-gray}{gray}{0.65} % defines color and its name
\definecolor{NavyBlue}{rgb}{0.,0,0.5}

% avoid clubs and widows
\clubpenalty=10000 \widowpenalty=10000
% \displaywidowpenalty=10000

\parindent=0mm  % removes indent at start of paragraphs

%% Drops space between and above items and enumerates
\let\olditemize=\itemize
\let\endolditemize=\enditemize
\renewenvironment{itemize}{%
    \vspace*{-1.5\parsep}%
    \begin{olditemize}%
      \setlength{\parskip}{0.1\parskip}%
      \setlength{\itemsep}{0.1\itemsep}%
  }%
  {%
    \end{olditemize}%
  }
\let\oldenumerate=\enumerate
\let\endoldenumerate=\endenumerate
\renewenvironment{enumerate}{%
    \vspace*{-1.5\parsep}%
    \begin{oldenumerate}%
      \setlength{\parskip}{0.1\parskip}%
      \setlength{\itemsep}{0.1\itemsep}%
  }%
  {%
    \end{oldenumerate}%
  }

\setstretch{1} %linespacing
%\onehalfspacing
%\doublespacing

% SET LONG OR SHORT VERSION OF CV
\usepackage{ifthen}
\newboolean{longcv}
\setboolean{longcv}{false}  % switch true, false, to produce long, short version of cv
\newcommand{\condcomment}[2]{\ifthenelse{#1}{#2}{}}
%  para usarlo se pone
%  \condcomment{\boolean{longcv}}{
%   ...
%  }

\usepackage[colorlinks=true,urlcolor=NavyBlue]{hyperref}

\usepackage{Tabbing} % allows accented letters (e.g. \'a) in tabbing environment without corrupting. \=, \+, and \> become \TAB=, \TAB+, and \TAB=. See http://cs.brown.edu/about/system/software/latex/doc/Tabbing.pdf

\usepackage{mathptmx}           % set font type to Times
\usepackage[scaled=.90]{helvet} % set font type to Times (Helvetica for some special characters)
\usepackage{courier}            % set font type to Times (Courier for other special characters)

\begin{document}

\textbf{\LARGE{Eric Magar}} \\
\emph{\Large{curriculum vitae}}   \\ [-2ex]
\makebox[\textwidth][r]{\hrulefill
\textcolor{light-gray}{\tiny{~updated \today, newer version at}}} \\ [-1.5ex]
\makebox[\textwidth][r]{
\textcolor{light-gray}{\tiny{\href{http://emagar.github.io/cv/}{\texttt{http://emagar.github.io/cv/}}}}}


\medskip

\begin{Tabbing}

\textbf{Current position:} \TAB= Professor, \href{http://politica.itam.mx/}{Department of Political Science}, \href{http://www.itam.mx/}{ITAM}, Mexico. \\
\\
\textbf{Contact info:} \TAB= \TAB+ ITAM-Ciencia Pol\'itica, R\'io Hondo 1, Col.\ Tizap\'an San Angel, 01000 M\'exico DF. \\ Tel. +52(55)5628-4079, Fax +52(55)5490-4672.  \href{mailto:emagar@itam.mx}{\small{\nolinkurl{emagar@itam.mx}}} \TAB- \\
\\
\textbf{Personal info:} \TAB= \TAB+ Mexican and French citizen. Born
January 12, 1970 in Mexico City. \\ Married, proud father of Aurelia and Le�n Mart�n. \TAB- \\
\\
\textbf{Google Scholar profile:} \TAB= \TAB+ \href{http://scholar.google.com.mx/citations?user=Rs7pVXQAAAAJ}{\small{\nolinkurl{http://scholar.google.com.mx/citations?user=Rs7pVXQAAAAJ}}} \TAB- \\
\\
\textbf{Publons:} \TAB= \TAB+
\href{https://publons.com/researcher/1634653/eric-magar/}{\small{\nolinkurl{https://publons.com/researcher/1634653}}} \TAB- \\
\\
\textbf{ORCID:} \TAB= \TAB+ \href{http://orcid.org/0000-0002-6766-1677}{0000-0002-6766-1677} \TAB- \\
\\
\textbf{Web of Science ResearcherID:} \TAB= \TAB+ S-6603-2019 \TAB- \\
\\
\textbf{Web page:} \TAB= \TAB+ \href{http://emagar.github.io}{\small{\nolinkurl{http://emagar.github.io}}} \TAB- \\

\end{Tabbing}



\bigskip

\section{Education}

\begin{itemize}

\item BA in Political Science (1994, Instituto Tecnol\'ogico Aut\'onomo de M\'exico).

\item PhD in Political Science (2001, University of California at San
    Diego). \\ \textbf{Dissertation:} ``\href{https://github.com/emagar/cv/blob/master/papers/magar-diss2001ucsd.pdf}{Bully Pulpits: Posturing, Bargaining, and Polarization in the Legislative Process of the Americas}'' co-advised by Gary W.\ Cox and Paul W.\ Drake.
    \condcomment{\boolean{longcv}}{Other committee members: Elizabeth Gerber, Matthew S.\ Shugart and Carlos Waisman.}
    Citations: \href{http://allman.rhon.itam.mx/~emagar/cv/citasEMM.pdf}{6}.
    %(The last page of this CV contains a dissertation abstract.)

\end{itemize}



\section{Research interests}

\begin{itemize}

\item Comparative Political Institutions; legislative process; elections and electoral systems; \condcomment{\boolean{longcv}}{Latin- and Anglo-American politics; }formal theory; applied statistics.

\end{itemize}



\condcomment{\boolean{longcv}}{

\section{Other studies}

\begin{itemize}

\item Summer 1992, ICPSR courses in linear regression analysis and game theory, University of Michigan, Ann Arbor.

\end{itemize}
}




\section{Employment}

\begin{itemize}

\item 2014--15, Visiting Scholar, \href{https://polisci.wustl.edu/}{Department of Political Science}, Washington University in St.\ Louis.

\item 2000--present, Associate Professor, \href{http://politica.itam.mx/}{Department of Political Science}, ITAM.

\item 2007, Visiting Professor, \href{http://www.ibei.org}{Institut Barcelona d'Estudis Internacionals}, Barcelona, Spain.

\condcomment{\boolean{longcv}}{\item 2007, Visiting Professor,
Departamento de Ciencias Pol\'iticas y Sociales, Universidat Pompeu
Fabra, Barcelona, Spain.}

\item 2004--2007, Chair, \href{http://politica.itam.mx/}{Department of Political Science}, ITAM.

\item 2000--2003, Director of Undergraduate Studies, ITAM.

\condcomment{\boolean{longcv}}{\item 1998--2000, Research Assistant
for Gary W.\ Cox and Mathew D.\ McCubbins,
\href{http://polisci.ucsd.edu/}{Department of Political Science},
UCSD.}

\condcomment{\boolean{longcv}}{\item 1995--1998, Research Assistant,
\href{http://usmex.ucsd.edu/}{Center for US-Mexican
Studies}, UCSD.}

\item 1994, Advisor to the Director Ejecutivo de Prerrogativas y Partidos Pol\'iticos, \href{http://www.ife.org.mx}{Instituto Federal Electoral}, Mexico City.

\condcomment{\boolean{longcv}}{\item 1992--1993, Research Assistant,
Departamento de Ciencias Sociales, ITAM.}

\end{itemize}





\section{Awards}

\begin{itemize}

\condcomment{\boolean{longcv}}{
\item 2014, SSRN Top Ten Paper Award (``Effects of automated redistricting'').

\item 2011, SSRN Top Ten Paper Award (``Factions with clout'').

\item 2010, SSRN Top Ten Paper Award (``Partisanship among the Experts'').

\item 2010, SSRN Top Ten Paper Award (``Legalist vs.\ Interpretativist'').

\item 2009, SSRN Top Ten Paper Award (``Factions with clout'').

\item 2008, SSRN Top Ten Paper Award (``Gubernatorial coattails'').
}

\item 2017--present, Level II Researcher, \href{http://www.conacyt.gob.mx/SNI/Index_SNI.html}{Sistema Nacional de Investigadores}, Mexico.

\item 2014, Premio al M\'erito Profesional en el Sector Acad\'emico, ex-ITAM (former alumni association). 

\item 2005--2016, Level I Researcher, Sistema Nacional de Investigadores, Mexico.

\item 2002--2005, Researcher candidate, Sistema Nacional de Investigadores, Mexico.

\item 2001, The \href{http://polisci.ucsd.edu/about/recognition.html}{Peggy Quon Prize} for distinguished research contribution to political science, \\
    Dept.\ of Political Science, UCSD.

\item 1994, Premio Los Mejores Estudiantes de M\'exico,
    \condcomment{\boolean{longcv}}{Ateneo Nacional de Ciencia y Tecnolog\'ia,} \emph{El Diario de M\'exico}.

\item 1994, Premio a la Excelencia Acad\'emica Miguel Palacios Macedo,
    ITAM.

\item 1994, \href{http://politica.itam.mx/exalumnos/documentos/ex-alumnos_cp.pdf}{Honors Mention} in the BA thesis defense,
    ITAM.

\condcomment{\boolean{longcv}}{\item 1988, Mention Assez-Bien in the
Baccalaur\'eat exams, \href{http://www.lfm.edu.mx/}{Lyc\'ee
Franco-Mexicain}, Mexico City.}

\end{itemize}




% \section{Grants and scholarships}

% \begin{itemize}

% \item 2014--15, Beca Fulbright-Garc\'ia Robles para Estancias de Investigaci\'on en Estados Unidos.
%     \condcomment{\boolean{longcv}}{US\$24,500.}

% \item 2008--present, Dean of Economics, Law, and Social Sciences Research Grant (annual), ITAM.
%     \condcomment{\boolean{longcv}}{US\$5,000 yearly.}

% \item 2005, Funding for a project on women representation in the Mexican Chamber of Deputies in the 1994--2003 period,
%      \href{http://www.inmujeres.gob.mx/}{INMUJERES} and \href{http://www.conacyt.mx/}{CONACYT}, Mexico.
%     \condcomment{\boolean{longcv}}{Mx\$232,000.}

% \item 2004--2006, Dean of Economics, Law, and Social Sciences Research Grant (annual), ITAM.
%     \condcomment{\boolean{longcv}}{US\$5,000 yearly.}

% \condcomment{\boolean{longcv}}{\item 1999, Dean's Social Science
%     Travel Research Fund Grant, \href{http://weber.ucsd.edu/}{Division of Social Sciences}, UCSD.
%     US\$1,500.

% \item 1999, \href{http://cilas.ucsd.edu/}{Center for Iberian and Latin American Studies} Field Research Grant, UCSD.
%     US\$1,500.

% \item 1998, \href{http://icenter.ucsd.edu/pao/friends.htm}{Friends of the International Center scholarship}, UCSD.
%     US\$1,000.

% \item 1995--1998, Beca-cr\'edito CONACYT para estudios de posgrado
%     (partial stipend for living expenses),
%     \href{http://www.conacyt.mx/}{CONACYT}, Mexico.
%     US\$21,390.

% \item 1995--1997, Ford-McArthur scholarship for graduate studies
%     (partial stipend for living expenses),
%     \href{http://www.iie.org/}{Institute of International Education}, New York.
%     US\$12,400.}

% \end{itemize}

% \condcomment{\boolean{longcv}}{\medskip \makebox[10mm]{} \emph{Total $\approx$ US\$75,000}.}





\section{Publications}

\subsection{In peer-reviewed journals}

    \begin{itemize}

    \end{itemize}

  \item \href{https://doi.org/10.1086/710015}{``Presidents on the Fast Track: Fighting Floor Amendments with Restrictive Rules''} with Valeria Palanza and Gisela Sin, \emph{The Journal of Politics}  vol.~83, num.~2 (2021) pp.~633--46. %https://ssrn.com/abstract=3514866 (\href{https://doi.org/10.1086/710015}{https://doi.org/10.1086/710015}). 

    \item \href{http://cienciassociales.edu.uy/institutodecienciapolitica/rucp-vol-26-no-1/}{``La partici\'on de un cartel: votaciones nominales y guerra faccional en la Asamblea Legislativa del Distrito Federal''} \emph{Revista Uruguaya de Ciencia Pol\'itica} vol.~26, num.~1 (2017) pp.~35--58.
      
    \item \href{http://dx.doi.org/10.1016/j.polgeo.2016.11.015}{``Components of partisan bias originating from single-member districts in multi-party systems: The case of Mexico''} with Alejandro Trelles, Micah Altman, and Michael P.\ McDonald, \emph{Political Geography} vol.\ 57, March 2017, pp.~1--12. 

    \item \href{http://www.politicaygobierno.cide.edu/index.php/pyg/article/view/825/606}{``Datos abiertos, transparencia y redistritaci\'on en M\'exico''} with Alejandro Trelles, Micah Altman y Michael P.\ McDonald, \emph{Pol\'itica y Gobierno} vol.\ 23, num.\ 2, $2^{nd}$ semester 2016, , pp.\ 331--64 (also appeared in English \href{http://www.politicaygobierno.cide.edu/index.php/pyg/article/view/822/614}{here}).

    \item \href{http://www.politicaygobierno.cide.edu/index.php/pyg/article/view/18}{``Consideraciones metodol\'ogicas para estudiantes de pol\'itica legislativa mexicana: sesgo por selecci\'on en votaciones nominales''} with Francisco Cant\'u and Scott Desposato, \href{http://www.politicaygobierno.cide.edu/}{\emph{Pol\'itica y Gobierno}} vol.\ 21, num.\ 1, $1^{st}$ semester 2014, pp.\ 25--53.

    \item \href{https://github.com/emagar/cv/blob/master/papers/magarGuberCoattails2012jop.pdf}{``Gubernatorial Coattails in Mexican Congressional Elections''} \href{http://www.journalofpolitics.org/}{\emph{The Journal of Politics}} vol.\ 74 no.\ 2, Apr.\ 2012, pp.\ 383--99. \condcomment{\boolean{longcv}}{In April 2008 a preliminary version made it to SSRN's Top Ten Paper list. }Citations: \href{http://allman.rhon.itam.mx/~emagar/cv/citasEMM.pdf}{2}.

    \item \href{https://github.com/emagar/cv/blob/master/papers/magarMoraes-Factions2012pp.pdf}{``Factions with clout: presidential cabinet coalition and policy in the Uruguayan Parliament''} with Juan Andr\'es Moraes, \href{http://ppq.sagepub.com/cgi/content/abstract/18/3/427}{\emph{Party Politics}} vol.\ 18 num.\ 3 (May 2012) pp.\ 427--51. Citations: \href{http://allman.rhon.itam.mx/~emagar/cv/citasEMM.pdf}{3}.\condcomment{\boolean{longcv}}{In April 2009 and Sept.\ 2011 it made it to SSRN's Top Ten Paper list.}

    \item \href{http://www.fcs.edu.uy/icp/downloads/revista/RUCP17/RUCP-17-02_Magar&Moraes.pdf}{``Coalici\'on y resultados: aprobaci\'on y duraci\'on del tr\'amite parlamentario en Uruguay (1985--2000)''} with Juan Andr\'es Moraes, \href{http://www.fcs.edu.uy/icp/revista.htm}{\emph{Revista Uruguaya de Ciencia Pol\'itica}} vol.\ 17, num.\ 1 (Jan.-Dec.\ 2008), pp.\ 39--70. Citations: \href{http://allman.rhon.itam.mx/~emagar/cv/citasEMM.pdf}{1}.
    %ISSN:0797-9789
    %http://www.fcs.edu.uy/icp/downloads/revista/RUCP16/RevistaICP17-02.pdf

    \item \href{https://github.com/emagar/cv/blob/master/papers/estevezMagarRosasIfeElecStud2008.pdf}{``Partisanship in Non-Partisan Electoral Agencies and Democratic Compliance: Evidence from Mexico's Federal Electoral Institute''} with Federico Est\'evez and Guillermo Rosas, \href{http://dx.doi.org/10.1016/j.electstud.2007.11.013}{\emph{Electoral Studies}} vol.\ 27, num.\ 2 (June 2008), pp.\ 257--71. Citations: \href{http://allman.rhon.itam.mx/~emagar/cv/citasEMM.pdf}{8}.
    %ISSN:0261-3794

    \item \href{http://www.scielo.cl/scielo.php?script=sci_arttext&pid=S0718-090X2008000100013&lng=es&nrm=iso}{``M\'exico: reformas pese a un gobierno dividido''} with Vidal Romero, \href{http://www.scielo.cl/scielo.php?pid=0718-090X&script=sci_serial}{\emph{Revista de Ciencia Pol\'itica}}
    vol.\ 28, num.\ 1 (June 2008), pp.~265--85. Citations: \href{http://allman.rhon.itam.mx/~emagar/cv/citasEMM.pdf}{1}
    %ISSN:0716-1417

    \item \href{https://github.com/emagar/cv/blob/master/papers/magar-romero-Accidentada-consolidacion2007rcp.pdf}{``M\'exico en 2006: la accidentada consolidaci\'on democr\'atica''} with Vidal Romero, \href{http://www.scielo.cl/scielo.php?pid=0718-090X&script=sci_serial}{\emph{Revista de Ciencia Pol\'itica}} vol.\ 27, special num.\ (July 2007), pp.~184--204. Citations: \href{http://allman.rhon.itam.mx/~emagar/cv/citasEMM.pdf}{2}.
    %ISSN:0716-1417

    \item \href{https://github.com/emagar/cv/blob/master/papers/cox-magar1999apsr.pdf}{``How Much is Majority Status in the U.S. Congress Worth?''} with Gary W.\ Cox, \href{http://www.jstor.org/pss/2585397}{\emph{American Political Science Review}} vol.\ 93, num.\ 2 (June 1999), pp.\ 299--309. Citations: \href{http://allman.rhon.itam.mx/~emagar/cv/citasEMM.pdf}{35}.

    \item \href{https://github.com/emagar/cv/blob/master/papers/magar-rosenblum-samuels1998.pdf}{``On the Absence of Centripetal Incentives in Double-Member Districts: The Case of Chile''} with Marc R.\ Rosenblum and David Samuels, \href{http://cps.sagepub.com/cgi/content/abstract/31/6/714}{\emph{Comparative Political Studies}} vol.\ 31, num.\ 6 (December 1998), pp.\ 714--39. Citations: \href{http://allman.rhon.itam.mx/~emagar/cv/citasEMM.pdf}{40}.

    \end{itemize}

% \subsection{In non-refereed journals}

%     \begin{itemize}

%     \item \href{https://github.com/emagar/cv/blob/master/papers/magarGCP2009.pdf}{``El inmovilismo democr\'atico: un modelo de relaciones ejecutivo-legislativo en reg\'imenes con poderes separados''},
%     \href{http://gacetadecienciapolitica.itam.mx/}{\emph{La Gaceta de Ciencia Pol\'itica}} vol.\ 6, num.\ 1 (Fall/Win.\ 2009), pp.~11--25.

%     \item \href{https://github.com/emagar/cv/blob/master/papers/MagarRomeroImpasseFAeE2007.pdf}{``El impasse mexicano desde la perspectiva latinoamericana''} with Vidal Romero,     \href{http://fal.itam.mx/FAE/}{\emph{Foreign Affairs en espa\~nol}} vol.\ 7, num.\ 1 (Jan.--Mar.\ 2007), pp.~117--31. Citations: \href{http://allman.rhon.itam.mx/~emagar/cv/citasEMM.pdf}{1}.
%     %ISSN:1665-1707

%     \end{itemize}

\subsection{Chapters in edited volumes}

    \begin{itemize}

    \item \href{https://github.com/emagar/cv/blob/master/papers/legdeb04.pdf}{``Ch.\ 28 Mexico: Parties and floor access in Mexico's C\'amara de Diputados''} in \emph{The Politics of Legislative Debate around the World}, ed.\ Hanna B\"ack, Marc Debus y Jorge M. Fernandes (Oxford: Oxford University Press, 2021, pp.\ , ISBN ).  %\textbf{\color{red}\ref{ncites:magar.debate.dips.2021} citas documentadas}}.
      
    \item \href{https://github.com/emagar/cv/blob/master/papers/magarTheElectoralInstitutions2015jhup.pdf}{``The electoral institutions: party subsidies, campaign decency, and entry barriers''} in \emph{Mexico's Evolving Democracy: A Comparative Study of the 2012 Elections}, ed.\ Jorge I. Dom\'inguez, Kenneth G. Greene, Chappell Lawson and Alejandro Moreno (Washington DC: Johns Hopkins University Press, 2015, pp.\ 63--85). 

    \item \href{https://github.com/emagar/cv/blob/master/papers/ePod03.pdf}{``Los contados cambios al equilibrio de poderes''} in \emph{Reformar sin mayor\'ias: La din\'amica del cambio constitucional en M\'exico, 1997--2012}, edited by Mar\'ia Amparo Casar and Ignacio Marv\'an Laborde (Mexico City: Taurus, 2014, pp.~259--94).

    \item \href{https://github.com/emagar/cv/blob/master/papers/sanchezMagaloniMagarLegalistInterpret2011cup.pdf}{``Legalist vs.\ Interpretativist: The Supreme Court and Democratic Transition in Mexico''} with Beatriz Magaloni and Arianna S\'anchez in \emph{Courts in Latin America}, edited by Gretchen Helmke and Julio R\'ios Figueroa (New York: Cambridge University Press, 2011). Translated to \href{https://github.com/emagar/cv/blob/master/papers/sanchezMagaloniMagarLegalistaInterpret2011espanol.pdf}{Spanish} in \emph{Tribunales constitucionales de Am\'erica Latina}, edited by Gretchen Helmke and Julio R\'ios Figueroa (Mexico City: Suprema Corte de Justicia de la Naci\'on, 2011). \condcomment{\boolean{longcv}}{In January 2010 it made it to SSRN's Top Ten Papers list. }Citations: \href{http://allman.rhon.itam.mx/~emagar/cv/citasEMM.pdf}{1}.

    \item \href{https://github.com/emagar/cv/blob/master/papers/magar-middlebrook.pdf}{``National Election Results for Argentina, Brazil, Chile, Colombia, El Salvador, Peru, and Venezuela during the 1980s and 1990s''} with Kevin J.\ Middlebrook, statistical appendix in \emph{Conservative Parties, the Right, and Democracy in Latin America}, edited by Kevin J.\ Middlebrook (Baltimore: Johns Hopkins University Press, 2000).

    \item \href{https://github.com/emagar/cv/blob/master/papers/MagarMolinar1995.pdf}{``Medios de comunicaci\'on y democracia''} with Juan Molinar in \emph{Elecciones, di\'alogo y reforma: M\'exico 1994}, vol.\ 2, edited by Jorge Alcocer and Jorge Carpizo (Mexico City: Centro de Estudios para un Proyecto Nacional Alternativo, 1996, pp.\ 145--42). Citations: \href{http://allman.rhon.itam.mx/~emagar/cv/citasEMM.pdf}{3}.

    \end{itemize}

\subsection{Edited book}

    \begin{itemize}

    \item \href{https://github.com/emagar/cv/blob/master/papers/huerta-magar-Mujeres-legisladoras-mexico2006book.pdf}{\emph{Mujeres legisladoras en M\'exico: avances, obst\'aculos, consecuencias y propuestas}}, with Magdalena Huerta Garc\'ia (Mexico City: INMUJERES-CONACYT-Friedrich Ebert Stiftung-ITAM, 2006, ISBN 970-95099-0-X, 575 pp.) Citations: \href{http://allman.rhon.itam.mx/~emagar/cv/citasEMM.pdf}{1}.


    \end{itemize}

\subsection{In magazine}

    \begin{itemize}

    \item \href{http://www.nexos.com.mx/?P=leerarticulo&Article=291}{``IFE: La casa de la partidocracia''} with Federico Est\'evez
    and Guillermo Rosas, \emph{Nexos} 376 (April 2009), pp.\ 124--7.

    \end{itemize}


\subsection{In working paper series}

    \begin{itemize}

    \item \href{http://papers.ssrn.com/abstract=1159621}{``Of Coalition and Speed: Passage and Duration of Statutes in Uruguay's Parliament, 1985--2000''} with Juan Andr\'es Moraes, IBEI working papers num.\ 2008/15 ISSN:1886-2802 (June 2008).

    \end{itemize}


\subsection{Book reviews}

    \begin{itemize}

    \item ``\href{https://github.com/emagar/cv/blob/master/talks/statonJudPowStratCommlaps2011PUBLISHEDVERSOIN.pdf}{Judicial Power and Strategic Communication in Mexico. By Jeffrey K. Staton},''
    \href{http://www.blackwellpublishing.com/journal.asp?ref=1531-426x}{\emph{Latin American
    Politics and Society}} vol.\ 53, num.\ 3 (Fall 2011), pp.\ 185--8.
    %ISSN:1531-426X (online ISSN:1548-2456)
    % o bien statonJudPowStratCommSUBMITTEDVERSION.pdf

    \item ``\href{http://www.politicaygobierno.cide.edu/Vol_XVII_N2_2010/06_PyG-Resenas_380-408.pdf}{Informal Coalitions and Policymaking in Latin America:
    Ecuador in Comparative Perspective. Por Andr\'es Mej\'ia Acosta},''
    \href{http://www.politicaygobierno.cide.edu/}{\emph{Pol\'itica y Gobierno}} vol.\ 17, num.\ 2 ($2^{nd}$ semester 2010), pp.\ 384--6.
    %ISSN:1665-2037

    \item \href{http://muse.uq.edu.au/journals/the_americas/v062/62.3magar.pdf}{``Ambition, Federalism, and Legislative Politics in Brazil. By David
    Samuels,''} \href{http://muse.uq.edu.au/journals/the_americas/toc/tam62.3.html}{\emph{The Americas}} vol.\ 62, num.\ 3 (January 2006), pp.~498--500.
    %ISSN:0003-1615

    \item \href{http://redalyc.uaemex.mx/src/inicio/ArtPdfRed.jsp?iCve=11500311}{``Nested Games: Rational Choice in Comparative Politics de George
    Tsebelis,''} \emph{Perfiles Latinoamericanos de FLACSO} vol.\ 2, num.\ 3 (December 1993), pp.~194--8.

    \end{itemize}

%% \subsection{Work under peer review}

%%     \begin{itemize}

      
%%     \end{itemize}
    
% \subsection{Unpublished manuscripts and conference papers}

%     \begin{itemize}

%     \item \href{http://ssrn.com/abstract=2486885}{``The Effects of Automated Redistricting and Partisan Strategic Interaction on Representation: The Case of Mexico''} with Micah Altman, Michael P.\ McDonald, and Alejandro Trelles (Aug.\ 2014).

%     \item \href{http://ssrn.com/abstract=2103866}{``The Veto as Electoral Stunt: EITM and a Test with Comparative Data''} (April 2013).

%     \item \href{http://ssrn.com/abstract=1808960}{``When cartels split: roll call votes and majority factional warfare in the Mexico City Assembly''} (Mar.\ 2011).

%     \item \href{http://ssrn.com/abstract=1683498}{``Partisanship among the experts: the dynamic party watchdog model of IFE, 1996--2011''} with Guillermo Rosas and Federico Est\'evez (April 2013). \condcomment{\boolean{longcv}}{In October 2010 it made it to SSRN's Top Ten Paper list. }Citations: \href{http://allman.rhon.itam.mx/~emagar/cv/citasEMM.pdf}{2}.

%     \item \href{http://allman.rhon.itam.mx/~emagar/cv/pdfs/ePod03.pdf}{``Los contados cambios al equilibrio de poderes''} (October 2012).

%     %\item ``Who calls the tune? The search for constituent effects in the Mexican Congress'' with Federico Est\'evez (April 2012).

%     \item \href{http://ssrn.com/abstract=1808960}{``When cartels split: roll call votes and majority factional warfare in the Mexico City Assembly''} (March 2011).

%     \item \href{http://ssrn.com/abstract=1642431}{``No self-control: Decentralized agenda power and the dimensional structure of the Mexican Supreme Court''} with Beatriz Magaloni and Arianna S\'anchez (August 2010).

%     \item \href{http://ssrn.com/abstract=1991066}{``The constructive veto and parliamentary discipline''} (April 2010).

%     \item \href{http://ssrn.com/abstract=1963863}{``The political economy of fiscal reforms in Latin America: Mexico''} with Vidal Romero and Jeffrey Timmons, commissioned by the Research Department of the Inter-American Development Bank (March 2009).
%         %% Also availaboe at http://allman.rhon.itam.mx/~emagar/mywork/IADB-Fiscal_reform_Mexico_Chapterv30.pdf

%     \item \href{http://ssrn.com/abstract=1991076}{``Judges' law: ideology and coalitions in Mexico's Election Tribunal 1996--2006''} with Federico Est\'evez (November 2008).

% %    \item ``The constructive veto and floor cooperation under open rule: factional co-sponsoring in Uruguay, 1985--2005'' with Juan Andr\'es Moraes (April 2008).

%     \item \href{http://ssrn.com/abstract=1491112}{``The Incidence of Executive Vetoes in Comparative Perspective: The Case of American State Governments, 1979--1999''} (February 2007).

%     \item \href{http://ssrn.com/abstract=1486804}{``A Model of Executive Vetoes as Electoral Stunts with Testable Hypotheses''} (February 2007).

%     \item ``The Circuitous Path to Democracy: Legislative Control of the Bureaucracy in Presidencial Systems, the Case of Mexico'' with Alejandra R\'ios C\'azares (October 2004).

%     \item ``El efecto de arrastre de gobernadores en elecciones de diputados federales de M\'exico, 1979--2003'' (October 2004).

%     \item \emph{Bully pulpits: Veto politics in the legislative process of the Americas}, book manuscript (May 2003).

%     %\item ``The Pulse of Uruguayan Politics: Factions, Elections, and Veto Incidence, 1985--2002'' with Juan Andr\'es Moraes (April 2003).

%     \item \href{http://ssrn.com/abstract=1991099}{``The Paradox of the Veto in Mexico (1917--1997)''} with Jeffrey A.\ Weldon (September 2001). Citations: \href{http://allman.rhon.itam.mx/~emagar/cv/citasEMM.pdf}{6}.

%     \item \href{http://ssrn.com/abstract=1400409}{``Making Sound out of Fury: The Posturing Use of Vetoes in Chile and in Mexico''} (August 2001).

%     \item \href{http://ssrn.com/abstract=1400408}{``The Elusive Authority of Argentina's Congress: Decrees, Statutes, and Veto Incidence, 1983--1994''} (August 2001). Citations: \href{http://allman.rhon.itam.mx/~emagar/cv/citasEMM.pdf}{3}.

%     \condcomment{\boolean{longcv}}{\item ``Decrees and Statutes in Argentina, 1983--1994: Polarization and Usurpation of Authority or Bargaining and Concessions?'' (June 2000).}

%     \condcomment{\boolean{longcv}}{\item ``The Interplay of Separation of Power and the Legislative Process'' (May 2000).}

%     \item \href{http://ssrn.com/abstract=1991086}{``Veto Bargaining and Coalition Formation: A Theory of Presidential Policymaking with Application to Venezuela''} with Octavio Amorim Neto (March 2000). Citations: \href{http://allman.rhon.itam.mx/~emagar/cv/citasEMM.pdf}{1}.

%     \item \href{http://ssrn.com/abstract=1400402}{``The Value of Majority Status in the US House''} with Gary W.\ Cox (May 1999).

%     \condcomment{\boolean{longcv}}{\item ``Patterns of Executive-Legislative Conflict in Latin America and the US'' (May 1999).}

%     \item \href{http://ssrn.com/abstract=1400403}{``The Deadlock of Democracy Revisited: A model of Executive-Legislative Relations in Separation-of-Power Regimes''} (September 1998).

%     \item ``Elecciones municipales en el norte de M\'exico, 1970--1993: Bases de apoyo partidistas y alineaciones electorales'' BA thesis (July 1994). Citations: \href{http://allman.rhon.itam.mx/~emagar/cv/citasEMM.pdf}{2}.

%     \end{itemize}

% \subsection{Open editorials}

%     \begin{itemize}

    %% \item \href{http://redaccion.nexos.com.mx/?p=9684}{``Saldos de la reelecci�n municipal''}
    %% \emph{Nexos en l\'inea}, Friday 26 October 2018.

%     \item \href{http://redaccion.nexos.com.mx/?p=652}{``Quid pro quo''}
%     \emph{Nexos en l\'inea}, Monday 25 January 2010.

%     \item \href{http://redaccion.nexos.com.mx/?p=322}{``Para que bailen al son de sus representados''}
%     \emph{Nexos en l\'inea}, Wednesday 2 December 2009.

%     \item \href{http://laloncheria.com/2009/11/02/impuestos-presupuesto-partidos/}{``La ecuaci\'on que no cuadra: impuestos vs.\ gasto de partidos''} LaLoncheria.com, Monday 2 November 2009.

%     \item ``Fallos esperanzadores'' \emph{El Centro}, Thursday 4 September 2008, p.~14.

%     \item ``Crimen y castigo'' \emph{El Centro}, Thursday 21 August 2008, p.~17.

%     \item ``El sufragio sin reelecci\'on'' \emph{El Centro}, Thursday 7 August 2008, p.~15.

%     \item ``El presidente mulato'' \emph{El Centro}, Thursday 24 July 2008, p.~12.

%     \item ``Forever young'' \emph{El Centro}, Thursday 10 July 2008, p.~15.

%     \item ``La maldici\'on del oro negro'' \emph{El Centro}, Thursday 26 June 2008, p.~17.

%     \item ``Marabunta'' \emph{El Centro}, Thursday 12 June 2008, p.~7.

%     \item ``Los piratas del Cofipe'' \emph{El Centro}, Thursday 29 May 2008, p.~13.

%     \item ``Juego en dos canchas'' \emph{El Centro}, Thursday 15 May 2008, p.~10.

%     \item ``La equidad de g\'enero: pendiente en el Congreso'' with Magdalena Huerta Garc\'ia, \emph{La Jornada} (Masiosare section, num.\ 420), Sunday 8 January 2006, p.~5.

%     \item ``Un superveto para el Ejecutivo'' with Jeffrey A.\ Weldon, \emph{Reforma}, Tuesday 17 April 2001, p.~6-Negocios.

%     \item ``Los vetos no son excepcionales'' with Jeffrey A.\ Weldon, \emph{Reforma}, Saturday 17 March 2001, p.~\textsc{5a}.

%     \end{itemize}

\section{Public data bases}

\begin{itemize}

   \item ``Recent Mexican electoral geography'' repository, \url{https://github.com/emagar/mxDistritos}, GitHub (updated Nov.\ 26, 2019). 
  
   \item ``Recent Mexican Election Vote Returns'' repository, \url{https://github.com/emagar/elecRetrns}, GitHub (updated Nov.\ 20, 2019). 
  
   \item  ``Roll call votes data for the Mexican Chamber of Deputies'' repository, \url{https://github.com/emagar/dipMex}, GitHub (updated Sep.\ 11, 2018). 
  
   \item  ``Consecutive reelection institutions and electoral calendars since 1994 in Mexico'', \url{http://dx.doi.org/10.7910/DVN/X2IDWS}, Harvard Dataverse (Apr.\ 2017). 
  
   \item ``M\'exico Estatal: Conformaci\'on de las Legislaturas Locales de M\'exico por Partido Pol�tico, por Principio de Representaci\'on y por G\'enero de 1985--2015'' with Alejandra R\'ios C\'azares, \href{http://biiacs-dspace.cide.edu/handle/10089/17392}{\url{http://datos.cide.edu/handle/10089/17392}}, Laboratorio Nacional de Politicas P\'ublicas CIDE (Sep.\ 2015).
     
%    \item ``Mexican Chamber of Deputies Roll Call Votes 2006--2012'' con Francisco Cant\'u y Scott Desposato, \url{http://ericmagar.com/data/rollcall/dipFed/} and \url{https://github.com/emagar/dipMexRepo} (Feb.\ 2016).

\end{itemize}
    
\section{Professional conferences and seminars (since 2015)}

\begin{itemize}

\item  17 May 2019, ``The dark side of electoral reform: the geography of public good distribution'', IV Taller La ciencia pol\'itica desde M\'exico, Casa de la Marquesa, ITAM.

\item 7 Apr.\ 2019, \href{https://github.com/emagar/cv/blob/master/talks/lucardi-magar-presentation-mpsa2019.pdf}{``The dark side of electoral reform''} (with Adri\'an Lucardi) at the Annual conference of the Midwest Political Science Association, Chicago, IL. 
  
\item 4 Apr.\ 2019, discussant at the ``JSS SESSION: Don't Stop 'Til You Get Enough: Resources and Redistribution,'' Annual conference of the Midwest Political Science Association, Chicago, IL. 
  
\item 7 Dec.\ 2018, \href{https://github.com/emagar/cv/blob/master/talks/cideReelec2018.pdf}{``\emph{Incumbency advantage} en elecciones municipales''}, preparation seminar for a special issue on the 2018 elections for \emph{Pol�tica y Gobierno}, CIDE, Mexico City.
  
\item 5 Oct., 2018, \href{https://github.com/emagar/cv/blob/master/talks/tepjf2018.pdf}{``Del gobierno abierto a la participaci�n efectiva''}, at the Observatorio de participaci�n ciudadana y cultura de la transparencia y la legalidad: E-lecciones en tiempos de internet conference, Universidad Aut�noma de Coahuila, Saltillo.
  
\item 10 Aug., 2018, \href{https://github.com/emagar/cv/blob/master/talks/gelReelec2018.pdf}{``The removal of single-term limits, redistricting, and name recognition: The case of Coahuila's state races''}, IVth meeting of the Grupo de Estudios Legislativos, Mexico City. 
  
\item 10 Aug., 2018, \href{https://github.com/emagar/cv/blob/master/talks/gelUrge2018.pdf}{``Presidents on the Fast Track: Fighting Floor
Amendments with Restrictive Rules''}, IVth meeting of the Grupo de Estudios Legislativos, Mexico City. 
  
\item 25 Jun.\ 2018, \href{https://github.com/emagar/cv/blob/master/talks/magar2018wwc.pdf}{``Mexico's 2018 Congressional elections''}, Mexico Institute, Woodrow Wilson Center, Washington DC.

\item 18 May 2018, \href{https://github.com/emagar/cv/blob/master/talks/taller2018.pdf}{``Name recognition en Coahuila''}, III Taller La ciencia pol�tica desde M�xico, Casa de la Marquesa, ITAM.

\item 28 Feb.\ 2018, ``Experiencias acad�micas'', roundtable at the 25th anniversary of ITAM's political science department ITAM. 
  
\item 9 Nov.\ 2017, \href{https://github.com/emagar/cv/blob/master/talks/uamSelenium2017.pdf}{``Un m�todo para obtener texto del internet y analizarlo''}, 1er Simposio TLH en las ciencias sociales, UAM-Cuajimalpa.

\item 8 Nov.\ 2017, ``Lo que nos espera en 2018'', roundtable at the Alonso Lujambio Prize ceremony, ITAM. 

\item 2 Nov.\ 2017, \href{https://github.com/emagar/cv/blob/master/talks/usMxReelec2017.pdf}{``Another nail in the coffin of Mexican exceptionalism: The removal of (most) single-term limits''} at the Mexico 2018 Election Conference, Center for U.S.--Mexican Studies, UCSD.

\item 14 Sept.\ 2017, \href{https://github.com/emagar/cv/blob/master/talks/itam2017urge.pdf}{``Restrictive rules in the Chilean C�mara: Fighting floor amendments with urgency authority''}, at the Seminario de Investigaci\'on Pol\'itica, ITAM. 
  
\item 2 Sept.\ 2017, discussant at panel ``Legislative Activity and Output,'' annual meeting of the American Political Science Association, San Francisco, CA. 
  
\item 1 Sept.\ 2017, \href{https://github.com/emagar/cv/blob/master/talks/apsa2017urge.pdf}{``Restrictive rules in the Chilean C�mara: Fighting floor amendments with urgency authority''} at the annual meeting of the American Political Science Association, San Francisco, CA. 
  
\item 24 Aug.\ 2017, ``Tensiones esperables entre la reelecci�n consecutiva y las normas electorales,'' Seminario Permanente de An\'alisis de Justicia Electoral en M\'exico, ITAM-CIDE-IIJ.

\item 26 May 2017, \href{https://github.com/emagar/cv/blob/master/talks/urgeTaller2017.pdf}{``Restrictive rules in the Chilean Congress''}, II Taller La ciencia pol\'itica desde M\'exico, Casa de la Marquesa, ITAM.

\item 11 May 2017, ``Mesa de encuestas,'' pre-election polls roundable, ITAM. 

\item 5 Apr.\ 2017, ``Elecciones en Francia: contexto pol�tico e implicaciones macroecon�micas,'' roundtable, ITAM. 

\item 9 Mar.\ 2017, ``La reelecci\'on consecutiva de legisladores y alcaldes,'' at the Mexican constitution's centennial roundtable, ITAM Mexico City.

\item 12 Dec.\ 2016, \href{https://github.com/emagar/cv/blob/master/talks/tepjf2016.pdf}{``Redistritaci\'on automatizada en M\'exico: retos y oportunidades''}, at the Redistritaci\'on electoral, reelecci\'on legislativa y el rol de las cortes: perspectiva comparada desde M\'exico y los Estados Unidos de Am\'erica Seminar, Tribunal Electoral del Poder Judicial de la Federaci\'on, Mexico City.

\item 24 Nov.\ 2016, \href{https://github.com/emagar/cv/blob/master/talks/urgeICP2016.pdf}{``Presidential obstruction of the agenda in Chile's Congress''}, at the Seminar ICP--UC, Santiago, Chile.

\item 18 Nov.\ 2016, \href{https://github.com/emagar/cv/blob/master/talks/2016uconcGel.pdf}{``The crowded plenary: Urgency, logrolls, and the conclusive procedure in Brazil's C\^amara''}, at the III Encuentro GEL--ALACIP, Santiago, Chile.

\item 21 Sep.\ 2016, \href{https://github.com/emagar/cv/blob/master/talks/2016uconc.uam.pdf}{``Saturaci�n del pleno: urgencias del ejecutivo y el poder conclusivo en la C\^amara brasile�a''}, at the Seminario Racionalidad, evoluci�n y aprendizaje, claves del cambio institucional en M\'exico, UAM-Xochimilco and AMEP, Mexico City.

\item 28 May 2016, \href{https://github.com/emagar/cv/blob/master/talks/2016urgeLasa.pdf}{``Presidential obstruction of the agenda in Chile's Congress''}, at the sesquiannual meeting of the Latin American Studies Association, New York, NY. 

\item 13 May 2016, ``Automated redistricting and partisan strategic interaction in Mexico'', Taller La ciencia pol\'itica desde M\'exico, Casa de la Marquesa, ITAM.

\item 29 Oct.\ 2015, \href{https://github.com/emagar/cv/blob/master/talks/2015strategy.colmex.pdf}{``Transparency, automated redistricting, and partisan startegic interaction in Mexico''}, at El Colegio de M\'exico. 

\item 29 Oct.\ 2015, \href{https://github.com/emagar/cv/blob/master/talks/2015pubMap.ogpSummit.pdf}{``From open data to participation in redistricting''}, at the Open Government Partnership Summit, Mexico City. 

\item 25 Sep.\ 2015, \href{https://github.com/emagar/cv/blob/master/talks/2015partisanBias.itam.pdf}{``Measuring malapportionment, gerrymander, and turnout effects in multi-party systems''}, at the Political Science Research Seminar, ITAM. 

\item 4 Sep.\ 2015, \href{https://github.com/emagar/cv/blob/master/talks/2015urgenciaChile.apsa.pdf}{``Presidential obstruction of the agenda in Chile's Congress''}, at the annual meeting of the American Political Science Association, San Francisco, CA. 

\item 2 Sep.\ 2015, \href{https://github.com/emagar/cv/blob/master/talks/2015automatedRedistricting.eip.pdf}{``Transparency, automated redistricting, and partisan startegic interaction in Mexico''}, at the Electoral Integrity Project pre-APSA workshop, San Francisco, CA. 

\item 28 Jul.\ 2015, \href{https://github.com/emagar/cv/blob/master/talks/2015partisanBias.casaMateOaxaca.pdf}{``Measuring malapportionment, gerrymander, and turnout effects in multi-party systems''}, at the Political Economy of Social Choices workshop, Casa Matem\'atica Oaxaca. 

\item 18 Apr.\ 2015, \href{https://github.com/emagar/cv/blob/master/talks/2015bias.mpsa.pdf}{``The effects of malapportionment, turnout, and gerrymandering in Mexico's mixed-member system''}, at the annual meeting of the Midwest Political Science Association, Chicago, IL.

\item 13 Mar.\ 2015, \href{https://github.com/emagar/cv/blob/master/talks/2015bias.uf.pdf}{``Malapportionment, party bias, and responsiveness in Mexico's mixed-member system''}, at the Seminar in Politics, University of Florida, Gainesville. 

\item 9 Feb.\ 2015, \href{https://github.com/emagar/cv/blob/master/talks/2015urge.urbanaFeb.pdf}{``Congress skips turn, again: agenda obstruction in the Chilean Congress''}, at the Comparative Politics Workshop, University of Illinois at Urbana-Campaign. 

%% \item 14 Nov.\ 2014, \href{https://github.com/emagar/cv/blob/master/talks/uhBias2014.pdf}{``Malapportionment and representation: party bias and responsiveness in Mexico''}, at the Analizing Latin American Politics conference, University of Houston. 

%% \item 3 Oct.\ 2014, ``Who calls the tune? The search for constituent effects in the Mexican Congress'', Rice University.

%% \item 31 Aug.\ 2014, ``The Effects of Automated Redistricting and Partisan Strategic Interaction on Representation: The Case of Mexico'', at the annual meeting of the American Political Science Association, Washington, DC.

%% \item 13 Jun.\ 2014, ``The veto as electoral stunt: EITM and test with subnational comparative data'', Executive Politics Conference, Washington University in St.\ Louis.

%% \item 16 May 2014, ``Partidismo entre expertos,'' on-line conference for the Reconexi\'on event, ITAM Mexico City.

%% \item 4 Apr.\ 2013, Discussant at panel 26--5 ``When Candidates Win Elections'' at the annual meeting of the Midwest Political Science Association, Chicago, IL.

%% \item 3 Apr.\ 2013, Discussant at panel 6--5 ``Candidate Selection and Party Entry'' at the annual meeting of the Midwest Political Science Association, Chicago, IL.

%% \item 20 Nov.\ 2013, ``El PAN ante las reformas pendientes,'' National Executive Committee of the Partido Acci\'on Nacional, Mexico City.

%% \item 8 Nov.\ 2013, ``Partidismo entre expertos,'' TEDxITAM, Mexico City.

%% \item 29 Aug.\ 2013, ``Judges' Law: Ideology and coalitions in Mexico's Electoral Tribunal, 1996--2006'' at the annual meeting of the American Political Science Association, Chicago IL.

%% \item 25 Apr.\ 2013, Book presentation, \emph{T\'acticas parlamentarias hispanomexicanas} by Alonso Lujambio and Rafael Estrada Michel, ITAM, Mexico City.

%% \item 17 Apr.\ 2013, ``Dibuja M\'exico: quienquiera podr\'a redistritar con esta herramienta'' at the IV congress of the Asociaci\'on Mexicana de Estudios Parlamentarios, Universidad Iberoamericana, Mexico City.

%% \item 13 Apr.\ 2013, Discussant at panel 6--22 ``Ruling politics'' at the annual meeting of the Midwest Political Science Association, Chicago, IL.

%% \item 12 Apr.\ 2013, ``The Veto as Electoral Stunt: EITM and a Test with Comparative Data'' at the EITM conference of the annual meeting of the Midwest Political Science Association, Chicago, IL.

%% \item 11 Apr.\ 2013, ``Partisanship among the experts: The dynamic party watchdog model of IFE, 1996--2011'' at the annual meeting of the Midwest Political Science Association, Chicago, IL.

%% \item 30 Nov.\ 2012, ``Gubernatorial coattails in Mexican Congressional elections,'' Banco de M\'exico.

%% \item 19 Oct.\ 2012, ``Methodological Considerations for Students of Mexican Legislative Politics: Selection Bias in Roll Call Publications'' at the Legislatures in Mexico and the Americas Conference of the Asociaci\'on Mexicana de Estudios Parlamentarios, Center for U.S.--Mexican Studies, UCSD.

%% \item 17 Oct.\ 2012, ``Why and when is the election referee trustworthy? IFE�s dynamic party watchdog model, 1996�2012'' at the Festschrift in honor of Wayne Cornelius' career, Center for U.S.--Mexican Studies, UCSD.

%% \item 30 Aug.\ 2012, ``The Veto as Electoral Stunt: EITM and a Test with Comparative Data'' at the EITM conference of the annual meeting of the American Political Science Association, New Orleans, LA.

%% \item 9 May 2012, Discussant of the paper ``Misfits, groundbreakers, or plain old politics'' by Ver\'onica Hoyo presented at CIDE's Pol\'itica y Gobierno research seminar, Mexico City.

%% \item 27 Apr.\ 2012, ``Who calls the tune? The search for constituent effects in the Mexican Congress'' (with Federico Est\'evez) at the Workshop on Evidence-Based Approaches to Latin American Constitutionalism, UT--Austin, TX.

%% \item 15 Apr.\ 2012, Discussant at panel 6--19 ``Comparative Legislative Politics and Executive Power'' at the annual meeting of the Midwest Political Science Association, Chicago, IL.

%% \item 13 Apr.\ 2012, ``Are missing votes a problem for research on the Mexican Congress? No'' (with Francisco Cant\'u and Scott Desposato) poster at the annual meeting of the Midwest Political Science Association, Chicago, IL.

%% \item 12 Apr.\ 2012, ``Who calls the tune? Gubernatorial and constituent effects in roll-call voting in the Mexican Congress'' (with Federico Est\'evez) at the annual meeting of the Midwest Political Science Association, Chicago, IL.

%% \item 30 Mar.\ 2012 \href{http://allman.rhon.itam.mx/~ebarrios/marzo2012}{``Preferencias en regulaci\'on electoral: partidismo en el IFE''} at the R users workshop, ITAM, Mexico City.

%% \item 14 Mar.\ 2012, ``Equilibrio de poderes (o y, sin embargo, no se mueve)'' at the Pluralismo y reformas constitucionales en M\'exico: 1997--2012 conference, CIDE--UNDP, Mexico City.

% \item 2 Apr.\ 2011, ``When cartels split: roll call votes and majority factional warfare in the Mexico City Assembly'' at the annual meeting of the Midwest Political Science Association, Chicago, IL.

% \item 1 Apr.\ 2011, Discussant at panel 6--5 ``Institutional Incentives and Policy Provision'' at the annual meeting of the Midwest Political Science Association, Chicago, IL.

% \item 2 Oct.\ 2010, Discussant at the ``Los motivos del votante'' panel of the $4^{th}$ Seminario Encuestas y Elecciones of the Instituto Federal Electoral, Cocoyoc, Morelos.

% \item 23 Sep.\ 2010, ``Partisanship among the experts: the dynamic party watchdog model of IFE, 1996--2010'' at the Electoral Administration in Mexico research workshop organized by the Center for U.S.--Mexican Studies, UCSD, La Jolla, CA.

% \item 3 Sep.\ 2010, ``No self-control: Decentralized Agenda Power and the Dimensional Structure of the Mexican Supreme Court'' at the annual meeting of the American Political Science Association, Washington, DC.

% \item 18 Aug.\ 2010, Discussant of ``A long history of the political manipulation of the supreme courts in Central America and the Caribbean, 1900--2009'' by Andrea Castagnola at the Centro de Estudios y Programas Interamericanos, ITAM, Mexico City.

% \item 26 May.\ 2010, ``A Pandora's box? Activism and ideological drift in the Mexican Supreme Court'' at CIDE's Pol\'itica y Gobierno research seminar, Mexico City.

% \item 22 Apr.\ 2010, ``The constructive veto and parliamentary discipline''
%     at the annual meeting of the Midwest Political Science Association, Chicago, IL.

% \item 15 Apr.\ 2010, Discussant in the panel ``Las propuestas para nuevo equilibrio de poderes'' with presentations by Mar\'ia Amparo Casar and Julio R\'ios Figueroa at the Las agendas de reforma pol\'itica 2010 en perspectiva comparada seminar, CIDE, Mexico City.

% \item 24 Mar.\ 2010, Discussant of the paper ``Dimensionality and party effects: An analysis of legislative preferences in Mexican States'' by Ra\'ul C.\ Gonz\'alez presented at CIDE's Pol\'itica y Gobierno research seminar, Mexico City.

% \item 12 Mar.\ 2010, ``Reflexiones sobre tres propuestas de Reforma Pol\'itica'' at the University Presidents' meeting of the Federaci\'on de Instituciones Mexicanas Particulares de Educaci\'on Superior, ITAM, Mexico City.

% \item 19 Jan.\ 2010, Participant in roundtable ``Agenda Ciudadana y Gobernabilidad: La Reforma Pol\'itica. Di\'alogo
%     Ciudadano'' organized by the Interior Ministry, Centro Cultural Bella \'Epoca, Mexico City.

% \item 3 Dec.\ 2009, ``La Suprema Corte en la transici\'on democr\'atica de M\'exico''
%     paper presented at the Seminario de Investigaci\'on en Ciencia Pol\'itica, ITAM, Mexico City.

% \item 5 Sep.\ 2009, ``Voting and sincerity. Ideal point drift and strategy in a regulatory
%     board'' at the annual meeting of the American Political Science Association, Toronto, Canada.

% \item 7 Mar.\ 2009, ``Activists vs.\ Legalists: The Mexican Supreme Court and its
%     Ideological Battles'' Conference on judicial politics in Latin America, CIDE, Mexico City.

%%\item 3 Dec.\ 2008, ``�Partidismo entre expertos? Un an\'alisis de las votaciones en el
%%    Consejo General del IFE'' ad CIDE's Pol\'itica y Gobierno research seminar, Mexico City.
%%
%%\item 6 Nov.\ 2008, ``Judges' law: ideology and coalitions in Mexico's Election Tribunal 1996--2006''
%%    paper read at the Bayesian Methods in the Social Sciences seminar, CIDE, Mexico City.
%%
%%%Esto debiera de ir en la secci\'on `cursos' o `docencia'...
%%\item 24--5 oct.\ 2008, short course on positive political theory, Masters in Government,
%%    Instituto de Ciencia Pol\'itica, Universidad de la Rep\'ublica, Uruguay.
%%
%%\item 20 oct.\ 2008, ``Factions with clout: presidential cabinet coalition and policy
%%    in the Uruguayan Parliament'' paper read at the $2^{nd}$
%%    Conference of the Asociaci\'on Uruguaya de Ciencia Pol\'itica, Intendencia Municipal de Montevideo,
%%    Uruguay\condcomment{\boolean{longcv}}{ (panel D3 Political Institutions in Uruguay)}.
%%
%%\item 1 Oct.\ 2008, Discussant of the paper ``Sources of competitiveness
%%    in authoritarian elections'' by Andreas Schedler
%%    presented at CIDE's Pol\'itica y Gobierno research seminar, Mexico City.
%%
%%\item 26 Sep.\ 2008, ``Factions with clout: presidential cabinet coalition and policy
%%    in the Uruguayan Parliament'' paper read at the $2^{nd}$
%%    Conference of the Asociaci\'on Mexicana de Estudios Parlamentarios, Puebla, Mexico.
%%
%%\item 6 Aug.\ 2008, ``Raising the odds of legislation: cabinet and bill-sponsor coalition in
%%    Uruguay�s Parliament, 1985�-2005'' paper read at the $4^{th}$
%%    Conference of the Asociaci\'on Latinoamericana de Ciencia Pol\'itica, San Jos\'e, Costa Rica.
%%
%%\item 18 Apr.\ 2008, ``The constructive veto and floor cooperation under open rule:
%%    factional co-sponsoring in Uruguay'', paper presented at the Seminario de
%%    Investigaci\'on en Ciencia Pol\'itica, ITAM, Mexico City.
%%
%%\item 26 Oct.\ 2007, Discussant of the paper ``The legal toolbox
%%    for the different areas of Regional Integration'' by Ram\'on Torrent
%%    (Universidad Aut\'onoma de Barcelona) presented at the Fifth Annual
%%    Conference of the Euro-Latin Study Network on Integration and Trade
%%    of the Inter-American Development Bank, Barcelona,
%%    Spain.
%%
%%\item 6 Sep.\ 2007, ``Electoral stunts in parliamentary government with exogenous timetable: Evidence
%%    from Spain's Autonom\'ias'', paper presented at the $4^{th}$ General Conference
%%    of the European Consortium of Political Research, Pisa, Italy\condcomment{\boolean{longcv}}{ (Political Economy section 18, panel 307)}.
%%
%%\item 3 May 2007, ``\'Exito legislativo del presidente y duraci\'on del tr\'amite parlamentario: el caso de Uruguay,
%%    1985--2000,'' paper presented at the Instituto Interuniversitario de Iberoam\'erica, Universidad de Salamanca, Spain.
%%
%%\item 22 Feb.\ 2007, ``Partisanship in Non-Partisan Electoral Agencies and Democratic Compliance:
%%    Evidence from Mexico's Federal Electoral Institute,'' paper presented at IBEI, Barcelona, Spain.
%%
%%\item 19 Feb.\ 2007, ``Veto Politics in the Americas: A Study of the Legislative Process,'' paper presented at the
%%    Political Economy Workshop of the Government Department, University of Essex, United Kingdom.
%%
%%\condcomment{\boolean{longcv}}{\item 8 Nov.\ 2006,  Presentation of
%%    the book \emph{Mujeres legisladoras en M\'exico: avances, obst\'aculos,
%%    consecuencias y propuestas} to the Equality and Gender Committee of the Chamber of
%%    Deputies, Mexico City.}
%%
%%\item 24 Oct.\ 2006, ``Principales Sistemas de Gobierno en Europa y Am\'erica'' conference for students
%%    of the Masters in Military Administration for National Security and Defense, Colegio de Defensa Nacional, Mexico City.
%%
%%\item 18 Oct.\ 2006,  ``Perspectivas Ejecutivo--Legislativo'' roundtable at the $1^{st}$ Congress of the
%%    Asociaci\'on Mexicana de Estudios Parlamentarios, UNAM, Mexico City.
%%
%%\condcomment{\boolean{longcv}}{\item 17 Oct.\ 2006,  ``Presentaci\'on
%%de Resultados sobre Participaci\'on Pol\'itica y Toma de
%%    Decisiones de las Mujeres en M\'exico'' workshop on research projects funded by the 2004 call for
%%    paper of the Fondo Sectorial de Investigaci\'on y Desarrollo INMUJERES-CONACYT (with Magdalena Huerta).}
%%
%%\item 12 Sep.\ 2006, ``El veto y las relaciones presidente-congreso 2006--2009,'' at the Coaliciones
%%    multipartidistas: condici\'on para un nuevo gobierno conference, organized by
%%    Fundaci\'on Rafael Preciado Hern\'andez, Club France, Mexico City.
%%
%%\item 7 Jun.\ 2006,   ``Equidad y medios en el proceso electoral 2006'' at the Seminar on Institutions
%%    and Processes of Mexico's Democracy, Facultad de
%%    Ciencias Pol\'iticas y Sociales, UNAM, Mexico City.
%%
%%\item 22 Apr.\ 2006,  Discussant in panel 3--11 ``The formation of national party systems'', at the annual
%%    meeting of the Midwest Political Science Association, Chicago, IL.
%%
%%\item 21 Apr.\ 2006, ``Can parties reduce agency slack through the judges?'',
%%    paper read at the annual meeting of the Midwest Political Science
%%    Association, Chicago, IL.
%%
%%\condcomment{\boolean{longcv}}{\item 15 Nov.\ 2005, ``Relaciones
%%    presidente-congreso: panorama 2006--2009,'' talk given at Banamex's
%%    Department of Sociopolitical Studies, Mexico City.}
%%
%%\item 2 Sep.\ 2005, ``Are Non-Partisan Technocrats the Best Party Watchdogs Money Can Buy?  An
%%    Examination of Mexico's Instituto Federal Electoral'' poster presented at the annual meeting of the American
%%    Political Science Association, Washington, DC.
%%
%%\item 9 Apr.\ 2005, ``Party Sponsorship and Voting Behavior in Small Committees: Mexico's Instituto
%%    Federal Electoral'' paper read at the annual meeting of the Midwest Political Science Association, Chicago, IL.
%%
%%\item 4--5 Mar.\ 2005, ``Gubernatorial Coattails and Mexican Congressional Elections since 1979,''
%%    paper read at the Conference What Kind of Democracy Has Mexico? The
%%    Evolution of Presidentialism and Federalism at the Center for US-Mexican Studies,
%%    UCSD.
%%
%%\item 7 Oct.\ 2004, ``The Circuitous Path to Democracy: Legislative Control of the Bureaucracy in
%%    Presidencial Systems, the Case of Mexico'' at the conference on Democratic Accountability and Rule of Law in Mexico, Stanford
%%    University.
%%
%%\condcomment{\boolean{longcv}}{\item 1 Oct.\ 2004, ``El efecto de arrastre de gobernadores en elecciones de diputados federales de
%%    M\'exico, 1979--2003'' paper read at the Fridays Political Science Research
%%    Workshop, ITAM, Mexico City.}
%%
%%\item 30 Sep.\ 2004, ``El efecto de arrastre de gobernadores en elecciones de diputados federales de
%%    M\'exico, 1979--2003'' paper read at the $2^{nd}$ Congress of the Asociaci\'on
%%    Latinoamericana de Ciencia Pol\'itica, Mexico City.
%%
%%\condcomment{\boolean{longcv}}{\item 15 Jul.\ 2004,  Chair at the
%%session on ``Institutions in Presidential Democracy: Comparative
%%    Perspective from the Americas'' at the conference on Democratic Institutions in Latin
%%    America: Implications for Mexico's Evolving Democracy, Center for US-Mexican
%%    Studies, UCSD.}
%%
%%\item 19 Apr.\ 2004,  Discussant in panel 3--17 ``The new Mexican Political Economy'' at the annual
%%    meeting of the Midwest Political Science Association, Chicago, IL.
%%
%%\item 6--9 May 2003, Presented the manuscript of my book \emph{Bully Pulpits} at the Political Institutions and
%%    Public Choice seminar, Dept. of Political Science, Michigan State University.
%%
%%\item 7 Apr.\ 2003, ``The Pulse of Uruguayan Politics: Factions, Elections, and Veto Incidence, 1985--2002,'' paper presented at the annual meeting of the Midwest Political Science
%%    Association, Chicago, IL.
%%
%%\condcomment{\boolean{longcv}}{\item 7 Apr.\ 2003, ``Theoretical and
%%Empirical Models of Post-Election Coalition Bargaining in
%%    presidential democracies: Ecuador and Uruguay in comparative perspective,''
%%    poster presented with Andr\'es Mej\'ia Acosta at the annual meeting of the Midwest
%%    Political Science Association, Chicago, IL.}
%%
%%\condcomment{\boolean{longcv}}{\item 12 Feb.\ 2003,  Participant in
%%the roundtable ``The Mexican Left: Resurgence or Default Option,''
%%    Center for US-Mexican Studies, UCSD.
%%
%%\item 6 Sep.\ 2002,  Discussant of the paper ``Majoritarian Electoral Systems and Consumer Power:
%%    Price-Level Evidence from the OECD Countries'' by Mark Andreas Kayser
%%    and Ronald Rogowski, at the Research Workshop in Political Science, ITAM, Mexico City.
%%
%%\item Jun.--Jul.\ 2002, Participant in the Empirical Implications of Theoretical Models Summer Institute
%%    at the Center for Basic Research in the Social Sciences, Harvard University.
%%
%%\item 12 Apr.\ 2002,   Discussant of the paper ``Presidential Vetoes in the Early Republic'' by Nolan
%%    McCarty at the Research Workshop in Political Science, ITAM, Mexico City.}
%%
%%\item 30 Nov.\ 2001,   Discussant of the manuscript \emph{Lawmaking and Bureaucratic Discretion in
%%    Modern Democracies} by John D.\ Huber and Charles D.\ Shipan, at the CIDE's Pol\'itica y Gobierno research seminar, Mexico City.
%%
%%\item 6 Sep.\ 2001,  ``The Paradox of the Veto in Mexico (1917--1997)'' paper presented at
%%    the annual meeting of the Latin American Studies Association, Washington, DC.
%%
%%\item 31 Aug.\ 2001,  ``Making Sound out of Fury: The Posturing Use of Vetoes in Chile and in
%%    Mexico,'' paper read at the annual meeting of the American Political
%%    Science Association, San Francisco, CA.
%%
%%\item 30 Aug.\ 2001,  ``The Elusive Authority of Argentina's Congress: Decrees, Statutes, and Veto
%%    Incidence, 1983--1994,'' paper read at the annual meeting of the American
%%    Political Science Association, San Francisco, CA.
%%
%%\item 20 Apr.\ 2001, ``The Incidence of Executive Vetoes in Comparative Perspective: Position-Taking
%%    and Uncertainty in US State Governments, 1983--1993,'' paper read at the annual meeting of the Midwest Political Science Association, Chicago, IL.
%%
%%\condcomment{\boolean{longcv}}{\item 16 Mar.\ 2001, ``El position-taking y la incertidumbre como fuentes de vetos del ejecutivo en los gobiernos estatales de los EEUU'' paper read at the Research Workshop in Political Science, ITAM, Mexico City.
%%
%%\item 13 Oct.\ 2000,  Discussant of the paper ``Democratization and the Economy in Mexico:
%%    Equilibrium (PRI) Hegemony and its Demise,'' by Alberto D\'iaz Cayeros, Beatriz Magaloni and Barry R.\ Weingast, at the Research Workshop in Political Science, ITAM, Mexico City.
%%
%%\item 17 Mar.\ 2000, ``Veto Bargaining and Coalition Formation: A Theory of Presidential Policymaking
%%    with Application to Venezuela,'' paper read at the annual meeting of the
%%    Latin American Studies Association, Miami, FL.}
%%
%%\item 15 Oct.\ 1999, ``La incidencia de vetos en perspectiva comparativa: el caso de los gobiernos
%%    estatales de los EEUU,'' paper presented at the Centro de Estudios para
%%    el Desarrollo Institucional, Universidad de San Andr\'es, Argentina.
%%
%%\condcomment{\boolean{longcv}}{\item 23 Ago.\ 1999, ``Veto Incidence in Comparative Perspective: The Case of US State
%%    Governments,'' paper presented at the Department of Political Science,
%%    UCSD.}
%%
%%\item 7--14 Jul.\ 1999,    Discussion of the final manuscript of \emph{Political Economics}, by Torsten Persson
%%    y Guido Tabellini, at the ``Encounters with Authors'' seminar of the Center for
%%    Basic Research in Social Science, Harvard University.
%%
%%\item 9 May 1999, \href{http://sites.google.com/site/emagar/MagarUCLA-2.PDF}{``Patterns of
%%    Executive-Legislative Conflict in Latin America \& the US''}
%%    paper read at the $1^{st}$ International Graduate Student Retreat for
%%    Comparative Research, organized by the Society for Comparative Research and the
%%    Center for Comparative Social Analysis, UCLA.
%%
%%\item 4 Sep.\ 1998, ``The Deadlock of Democracy Revisited: A model of Executive-Legislative
%%    Relations in Separation-of-Power Regimes,'' paper read at the
%%    annual meeting of the American Political Science Association, Boston, MA.

\end{itemize}




\section{Referee/reviewer}

\subsection{Journal articles}

\emph{American Journal of Political Science} (2000, 2004, 2004, 2006),
\emph{American Political Science Review} (2003, 2004, 2004, 2007, 11/2018),
\emph{Applied Geography} (11/2017),
\emph{Comparative Political Studies} (1998, 2009, 3/2018),
\emph{Econom\'ia Mexicana} Nueva \'Epoca (2008, 2011),
\emph{EconoQuantum} (2012, 2012),
\emph{Foro Internacional} (2012),
\emph{Journal of Legislative Studies} (2010),
\emph{Journal of Politics in Latin America} (2014),
\emph{Journal of Politics} (2011, 2015, 2016, 1/2017, 6/2019, 10/2020),
\emph{Latin American Politics and Society} (2015),
\emph{Legislative Studies Quarterly} (2013, 2014),
\emph{Revista Sociol\'ogica} de la UAM (2003),
\emph{Party Politics} (2008),
\emph{Perfiles Latinoamericanos} de FLACSO--M\'exico (2009),
\emph{Pol\'itica y Gobierno} (2001, 2002, 2010, 2011, 2012, 2013, 2015, 5/2018, 2/2020),
\emph{Political Analysis} (2/2021),
\emph{Political Research Quarterly} (2013),
\emph{Revista Mexicana de Sociolog\'ia} (2001, 2005),
\emph{Social Choice and Welfare} (2012),
\emph{World Politics} (2004).

\subsection{Books}

El Colegio de M\'exico (6/2019), FLACSO--Sede M\'exico (2003, 2004), Fondo de Cultura Econ\'omica (2013), Oxford University Press (2008).

\subsection{Book chapters}

Asociaci\'on Latinoamericana de Ciencia Pol\'itica (\href{https://github.com/emagar/cv/blob/master/evidence/dictamenALACIP2012n1.pdf}{2012}, 2012)


\subsection{Research projects}

\begin{itemize}

\item Fondo de Cooperaci�n M�xico--Chile of the Agencia Mexicana de Cooperaci�n Internacional para el Desarrollo and Chile's Agencia de Cooperaci�n Internacional para el Desarrollo (2/2020).
  
\item CONICYT, Chile (9/2019).
  
\item Consejo Nacional de Ciencia y Tecnolog\'ia, Mexico (2009, 2009)

\end{itemize}





\section{Academic and Research Committees}

\begin{itemize}

\item 2020, \href{https://www.mpsanet.org/Professional-Development/Awards-Call-for-Nominations/Award-Recipients-2020}{Evan Ringquist Award} for best paper in the topic of political institutions selection committee, Midwest Political Science Association.

\item 2016, Jewell/Loewenberg award selection committee, Legislative Studies Section, American Political Science Association.

\item 2014, Fulbright-Garc\'ia Robles Scholarship Award committee member, (COMEXUS, Mexico City).

\item 2014--18, Editorial Board \href{https://github.com/emagar/cv/blob/master/evidence/EricMagarComiteEditorialPyG.2014.pdf}{member}, \emph{Pol\'itica y Gobierno}, (CIDE, Mexico City).

\item 2013,   \href{http://mpsanet.org/Awards/2014AwardRecipients/tabid/873/Default.aspx}{Kellogg/Notre Dame Award selection committee}, Midwest Political Science Association.

\item 2012--14,  \href{https://github.com/emagar/cv/blob/master/evidence/comiteFlacso2012.pdf}{Member} of the Academic Coordination Committee for the Masters in Government and Public Affairs, Facultad Latinoamericana de Ciencias Sociales Mexico.

\item 2010--12, Editorial Board member, Facultad Latinoamericana de Ciencias Sociales Mexico.

\item 2009--16,  Research Associate, Center for U.S.-Mexican Studies (UC, San Diego.)

\item 2006--,  Editorial board member, \emph{La Gaceta de Ciencia Pol\'itica} (ITAM, Mexico City.)

\item 2001--2004,  Member of the Academic Coordination Committee for the Masters in Government and Public Affairs, Facultad Latinoamericana de Ciencias Sociales Mexico.

\end{itemize}





%\condcomment{\boolean{longcv}}{

\section{Other committees}

\begin{itemize}

\item 2012,   Comit\'e T\'ecnico de Especialistas de la Comisi\'on Temporal de Debates, member of the advisory board to the committee in charge of televised presidential candidate debates, Instituto Federal Electoral, Mexico City.

\item 2006,   \emph{Reforma} newspaper, member of the Research Board, Mexico City.

\item 2005--2006,  Tu Rock es Votar, member of the Advisory Board, Mexico City.

\end{itemize}
%}



%\condcomment{\boolean{longcv}}{

\section{Professional association membership}

\begin{itemize}

\item American Political Science Association (since 1998)

\item Asociaci\'on Latinoamericana de Ciencia Pol\'itica (since 2002)

\item Asociaci\'on Mexicana de Estudios Parlamentarios (since 2011)

\item Midwest Political Science Association (since at least 2001)

\item Public Choice Society (since 2006)

\end{itemize}
%}



%\condcomment{\boolean{longcv}}{

\section{Recent academic events organized}

\begin{itemize}

\item 12 Nov.\ 2020, ``Mesa post-electoral 2020'', Mexico city (via zoom), ITAM.

\item 17 May 2019, ``IV Taller La Ciencia Pol\'itica desde M\'exico'', Mexico City, ITAM \href{https://emagar.github.io/intro-taller/}{$\rightarrow$ web page}.

\item 9--10 Aug., 2018, ``IVth meeting of ALACIP's Grupo de Estudios Legislativos'' with Adri�n Lucardi and Juan Pablo Micozzi, Mexico City, ITAM-WashU-LSQ \href{https://github.com/emagar/cv/blob/master/evidence/GEL2018Program03eng.pdf}{$\rightarrow$ program}.
  
\item 18 May 2018, ``III Taller La Ciencia Pol\'itica desde M\'exico'', Mexico City, ITAM \href{https://emagar.github.io/intro-taller/}{$\rightarrow$ web page}.

\item 26 May 2017, ``II Taller La Ciencia Pol\'itica desde M\'exico'', Mexico City, ITAM \href{https://emagar.github.io/intro-taller/}{$\rightarrow$ web page}.

\item 13 May 2016, ``Taller La Ciencia Pol\'itica desde M\'exico'', Mexico City, ITAM \href{https://emagar.github.io/intro-taller/}{$\rightarrow$ web page}.

\item 13--17 Jun.\ 2011, ``Latent variable estimation workshop'' co-taught with Professor Guillermo Rosas from Washington University St.\ Louis, Mexico City, ITAM.

\item 6--7 Nov.\ 2008, ``Bayesian methods in the social sciences'' seminar, co-organized
    by the Political Science Departments of Washington University in St.\ Louis, CIDE
    and ITAM, Mexico City, at CIDE and ITAM.

% \item 8 Jun.\ 2006,  ``2006: las encuestas en la mira,'' seminar, co-sponsored by WAPOR-
%     Mexico, the Center for US-Mexican Studies and ITAM's Political Science Department,
%     Casa de California, Chimalistac, Mexico City.

% \item Oct.\ 2004,  Dr. Hans-J�rgen Beerfeltz's keynote address on the German electoral system,
%     co-organized with the Friedrich Naumann Foundation and the Instituto Federal
%     Electoral, ITAM.

% \item 2001--2003,  Seminario de Investigaci\'on Pol\'itica de los Viernes,
%     Political Science Department, ITAM.

\end{itemize}
%}


%\condcomment{\boolean{longcv}}{
\section{Courses taught}

\subsection{Graduate level}

\begin{itemize}
 \item La separaci\'on de poderes en Am\'erica Latina y Estados Unidos, Judicial Training Program for Argentine Judges, Washington University in St.\ Louis (with Guillermo Rosas, 2 hours: S2015).  
 \item Teor\'ia pol\'itica positiva y estudios legislativos en Latinoam\'erica, master and Ph.D. in political science of the Instituto de Ciencia Pol\'itica, Universidad de la Rep\'ublica, Uruguay (6 hours: F2008).
 \item Estudios de Am\'erica Latina, M\'aster en Relaciones Internacionales IBEI, Barcelona (16 hours: F2007).
 \item Introducci\'on al An\'alisis Pol\'itico, mtr\'ia.\ en pol\'iticas p\'ublicas ITAM (33 hours: W2006).
 \item An\'alisis pol\'itico institucional, mtr\'ia.\ en pol\'iticas p\'ublicas ITAM (33 hours: F2000, F2001, F2002, F2003)
\end{itemize}

\subsection{Undergraduate level}

\begin{itemize}

 \item Latent variable estimation workshop (15 hours: S2011).
 \item  Pol\'itica Comparada II (comparative political institutions), lic.\ en ciencia pol\'itica ITAM (48 hours: S2014, F2013, F2012, S2012, F2011, F2010, S2010, S2009, F2008, S2008, F2006, F2005, F2004, F2003, S2003, F2002).
 \item Pol\'itica Comparada III (Latin America and the U.S.), lic.\ en ciencia pol\'itica ITAM (48 hours: F2001, F2000).
 \item Elecci\'on P\'ublica III (legislative bargaining), lic.\ en ciencia pol\'itica ITAM (48 hours: S2014, S2013, F2012, S2012, F2011, S2011, F2010, S2010, F2009, S2009, F2008, S2008, F2006, S2006, S2005, S2004, S2003).
 \item Seminario de Investigaci\'on Pol\'itica (thesis workshop), lic.\ en ciencia pol\'itica ITAM (48 hours: F2013, S2013, S2011, F2009, S2004, S2002, S2001).

\end{itemize} 
 
%}


\section{Students}

I have advised theses for one Ph.D.\ student, four M.A.\ students, and twenty-one B.A.\ students. 

% \subsection{PhD thesis}

%     \begin{itemize}

%     \item Dec.\ 2009, Mario Alejandro Torrico Ter\'an,
%     ``Factores explicativos y dimensiones de la estabilidad pol\'itica: un estudio mundial,''
%     Doctorado en Ciencias Sociales, FLACSO--M\'exico.

% \condcomment{\boolean{longcv}}{
%         \begin{list}{--}{\topsep=-10pt \parsep=0pt \parskip=0pt} %% itemize sin espacios de m\'as para premios

%         \item Received Excellent Grade with Recommendation to Publish at the oral defense.

%         \end{list}
% }

%     \end{itemize}


% \subsection{Masters theses}

%     \begin{itemize}

%     \item Sep.\ 2007, Tom\'as Pinheiro Fiori, ``The Political Economy of Hydrocarbons in Latin America,''
%     Master in International Relations, IBEI, Barcelona.

%     \item Sep.\ 2007, Silvia G\'andara Berger, ``Aproximaci\'on a la situaci\'on actual de las relaciones civiles-militares en Guatemala desde una perspectiva democr\'atica,'' Master in International Relations, IBEI, Barcelona.

%     \item Nov.\ 2004,  Octael Nieto V\'azquez, ``La estructura, el gasto y el desempe\~no de las organizaciones
%     culturales federales en M\'exico, 1990--2003,'' Maestr\'ia en Sociolog\'ia Pol\'itica,
%     Instituto Mora, Mexico.

%     \item Jul.\ 2004,  Ren\'e Ram\'irez Gallegos, ``Pseudo-salida, silencio y �deslealtad?: entre la inacci\'on
%     colectiva, la desigualdad de bienestar y la pobreza de capacidades,'' Maestr\'ia en
%     Gobierno y Asuntos P\'ublicos, FLACSO-Mexico.

%  \condcomment{\boolean{longcv}}{
%         \begin{list}{--}{\topsep=-10pt \parsep=0pt \parskip=0pt} %% itemize sin espacios de m\'as para premios

%         \item Received Excellent Grade at the oral defense.

%         \end{list}
% }

%     \end{itemize}

% \subsection{BA theses}

% \begin{itemize}

    %\item Mar.\ 2019, Julia Madrazo Clavijo ``Diferencias etnoling{\''u}{\'i}sticas y fronteras electorales: el caso de la consulta ind\'igena en Chiapas'' ITAM.

%  \condcomment{\boolean{longcv}}{
%         \begin{list}{--}{\topsep=-10pt \parsep=0pt \parskip=0pt} %% itemize sin espacios de m\'as para premios

%         \item Received Special Mention at the oral defense.

%         \end{list}
% }

    %\item Nov.\ 2016, Luis Daniel Cubr\'ia Trujillo ``Determinantes  y efectos del calendario electoral en M\'exico: un estudio subnacional'' ITAM.

    % \item Aug.\ 2014, Adriana Mael S\'anchez L\'opez ``La conformaci\'on del bloque obregonista en la XXVII Legislatura y su relaci�n con el gobierno de Carranza'' ITAM.

    % \item Feb.\ 2014, Luis Fernando Godoy Rueda ``Reelecci\'on en la C\'amara de Diputados, 1917--1933: federalismo y ambici\'on pol\'itica'' ITAM.

    %     \begin{list}{--}{\topsep=-10pt \parsep=0pt \parskip=0pt} %% itemize sin espacios de m\'as para premios

%         \item Received Special Mention at the oral defense.

%     \item Jun.\ 2011, Luis Everdy Mej\'ia L\'opez ``Procesos de intermediaci\'on pol\'itica en la elecci\'on presidencial mexicana del 2006'' ITAM.

%  \condcomment{\boolean{longcv}}{
%         \begin{list}{--}{\topsep=-10pt \parsep=0pt \parskip=0pt} %% itemize sin espacios de m\'as para premios

%         \item Received Special Mention at the oral defense.

%         \end{list}
% }

%     \item Oct.\ 2010 Jaime Arredondo ``El subsidio municipal para la seguridad p\'ublica: an\'alisis de la f\'ormula de elegibilidad'' ITAM.

% \condcomment{\boolean{longcv}}{
%         \begin{list}{--}{\topsep=-10pt \parsep=0pt \parskip=0pt} %% itemize sin espacios de m\'as para premios

%         \item Received Special Mention at the oral defense.

%         \end{list}
% }

%     \item Aug.\ 2010, C\'esar Montiel Olea ``Repensando los poderes presidenciales: un estudio del veto total y parcial en M\'exico, 1997--2009'' ITAM.

% \condcomment{\boolean{longcv}}{
%         \begin{list}{--}{\topsep=-10pt \parsep=0pt \parskip=0pt} %% itemize sin espacios de m\'as para premios

%         \item Received Special Mention at the oral defense.

%         \end{list}
% }

%     \item Oct.\ 2009, Roberto Ponce L\'opez ``La geograf\'ia del voto en M\'exico: organizaci\'on partidista en las
%     secciones electorales, 1997--2003'' ITAM.

% \condcomment{\boolean{longcv}}{
%         \begin{list}{--}{\topsep=-10pt \parsep=0pt \parskip=0pt} %% itemize sin espacios de m\'as para premios

%         \item Received Special Mention at the oral defense.

%         \end{list}
% }
%     \item Mar.\ 2009, Isabel Zapata Morales ``Las cuotas de g\'enero en los congresos locales
%     mexicanos: una aproximaci\'on emp\'irica a un debate normativo'' ITAM.

% \condcomment{\boolean{longcv}}{
%         \begin{list}{--}{\topsep=-10pt \parsep=0pt \parskip=0pt} %% itemize sin espacios de m\'as para premios

%         \item Received Special Mention at the oral defense.

%         \end{list}
% }

%     \item Jan.\ 2009, Luis Esteban Islas Bacilio ``El Sindicato Nacional de Trabajadores
%     de la Educaci\'on (SNTE) y la calidad de la educaci\'on en M\'exico'' ITAM.

%     \item Nov.\ 2008, Gabriel Alfredo Peto ``Participaci\'on electoral, concurrencia y movilizaci\'on en
%     las elecciones municipales en M\'exico, 1991--2006'' ITAM.

%     \item Jun.\ 2008,  Ana Quesada Reynoso, ``Noche de gatos pardos: autoridad, se\~nales y
%     aprobaci\'on en el ITAM'' ITAM.

%     \item Dec.\ 2007, Alejandro Trelles Yarza and Diego Mart\'inez Cant\'u
%     ``Fronteras electorales: aportaciones del modelo de redistritaci\'on mexicano al estado de
%     California'' ITAM.

%         \begin{list}{--}{\topsep=-10pt \parsep=0pt \parskip=0pt} %% itemize sin espacios de m\'as para premios

% \condcomment{\boolean{longcv}}{
%         \item Received Special Mention at the oral defense;
% }

%         \item received $1^{st}$ place of Ex-ITAM's Research Prize
%         2007.

%         \end{list}

%     \item Jun.\ 2006,  Magdalena Huerta Garc\'ia, ``�Qu\'e hace falta para que haya m\'as mujeres
%     en la C\'amara de Diputados? Determinantes del acceso de mujeres a candidaturas y a
%     puestos de elecci\'on en la C\'amara de Diputados, 1994--2006'' ITAM.

% \condcomment{\boolean{longcv}}{
%         \begin{list}{--}{\topsep=-10pt \parsep=0pt \parskip=0pt} %% itemize sin espacios de m\'as para premios

%         \item  Received Special Mention at the oral defense.

%         \end{list}
% }

%     \item Jun.\ 2006,  Gabriela Enrigue Gonz\'alez, ``La Suprema Corte y la desigualdad en el acceso a
%     la justicia: �a qui\'en sirve el amparo en materia fiscal?'' ITAM.

%         \begin{list}{--}{\topsep=-10pt \parsep=0pt \parskip=0pt} %% itemize sin espacios de m\'as para premios

% \condcomment{\boolean{longcv}}{
%         \item Received Special Mention at the oral defense;
% }

%         \item received $2^{nd}$ prize of the Premio Banamex de Econom\'ia 2006;

%         \item received $1^{st}$ place of Ex-ITAM's Research Prize 2006;

% \condcomment{\boolean{longcv}}{
%         \item received Honors Mention of the Premio Nacional Tlaca\'elel de Consultor\'ia Econ\'omica
%         2008.
% }

%         \end{list}

%     \item Jun.\ 2006,  Yunuel Cruz Guerrero, ``Del simbolismo a la equidad, el impacto del g\'enero
%     en los partidos pol\'iticos y la ALDF: representaci\'on en comisiones y prioridades
%     legislativas'' ITAM.

%     \item Jun.\ 2006,  Leticia Lowenberg Cruz, ``El sesgo de g\'enero en la nominaci\'on de candidatos y la
%     elecci\'on de diputados locales para la Asamblea Legislativa del Distrito Federal,
%     1994--2003'' ITAM.

%     \item Oct.\ 2004,  Gildardo Zafra Amescua, ``Efectos de arrastre en elecciones concurrentes en
%     M\'exico'' ITAM.

%     \item Sep.\ 2004,  Manlio Guti\'errez V\'azquez, ``La coordinaci\'on de las acciones legislativas del
%     poder ejecutivo en M\'exico: una propuesta de reorganizaci\'on'' ITAM.

%     \item May.\ 2003,  Jimena Otero Zorrilla, ``�Gerrymandering en M\'exico? La geograf\'ia pol\'itica federal,
%     1994--1997'' ITAM.

%         \begin{list}{--}{\topsep=-10pt \parsep=0pt \parskip=0pt} %% itemize sin espacios de m\'as para premios

% \condcomment{\boolean{longcv}}{
%         \item Received Special Mention at the oral defense;
% }

%         \item received Special Mention of Ex-ITAM's Research Prize
%         2003.

%         \end{list}

%     \item May.\ 2003,  Cristina Rivas Ochoa and Diana Toledo Figueroa, ``Comisiones: la brecha
%     institucional'' ITAM.

%     \end{itemize}


\section{Consulting}
\begin{itemize}

\item 2020, Electoral Boundaries Specialist of the
    Electoral Observation Mission of the United States,
    Organization of American States (Washington D.C.). 
  
\item 2008--09, ``The Political Economy of Fiscal Reforms in Latin America: The
    Case of Mexico'' with Vidal Romero and Jeffrey F.\ Timmons,
    background paper for the Inter-American Development Bank \emph{The
    Political Economy of Fiscal Reforms in Latin America} by Mark
    Hallerberg.

\item 2006, ``Changing Patterns of Governance in
    Mexican States: What do They Tell Us About Future Directions of National
    Politics?'' with Vidal Romero, background paper for the Institutional Governability Report
    \emph{Democratic Governance in Mexico: Beyond State Capture and Social Polarization} by
    Philip Keefer (Mexico City: World Bank-INAP-CIDE, 2007).

\end{itemize}


\condcomment{\boolean{longcv}}{
\section{Radio and TV commentary}
\begin{itemize}

\item 20 May 2010, Reforma pol\'itica: Iniciativas de ley, ``Voces de la Democracia'' program of the Instituto Federal Electoral, Canal del Congreso, Mexico City.

\item 19 May 2010, Reforma pol\'itica: Iniciativas de ley, ``Voces de la Democracia'' program of the Instituto Federal Electoral, Radio UNAM, Mexico City (\textsc{fm96.1mh}z y \textsc{am860kh}z).

\item 18 Nov.\ 2008, Roundtable on the 2009 budget, ``Punto y coma'', Canal 40, Mexico City.

\item 25 Aug.\ 1995, ``Madrazo contra Zedillo'' Radio UNAM, Mexico City (\textsc{fm96.1mh}z y \textsc{am860kh}z).

\item 14 Jul.\ 1995, ``Impuesto a la gasolina'' Radio UNAM, Mexico City (\textsc{fm96.1mh}z y \textsc{am860kh}z).

\item 7 Jul.\ 1995, ``Masacre en Aguas Blancas, Gro.'' Radio UNAM, Mexico City (\textsc{fm96.1mh}z y \textsc{am860kh}z).

\item 30 Jun.\ 1995, ``Relevan a Esteban Moctezuma'' Radio UNAM, Mexico City (\textsc{fm96.1mh}z y \textsc{am860kh}z).

\end{itemize}
}

\condcomment{\boolean{longcv}}{

\section{Languages}

Spanish, French, English. }


\section{References}
\begin{itemize}

\item \textsc{Gary Cox}, Professor of Political Science, Stanford. \\
+1(650)723-4278, \url{gwcox@stanford.edu}.

\item \textsc{Paul Drake},
Professor of Political Science, UCSD. \\
+1(858)534-6868, \url{pdrake@weber.ucsd.edu}.

\item \textsc{Mathew McCubbins},
Professor of Political Science and Law, Duke University. \\
+1(919)660-4324, \url{mathew.mccubbins@duke.edu}.

\item \textsc{Wayne Cornelius},
CCIS Director Emeritus and Theodore Gildred Distinguished Professor Emeritus of Political Science, UCSD. \\
+1(885)822-4447, \url{wcorneli@weber.ucsd.edu}.

\item \textsc{Federico Est\'evez},
Professor de Ciencia Pol\'itica, ITAM. \\
+52(55)5628-4000x3702, \url{festevez@itam.mx}.

% \item \textsc{Beatriz Magaloni},
% Professor of Political Science, Stanford \\
% +1(650)724-5949, \url{magaloni@stanford.edu}.

\item \textsc{Elisabeth Gerber},
Jack L.\ Walker, Jr.\ Professor of Public Policy, Gerald R.\ Ford School of Public Policy, Univ.\ of Michigan. \\
+1(734)647-4004, \url{egerber@umich.edu}.

\end{itemize}

\end{document}


\newpage

\begin{center}
\textbf{\large{Abstract of my dissertation}} \\
``Bully pulpits: Posturing, bargaining, and polarization in the legislative process of the Americas''

by

Eric Magar \\
Doctor of Philosophy in Political Science \\
University of California, San Diego, 2001 \\
Professors Gary W. Cox and Paul W. Drake, Chairs
\end{center}

I study the legislative process under the presidential form of
separation of power (\textsc{sop}).  Three themes converge in my
study. First I ask why vetoes occur in SOP systems?  In the
US the conventional answer is that vetoes are tactical
maneuvers of normal democratic politics.  Vetoes are not harbingers
of imminent democratic breakdown, nor evidence of gridlock.  Vetoes
are part of everyday policy bargaining.

\medskip Second, I ask whether the U.S. is an exception among presidential
systems.  Does this logic apply to other cases such as those in
Latin America?  Is it applicable to sub-national levels of
government?  If vetoes are bargaining tactics then their use should
vary with the institutional context.

\medskip Third, I inquire about different explanations of veto incidence. Are
vetoes the product of incomplete information in bargaining hence
mistakes by somewhat shortsighted politicians?  Are vetoes
bargaining ploys meant to build a reputation of toughness in light
of asymmetric information?  Or are vetoes better understood as
publicity stunts, maneuvers aimed towards the public in search of
support?  Incomplete information and position-taking compete to
explain veto incidence.

\medskip This dissertation reviews strands of literature that share the topic
(\textsc{sop}) but not the audience (Anglo- and Latin-Americanists),
offering a bridge over the gap.  I extend a formal model of
executive-legislative bargaining (the agenda-setter model) to
include a posturing motive for politicians to provoke vetoes.  The
extended model is put to a test covering state-level governments of
the US.  I then expand my research to Latin America,
studying veto incidence in Argentina and three case studies of
inter-branch bargaining in Argentina, Chile, and Mexico.

\medskip Studies of presidentialism in the last decade have shifted attention
from the dysfunctionality of gridlock (e.g. Linz, Sundquist) to a
wide range of tactical maneuvers that SOP offers politicians (e.g.
Kernell, Cameron).  I follow the recent literature in analyzing a
richer, livelier breed of executive-legislative relations.  Bully
pulpit veto politics involve the presidency as well as Congress.



\condcomment{\boolean{longcv}}{
\bigskip
\textbf{\large{}}
\begin{itemize}
\item
\end{itemize}
}
